% !TEX root = ../main.tex

\chapter{Abstract}

\Glspl{MOF} are a class of hybrid porous materials
which has been the focus of much scientific interest since their
discovery in the late 20\textsuperscript{th} century. These
coordination polymers consist of three-dimensional networks of 
metal nodes interlinked by polytopic organic molecules. Their 
potential stems from the possible rational design
of \glspl{MOF} for highly efficient or unique applications. 
With judicious choice of building blocks, the framework can be used for
carbon dioxide capture from air, hydrogen storage or regioselective 
catalysis. Furthermore, taking advantage of specific properties such 
as flexibility, magnetism or photoelectrics may even result in 
their use in drug delivery, micro mechanical devices,
sensors and optoelectronics.

However, their properties also introduce significant difficulty
in \gls{MOF} adsorption characterisation.
Structural defects, crystal size, shaping procedure
and the aforementioned flexible behaviour can lead to irreproducibility 
in isotherms. The first step to ensuring consistent results
is to ensure that a common framework of processing methods is 
established. In this thesis, the creation of an open-source codebase
is detailed, which is intended to standardize the processing of 
isotherms. Using this framework, high throughput processing of more
than 18 000 isotherms is used to explore the scale of uncertainty 
present in published adsorption data.
The focus turns towards using complementary methods to extend the
high throughput methodology through advanced characterisation.
Direct measurement of the differential enthalpy of adsorption
using \textit{in situ} microcalorimetry is shown to be an excellent
avenue of obtaining further insight into the contribution of material, 
guest-host and host-host interactions to the overall energetics of
adsorption.

Together, these methods are used to study some of the sources 
of the variability of \gls{MOF}s, and quantify their effect.
First, the impact of structural defects is investigated, through an
alternative post-synthetic method of missing linker/cluster generation
in the prototypical UiO-66(Zr) \gls{MOF}. The processing of 
materials for their use in an industrial environment through 
shaping is another potential source of performance modification,
which is here studied as the effect of wet granulation on
three topical \gls{MOF}s (UiO-66(Zr), MIL-100(Fe) and MIL-127(Fe)). 
Finally, counterintuitive behaviours intrinsic to
``soft'' porous crystals are detailed, where the structure itself 
is responsible for fluctuation in adsorption isotherms. Here,
a fundamental study on a copper paddlewheel based material, DUT-49(Cu)
yields know-how on the source of adsorption induced compliance
and its tunability through structural modification.

\pagebreak

\chapter{Résumé}

Les réseaux métallo-organiques (MOF) sont une classe de matériaux poreux
hybrides qui ont suscité un grand intérêt scientifique depuis leur 
découverte à la fin du 20ème siècle. Ces polymères de coordination sont 
constitués de réseaux tridimensionnels de nœuds métalliques interconnectés 
par des molécules organiques polytopiques. L'intérêt pour ces matériaux 
provient de la conception rationnelle possible des MOF pour des 
applications hautement efficaces ou uniques.
Leur flexibilité, magnétisme ou effet photoélectrique
peut même conduire à leurs utilisations dans l'adsorption et stockage du
gaz, catalyse, délivrance de médicaments, des dispositifs micro-mécaniques
ou des capteurs.

Cependant, leurs propriétés uniques introduisent également des difficultés
significatives dans la caractérisation par adsorption de gaz, avec les 
défauts structurels, la taille des cristaux, la procédure de mise 
en forme et le comportement flexible susmentionné comme un exemple. 
La première étape pour obtenir des résultats reproductibles consiste 
à s'assurer qu'un cadre commun de méthodes de traitement 
soit établi. Dans cette thèse, la création d'un code source 
libre est détaillé, 
pour standardiser le traitement des isothermes. En utilisant ce code, 
un traitement à haut débit de plus de 18 000 isothermes est 
utilisé pour explorer l'échelle d'incertitude présente dans les 
données publiées sur l'adsorption dans les matériaux poreux.
Ensuite, l'accent sera mis sur l'utilisation de méthodes complémentaires 
pour étendre la méthodologie à haut débit. La mesure directe de 
l'enthalpie différentielle de l'adsorption en utilisant la
microcalorimétrie \textit{in situ} s'avère être un excellent 
moyen d'obtenir la contribution des interactions particulières sur
l'énergie d'adsorption.

Ensemble, ces méthodes peuvent être utilisées pour étudier les sources 
d'incertitude des MOF. On étudie d’abord l’impact des défauts
structurels au moyen d’une méthode post-synthétique alternative de
génération de linker/cluster manquant dans l'UiO-66(Zr). 
Le traitement des matériaux pour leurs utilisations dans 
un environnement industriel par façonnage est une autre source
potentielle de modification des performances, étudié ici sous l’effet
de la granulation par voie humide sur trois MOF topiques (UiO-66(Zr),
MIL-100(Fe) et MIL-127(Fe)). 
Enfin, les comportements contre-intuitifs intrinsèques aux cristaux poreux 
«souples» sont étudiés, où la structure elle-même est responsable de la 
fluctuation dans les isothermes d'adsorption. Ici, une étude 
fondamentale sur un matériau flexible DUT-49 (Cu), apporte 
des informations sur la source de
flexibilité induite par adsorption et son changeabilité 
par modification structurelle.
