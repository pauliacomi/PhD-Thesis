% !TEX root = ../main.tex

\chapter{Abstract}

\gls{MOF} are a class of hybrid porous materials
which has been the focus of much scientific interest since their
discovery in the late 20\textsuperscript{th} century. These
coordination polymers consist of three-dimensional networks of 
metal nodes interlinked by polytopic organic molecules. The 
interest in these materials stems from the possible rational design
of MOFs for highly efficient or unique applications. 
With judicious choice of building blocks, the framework can be used for
carbon dioxide capture from air, hydrogen storage or regioselective 
catalysis. Taking advantage of specific properties of porous 
coordination polymers such as flexibility, magnetism or photoelectrics
may even result in their use in drug delivery, micro mechanical devices,
sensors or optoelectronics.

However, their unique properties also introduce significant difficulty
in \gls{MOF} characterisation through gas adsorption. 
Structural defects, crystal size, shaping procedure
and the aforementioned flexible behaviour can lead to variability in
adsorption results. In this thesis large scale processing of 
isotherms is first used to explore the scale of uncertainty 
present in porous material adsorption data. The high throughput
methodology is then extended through the use of \textit{in situ} calorimetry
as an avenue to gain further insight into the contribution of material, 
guest-host and host-host interactions to the overall energetics of
adsorption. Together, these two methods can be used to quantify the effect
of structural defects in MOFs, material shaping with different binders 
and even yield fundamental know-how of how adsorption induces compliance
in flexible materials.

\pagebreak

\chapter{Résumé}

Les réseaux métallo-organiques (MOF) sont une classe de matériaux poreux hybrides qui ont suscité un grand intérêt scientifique depuis leur découverte à la fin du 20ème siècle. Ces polymères de coordination sont constitués de réseaux tridimensionnels de nœuds métalliques interconnectés par des molécules organiques polytopiques. L'intérêt pour ces matériaux provient de la conception rationnelle possible des MOF pour des applications hautement efficaces ou uniques. Avec un choix judicieux des blocs de construction, le cadre peut être utilisé pour la capture du dioxyde de carbone dans l'air, le stockage de l'hydrogène ou la catalyse régiosélective. Tirer parti des propriétés spécifiques des polymères de coordination poreux tels que la flexibilité, le magnétisme ou la photoélectrique peut même conduire à leur utilisation dans la délivrance de médicaments, des dispositifs micro-mécaniques, des capteurs ou en optoélectronique.

Cependant, leurs propriétés uniques introduisent également des difficultés importantes dans la caractérisation des MOF par adsorption de gaz. Les défauts structurels, la taille des cristaux, la procédure de mise en forme et le comportement flexible susmentionné peuvent entraîner une variabilité des résultats de l'adsorption. Dans cette thèse, le traitement à grande échelle des isothermes est d'abord utilisé pour explorer l'échelle d'incertitude présente dans les données d'adsorption des matériaux poreux. La méthodologie à haut débit est ensuite étendue à l’utilisation de la calorimétrie in situ comme moyen de mieux comprendre la contribution des interactions matériau-matériau, matériau-hôte et hôte-hôte à l’énergie globale de l’adsorption. Ensemble, ces deux méthodes peuvent être utilisées pour quantifier l'effet des défauts structurels dans les MOF, la mise en forme des matériaux avec différents liants et même un savoir-faire fondamental sur la manière dont l'adsorption induit la conformité dans les matériaux flexibles.
