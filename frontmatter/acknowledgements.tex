% !TEX root = ../main.tex

\chapter{Acknowledgements}

First and foremost I would like to thank my supervisor, Dr.\ Philip Llewellyn,
to whom I am most grateful for welcoming me into the lab and for all the help
along the way. Your door was always open for anything from rebuilding
an apparatus to finding some hidden spanner. I enjoyed our discussions 
both about science and creme eggs, MOF porosity and Minas Tirith, adsorption 
methodology and lightsabers. I can honestly say that you have shaped
me into the scientist and person I am now, and for that you have my gratitude.

To the members of the jury, I am thankful for agreeing to examine 
my work and for all the constructive feedback on my thesis.
Thank you Dr.\ Flor Siperstein for your impeccable teaching in university which
influenced me to follow the road to this PhD. My thanks also go to Dr.\ Rob Ameloot,
who has warmly received me in his group in Leuven for my secondment, and 
to his team, which immersed me into the arcane art of MOF synthesis.
Finally, I am grateful to Prof.\ Stefan Kaskel for a wonderfully interesting
collaboration and for welcoming me in the Dresden group for a
short while.

There is no \textit{I} in \textit{science} (from a conceptual, not orthographic
point of view), therefore I must extend my thanks to all the collaborators
who have joined me in this three year journey. From the DEFNET project,
I'd like to mention Prof.\ Roland Fischer and Dr.\ Harish Parala for the 
scientific and administration organisation, as without them this network
would not have precipitated, as well as all the groups which are part 
of the initiative around the European Union. To the youngest members of 
my Marie Curie project: dear defects, I was happy to be part of our little
adventures throughout the world. Carlos, Colin and, in particular, João,
thank you for going out of your way to receive me in your midst in Leuven
(and teaching me to enjoy belgian beer). "Grazie" to the italian crew 
Stefano (still looking forward for the next road trip), Chiara ``Cartarelli''
and Roberto ``The Alchemist''. Also thanks to Ganna for the philosophical 
conversations, to Emily for her boundless optimism, to Rifan for sharing
the burdens of the lab, to Anna for always being ready to help,
to Penghu for his enthusiasm, to Miguel for his straight-faced humour,
to Hilmar for his german stoicism and to Mujahid for his everpresent smile.
I could not have asked for better collaborators and friends.

Fruitful collaborations are often found outside initial domains. 
To this end, many thanks to Simon and Jack and the rest of the Dresden 
group for providing an avenue for interesting fundamental research.
I am waiting to hear your names in a Nobel prize sometime in the
future. No pressure. Other collaborations may be destined to bear fruit
through friendship, so I hope we will meet yet again Eder, preferably 
with some \textit{jamón}.

Several others have provided valuable training and input during the 
course of my PhD, and here I have to thank Dr. Dan Siderius of NIST
as well as Prof.\ Guillaume Maurin, Prof.\ Christian Serre, and 
Prof.\ Jong-San Chang and their groups
for the productive discussions and professional input.

Dans MADIREL, j'ai eu de la chance d'avoire une équipe magnifique.
Ici, je voudrais remercier Vanessa, pour son aide précieuse en sciences
et en dehors du laboratoire.
Je remercie egalement Emily et Sandrine pour leur aide et patience.
Je jure que ce n'est pas moi qui brise tout et j'espère que vous allez
trouver vos échantillons \cancel{cachées} rangées dans les boites 
et les armoires. Je suis honore d'avoir travaille dans le meme groupe 
que Jean Rouquerol, a qui je suis redevable de tout ce que j'ai pu apprendre
au cours de nos discussions intéressantes. Je tiens a remercier 
Renaud, Isabelle et Bogdan d'avoir toujours pris leur temps pour m'aider,
dans le domain practique ou scientifique. 
Finalement, un grand merci a l'equipe administrative et technique,
en particulier Joelle et Maryline, les magiciens de bureaucratie, qui 
m'ont permis de concentrer sur la science.

Life-work balance might be slightly skewed when in a PhD, but my
co-workers made one feel like the other.
All the best to Damien, may we meet for yet another coffee break 
one day, and to Pierre-Henry and his military-grade good nature.
To Kasia, which I am sure will never find a more annoying
office-mate, I wish all the best after our PhD journey comes to 
an end.
And finally a thanks to past, present and just starting MADIREL
associates: Virginie, Eva, Ania, Luis, Ege, Michele, Cintya, Behnam, Helena,
Ritu, Khac-Long, Sergei, Nico, Christophe, Girish, Alex, Yasemin, Ephrem, Hailong
and all whom I may have missed by name.

Shout out to the members of the happenstance international PhD crew of Marseille,
Joanna and Shuxian, as well as to older (but still very much current) friends
who have been at my side throughout the sojourn: Mihai, Naomi, Andreea, Bianca
and Till. For André, memes may die but friendships last forever.
To Pingping, for your wholehearted support,
\begin{CJK*}{UTF8}{gbsn}
    谢谢, 亲爱的。
\end{CJK*}

Și, la final, trebuie să mulțumesc familiei mele, părinți, bunici și 
unchi, care a crezut în mine și mi-a fost alături prin infinitele 
capricii ale unui doctorat.