% !TEX root = ../../main.tex

% Reset graphics to the current folder
\graphicspath{ {\thisch/figures/} }

\chapter*{Introduction to this thesis}\label{intro}
\addcontentsline{toc}{chapter}{Introduction to this thesis}

There's no doubt that the major advances of humanity in the last 
century have been defined by \textit{big thinking about tiny things}.
As Richard Feynman famously stated in his 1959 lecture, 
``there's plenty of room at the bottom'', the pursuit
of miniaturisation led to the development of nanotechnology and
the ever-increasing computational density of transistors. But we are
still a long way from mastery of this field.
The most intricate machine currently known to mankind, the human body
itself, which relies exclusively on the assembly of simple aminoacids into proteins 
to perform all its functions, is orders of magnitude away
from currently achievable complexity. There are two schools of thought
regarding the fabrication of nanoscale materials:
the top-down approach (using specialised tools to modify existing 
structures) or the bottom-up approach (self assembly of small 
components into more complex configurations).
The two methods intersect as more precise tools are required for the 
former and larger structures for the latter. 
While current advancements are no small feat, it stands to reason
that the next century of discoveries ought to lie not only in
furthering the understanding of the nanoscale, but also in the
extension of the nanoscopic world to the macroscopic one. 

A special class of materials whose properties are an interplay 
of multiple length scales are porous media. Characterised by
a network of voids through a solid matrix, they are 
commonly found in both natural (rocks, bone, wood) and man made 
(cement, ceramic) settings. The substantial interface between the
void and the solid lends itself to their use as adsorbents 
for gas storage and separation or as catalysts. Compounds such as 
activated carbons, activated aluminas, silica gels and aluminosilicates
have been extensively used in industry for over 70 years.
\glspl{MOF} are relative newcomers to the 
family of porous materials, having only been named and 
recognised as an independent class of adsorbents at the end of
the 20\textsuperscript{th} century but
have quickly generated scientific interest, due to their
potential applications as hyper-specific adsorbents and catalysts.
Their promise lies in the capability of rational design of 
porous crystal structures, starting from relatively simple building
blocks of metal nodes and organic ``struts''. \gls{MOF} synthesis can often
be as facile as dissolution of a metal salt and corresponding
polydentate organic linker in a solvent, followed by the slow 
growth of crystals as they precipitate out of solution.
Their highly tunable nature may allow previously unattainable
targets to be achieved. For example, energy-efficient capture of 
\ce{CO2} from waste streams and at low partial pressures is of
crucial interest  from an environmental point of view, as \ce{CO2} 
is considered the main contributor to global warming. Improvements may 
be devised in other parts of the energy cycle, like the storage of low
energy density potential fuels such as hydrogen and methane, serving as an 
alternative to energetically intensive compression and liquefaction.

However, the transition of \glspl{MOF} towards large scale application 
has not been as smooth as originally envisaged. The first materials
were mired by poor thermal and chemical stability, and the structures 
often took the form of two-dimensional sheets or chains rather than
true crystal analogues. Furthermore, removal of the solvent was likely
to cause structural collapse and loss of crystalinity. It was not until
the second generation of \glspl{MOF} when the factors contributing to 
the strength of the metal-linker bond were better understood, that
significantly more stable three-dimensional networks could be synthesised.
Part of this generation are the widely known 
HKUST-1~\cite{chuiChemicallyFunctionalizableNanoporous1999}, 
MOF-5~\cite{rosiHydrogenStorageMicroporous2003},
UiO-66~\cite{cavkaNewZirconiumInorganic2008},
MIL-100~\cite{fereyChromiumTerephthalateBasedSolid2005}
and the ZIF~\cite{huangLigandDirectedStrategyZeoliteType2006} 
family of zeolite analogues, some which are now available from chemical
suppliers.

More recently, the unique capabilities of metal organic frameworks
have begun to be the focus of research, leading to the so-called third 
generation of materials in the classification of 
Kitagawa~\cite{kondoMicroporousMaterialsConstructed2000}.
Previously detrimental properties, such as low bulk modulus
can be turned into advantages, using framework mobility to 
work at the interface between ``soft'' and ``hard'' matter.
The defining trait of such frameworks is their ability to 
undergo well defined structural changes upon adsorption,
compression or other stimuli.
It was also found~\cite{valenzanoDisclosingComplexStructure2011} 
that the properties of \glspl{MOF} are highly influenced
through defects in the crystal structure. These defects 
can be exploited as another parameter in the design of 
porous coordination compounds, leading to control over 
accessible surface area and selectivity towards certain molecules
as well as generation of Lewis sites for catalytic applications.

The extensive building blocks available for the construction 
of \glspl{MOF} lead to an unexpected conundrum, as hundreds of thousands
of potential structures can be generated even when using 
a restricted set of vertices and 
edges~\cite{wilmerLargescaleScreeningHypothetical2012}.
The question arises whether the required properties
for a specific application can be predicted from a hypothetical 
structure or found amongst the thousands of possibilities. 
If so, could the desired topology and connectivity be
obtained synthetically? And finally, in the resulting material,
what is the influence of factors beyond the ideal unit cell 
structure such as defects, crystal size, linker mobility, 
external stimuli, shaping and many others?
The answers to the first question may be found through a 
computational approach, with several efforts already underway 
to generate a database of existing crystal 
structures~\cite{chungComputationReadyExperimentalMetal2014, %
moghadamDevelopmentCambridgeStructural2017}
or predict the generation of new 
ones~\cite{raccugliaMachinelearningassistedMaterialsDiscovery2016}.
Rational and high throughput synthesis
methods~\cite{stockSynthesisMetalOrganicFrameworks2012} can 
evaluate the feasibility of predicted materials and discover 
new structures. However, it is the latter question which is
the subject of this thesis, as it pertains to the transition
and scale-up of \glspl{MOF} from a laboratory environment to 
large scale applications.
As such, the main objective of this work is to 
\textit{explore the sources of variability in metal organic 
frameworks and the use of advanced characterisation methods
to elucidate contribution of extrinsic or intrinsic factors}. 

\subsection*{Outline}

This thesis is split into a total of five chapters, each
connected through the common theme of characterisation 
using gas adsorption of metal organic frameworks. 
Each chapter is mostly 
self-contained and takes a similar form to a publication,
loosely based around an introduction, description of the 
relevant concepts, explanation of experimental methods and
a discussion of the obtained results. 
The topic of the chapters 
shifts subtly from large scale processing to fundamental
science from \autoref{pyg} and \autoref{dut}.
The work initially started as an attempt to characterise and
discover new structure-property relationships in \glspl{MOF}.
As it will be shown in \autoref{pyg}, inherent variability
in available adsorption data means that accurate laboratory
obtained data is crucial for such a purpose. A pure adsorption
methodology is not enough to fully understand the material,
with \autoref{calo} introducing \textit{in situ} calorimetry
as a complementary method for adsorbent characterisation.
The following chapters take an in-depth look at the effect of 
various factors on the ideal \gls{MOF} structure.
This approach provides a good overview of the source of 
variability in \glspl{MOF}, starting from their synthesis
(\autoref{def}), the post-processing required for industrial
adaptation (\autoref{shaping}) and culminating with 
unique and counterintuitive properties intrinsic to a subset 
of flexible materials (\autoref{dut}). Finally, appendices are 
dedicated to the description of common characterisation methods
(\autoref{appx:char}), the samples used in this work and their 
synthesis (\autoref{appx:synthesis}) and a cursory evaluation of 
error margins (\autoref{appx:errors}), as well as an appendix with 
all data in each chapter.

Chapter~\ref{pyg} presents the creation of a Python-based code for
common isotherm processing tasks, such as specific surface area 
determination, pore size distribution calculation or multicomponent 
adsorption prediction. This package, now published as an 
open-source program, is hoped to become the \textit{de facto}
method for obtaining such structural information from isotherms,
and is uniquely suited for the treatment of ``big data''.
An example of its capabilities is outlined in the processing of 
a dataset of 26 000 isotherms from the \gls{NIST} adsorption database,
which reveals interesting patterns in the repeatability and reproducibility
of adsorption isotherms on \glspl{MOF}. The chapter also serves as
an in-depth introduction of both the commonplace and state of the art adsorption methodology.

Chapter~\ref{calo} extends the analysis of porous materials through
the direct measurement of the enthalpy of adsorption through 
\textit{in situ} microcalorimetry. After a brief description 
of the energetic components of adsorption on surfaces
and in pores, the methods for determining the enthalpy of 
adsorption and the insights it provides into the contribution
of different interaction types to the overall energetics are 
discussed. Several example datasets are then discussed, including 
one on Zr Fumarate, an isoreticular analogue of UiO-66. The combined
approach highlights the potential of this \gls{MOF} for propane/propylene
separation.

In \autoref{def}, the presence and consequences of structural defects
in \glspl{MOF} are studied. An alternative method for 
post-synthesis generation of missing linker and missing cluster 
defects in the 
UiO-66(Zr) framework through leaching in an acid solution is presented.
Four solvents (\gls{DMF}, \ce{H2O}, methanol and \gls{DMSO}) are used in 
conjunction with different acids (formic, acetic, trifluoroacetic and
benzoic acid) at various concentrations, to obtain a map of 
the influence of factors such as solvent polarity and \gls{pKa}
on the defective nature of the resulting material.

Chapter~\ref{shaping} looks further downstream the path of 
\gls{MOF} industrialization, analysing the impact of shaping on the
original properties of the sample. Three topical \glspl{MOF}, chosen
due to their stability and well-studied adsorption behaviour
are shaped using a wet granulation method. Multiple gas probes 
in conjunction with microcalorimetry are used to highlight the 
changes in physical and chemical nature. Adsorption of water and 
methanol vapour are also performed in order to assess the impact on 
the hydrophobicity of the resulting pellets.

Finally, \autoref{dut} is dedicated to an in-depth characterisation
of DUT-49 a flexible \gls{MOF} with a counterintuitive adsorption behaviour.
In this material, structural contraction upon adsorption leads 
to a sudden expulsion of gas from within the shrinking pores.
Ambient and low temperature calorimetry is used to obtain 
insight into the temperature and adsorbate dependence of 
the transition mechanics, as well as to evaluate
the influence of framework modification on the capability of 
compliance. This concluding chapter foreshadows exciting future
prospects for \glspl{MOF}, where their innate flexible nature
can be used for novel stimuli-dependent applications.

\subsection*{Overarching project goal, location and funding}

This thesis and the work leading to the presented results 
was funded by the European union through Horizon 2020 Marie Curie
Actions Initial Training Network (H2020-MSCA-ITN-2014)
project DEFNET (DEFect NETwork materials science and engineering), 
Grant Agreement Number 641887. It is the first integrated 
European Training Network (ETN) at the intersection
of chemistry, physics and engineering which deals with the structural and
functional complexity of molecular network materials such as \glspl{MOF}. 
By connecting synthesis, materials characterization,
theory and materials simulation with application and technology, 
DEFNET investigates local and long range defects, heterogeneity, 
disorder and correlated phenomena in porous coordination polymers.

The second and fourth workpackages, relating to characterisation and
applications of defective \glspl{MOF} has as beneficiary the MADIREL 
laboratory in Marseille, France, which is incorporated in the
Aix-Marseille University, and part of the Centre
National de la Recherche Scientifique (CNRS). The MADIREL 
laboratory is focused on areas of work involving high surface
areas and large interfaces between phases of the materials in 
study: the capture and separation of gas using metal-organic 
frameworks, adsorption at the liquid-solid interface,
the study of transport within condensed matter and electrochemistry.

In the context of this project, the main objective my work is to
investigate the impact of structural defects on the adsorption
properties of \glspl{MOF}, relating it to catalysis, gas capture
and separation. Collaboration with KU Leuven, a project partner,
has allowed joint expertise on both synthesis and characterisation
of defective \glspl{MOF}. Partnerships outside DEFNET beneficiaries have
allowed avenues of defect generation such as shaping
(collaboration with KRICT) and the study of novel materials 
(TU Dresden) to be explored. Finally, association with NIST allowed 
for an integration of their database with the high throughput 
processing methodology herein developed.

\pagebreak

\let\oldaddcontentsline\addcontentsline% Store \addcontentsline
\renewcommand{\addcontentsline}[3]{}% Make \addcontentsline a no-op
\bibliographystyle{unsrtnat}
\bibliography{\thisch/biblio/bib}
\let\addcontentsline\oldaddcontentsline% Restore \addcontentsline