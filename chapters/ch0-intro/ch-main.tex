% !TEX root = ../../main.tex

% Reset graphics to the current folder
\graphicspath{ {\thisch/figures/} }

\chapter*{Introduction to this thesis}\label{intro}
\addcontentsline{toc}{chapter}{Introduction to this thesis}

MOFs are relative newcomers to the family of porous materials,
having only been named and classified as an independent class
of adsorbents at the end of the 20\textsuperscript{th} century.
They have been quick to generate scientific interest, due to their
potential applications as hyper-specific adsorbents and catalysts.
Their promise lies in the capability of rational design of 
complex macroscopic structures, starting from relatively simple building
blocks. Their synthesis can often be as facile as dissolution 
of a metal salt and polydentate organic linker in a solvent, followed
by precipitation of crystals from solution\footnote{method not guaranteed}.

However, their transition towards large scale application has not been
as smooth as originally envisaged. The first generation of materials
was mired by poor thermal and chemical stability, and the structures 
often took the form of two-dimensional sheets or chains rather than
true crystal analogues. It was not until the second MOF generation 
where the factors contributing to the stability of the metal-linker
bond were better understood, and significantly more stable three-dimensional
networks could be synthesised.
Part of this family are the widely known HKUST-1, MOF-5, UiO-66, MIL-100
and the ZIF zeolite analogues.

More recently, the unique capabilities of metal organic frameworks
have begun to be the focus of research, in the so-called third 
generation of materials. 


This work initially started as an attempt to characterise and discover 
new structure-property relationships in porous materials. 

\subsection*{Outline}

This thesis is split into a total of five chapters, each
connected through the common theme of characterisation 
using gas adsorption in metal organic frameworks. 
Each chapter is mostly 
self-contained and takes a similar form to a publication,
loosely based around an introduction, description of the 
relevant concepts, explanation of experimental methods and
a discussion of the obtained results. The topic of the chapters 
shifts subtly from large scale processing to fundamental
studies between \autoref{pyg} and \autoref{dut}. It was felt 
that this provides a good overview of the source of 
variability in MOFs, starting from their synthesis
(\autoref{def}), the post-processing required for industrial
adaptation (\autoref{shaping}) and culminating with 
unique and counterintuitive properties intrinsic to a subset 
of flexible materials (\autoref{dut}).

Chapter~\ref{pyg} presents the creation of a Python-based code for
common isotherm processing tasks, such as specific surface area 
determination, pore size distribution calculation or multicomponent 
adsorption prediction. This package, now published as an 
open-source program, is hoped to become the \textit{de facto}
method for obtaining such structural information from isotherms. 
An example of its capabilities is outlined in the processing of 
a large dataset from the NIST adsorption database, which reveals 
interesting patterns in the repeatability and reproducibility of 
adsorption isotherms on MOFs. The chapter also serves as an in-depth
introduction of both the common and state of the art adsorption methodology.

After the 

\subsection*{Overarching project goal, location and funding}

This thesis and the work leading to these results 
was funded by the European union through Horizon 2020 Marie Curie
Actions Initial Training Network (H2020-MSCA-ITN-2014)
project DEFNET, Grant Agreement Number 641887.

DEFNET (DEFect NETwork materials science and engineering) is
the first integrated European Training Network (ETN) at the intersection
of chemistry, physics and engineering which deals with the structural and
functional complexity of molecular network materials such as metal organic 
frameworks (MOFs). It provides a unique research and training platform for
early stage researchers (ESRs) in chemistry, materials science and engineering.
By connecting synthesis, materials characterization, theory and materials simulation
with application and technology. DEFNET will investigate local and long
range defects, heterogeneity, disorder and correlated phenomena in
porous coordination polymers.
Understanding and controlling defect structures is essential for advanced
functionality suited for catalysis, gas capture, and separation. DEFNET
materials based on MOFs hold promise for innovative functionalities which
cannot be achieved by other materials.

The second and fourth workpackages, relating to characterisation and applications 
of defective MOFs had as beneficiary the MADIREL laboratory in Marseille, France,
which is incorporated in the Aix-Marseille University, and part of the Centre
National de la Recherce Scientifique (CNRS). The MADIREL laboratory is focused
on areas of work involving high surface areas and large interfaces 
between phases of the materials in study: the capture and separation of
gas using metal-organic frameworks, adsorption at the liquid-solid interface,
the study of transport within condensed matter and electrochemistry.