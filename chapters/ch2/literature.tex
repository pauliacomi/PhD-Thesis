
\section{Literature}


\subsubsection{Isosteric heat}

The isosteric heats are calculated from experimental data using the Clausius-Clapeyron
equation as the starting point:

\begin{equation}
    \Big( \frac{\partial \ln P}{\partial T} \Big)_{n_a} = -\frac{\Delta H_{ads}}{R T^2}
\end{equation}

Where \(\Delta H_{ads}\) is the enthalpy of adsorption. In order to approximate the
partial differential, two or more isotherms are measured at different temperatures. The
assumption made is that the heat of adsorption does not vary in the temperature range
chosen. Therefore, the isosteric heat of adsorption can be calculated by using the pressures
at which the loading is identical using the following equation for each point:

\begin{equation}
    \Delta H_{ads} = - R \frac{\partial \ln P}{\partial 1 / T}
\end{equation}

and plotting the values of \(\ln P\) against \(1 / T\) we should obtain a straight
line with a slope of \(- \Delta H_{ads} / R\).

The isosteric heat is sensitive to the differences in pressure between the two isotherms. If
the isotherms measured are too close together, the error margin will increase.
The method also assumes that enthalpy of adsorption does not vary with temperature. If the
variation is large for the system in question, the isosteric heat calculation will give
unrealistic values.

Even with carefully measured experimental data, there are two assumptions used in deriving
the Clausius-Clapeyron equation: an ideal bulk gas phase and a negligible adsorbed phase
molar volume. These have a significant effect on the calculated isosteric heats of adsorption,
especially at high relative pressures and for heavy adsorbates.
