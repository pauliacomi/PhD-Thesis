% !TEX root = ../../main.tex

\section{Conclusion}

Having previously detailed the kind of information that isotherm
processing affords, this chapter extends the characterisation
of porous materials through a measurement of the enthalpy of 
adsorption.

This characterisation can be performed either 
directly, using a highly sensitive microcalorimetric setup, or 
indirectly, through the application of the isosteric method on
isotherms recorded at different temperatures. A short analysis
on the applicability and agreement of each method is performed,
noting that they are complementary with respect to different
adsorption regimes and conditions. Calorimetric methods
are shown to be particularly suitable to low coverage adsorption 
regarding non-specific interactions. Furthermore, the continuous method
of adsorbate introduction is seen to be extremely sensitive
to minute changes in the measured system due to its high resolution.

An example analysis of a reference microporous carbon is
used to benchmark useful methods of data processing. The 
enthalpy of adsorption at zero loading and initial Henry's 
constant are two parameters which are shown to be a good 
indicator of the interaction of the material with the 
adsorbed molecules.

Finally, the same methods are applied to a metal-organic
framework of recent interest and used to highlight 
a potentially interesting separation of propylene and 
propane. The material is seen to be more selective for 
propane at low pressures, a desirable feature for 
industrial processes. A further study on the viability 
of such a separation is recommended, combining simulations of 
ideal isotherms and diffusivity and experimental breakthrough
curves.

With the in-depth characterisation afforded by combined adsorption
measurements and microcalorimetry, the following chapters attempt
to delve deeper into some of the sources of variability in metal
organic framework isotherms.

\FloatBarrier{}
\pagebreak