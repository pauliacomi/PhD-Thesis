% !TEX root = ../../main.tex

\section{Method}

Experimentally, two methods are widely used for determining the 
enthalpy of adsorption. The first relies on the applicability of the 
Clausius-Clapeyron equation to two or more isotherms measured 
at different temperatures. The second method relies on a 
direct measurement of the heat produced during adsorption 
using a calorimeter.

There are also ways of determining the enthalpy of adsorption
from computer simulation methods. The accuracy of these methods
depends on the chosen model of interaction with the surface.

\subsubsection{Isosteric heat}

The isosteric heats are calculated from experimental data using the
Clausius-Clapeyron equation as the starting point:

\begin{equation}
    \Big( \frac{\partial \ln P}{\partial T} \Big)_{n_a} = -\frac{\Delta H_{ads}}{R T^2}
\end{equation}

Where \(\Delta H_{ads}\) is the isosteric heat of adsorption.
If it is assumed that the adsorption enthalpy does not vary with 
temperature, the equation can be rearranged to:

\begin{equation}
    \Delta H_{ads} = - R \frac{\partial \ln P}{\partial 1 / T}
\end{equation}

In order to approximate the partial differential, two or more
isotherms are measured at different temperatures. 
Afterwards the isosteric heat of adsorption can be calculated
by using the pressures at which the loading is identical using the 
rearranged equation. By plotting the values of \(\ln P\) against
\(1 / T\) we should obtain a straight line with a slope
of \(- \Delta H_{ads} / R\).

The isosteric heat is sensitive to the differences in pressure between
the two isotherms. If the isotherms measured are too close together, 
the error margin will increase. The method also assumes that enthalpy 
of adsorption does not vary with temperature. If the
variation is large for the system in question, the isosteric
heat calculation will give unrealistic values.

Even with carefully measured experimental data, there are two 
assumptions used in deriving the Clausius-Clapeyron equation: 
an ideal bulk gas phase and a negligible adsorbed phase
molar volume. These have a significant effect on the calculated 
isosteric heats of adsorption, especially at high relative pressures 
and for heavy adsorbates.

\subsubsection{Microcalorimetry}


The enthalpy of adsorption as measured by a calorimeter is a 
sum of all heat contributions during the experiment. They can 
be split into guest-host contributions and host-host contributions.
The guest-host contributions include 

In this thesis, combined isotherms and enthalpy of adsorption
measurements were made experimentally using a Tian-Calvet type
microcalorimeter coupled with a home-made manometric gas dosing
system~\cite{llewellynGasAdsorptionMicrocalorimetry2005}. 
This apparatus allows the simultaneous measurement 
of the adsorption isotherm and the corresponding differential 
enthalpies. Gas is introduced into the system using a step-by-step
method and each dose is allowed to stabilize in a
reference volume before being brought into contact with the
adsorbent located in the microcalorimeter. The introduction of the
adsorbate to the sample is accompanied by an exothermic thermal signal,
measured by the thermopiles of the microcalorimeter. The peak in the
calorimetric signal is integrated over time 
to give the total energy released during this adsorption step.
At low coverage the error in the signal can be estimated to around 
\( \pm \) \SI{0.2} {\kilo\joule\per\mol}. Around \SI{0.4}{\gram} of 
sample is used in each experiment. 
For each injection of gas, equilibrium was assumed to have
been reached after 90 minutes. This was confirmed by the return
of the calorimetric signal to its baseline (<\SI{5}{\micro\watt}).


\subsubsection{pyGAPS}