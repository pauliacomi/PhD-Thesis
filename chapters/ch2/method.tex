% !TEX root = ../../main.tex

\section{Method}

Gas adsorption isotherms and enthalpies were measured experimentally using a Tian-Calvet type 
microcalorimeter coupled with a home-made manometric gas dosing
system~\cite{LlewellynGasadsorptionmicrocalorimetry2005}. 
This apparatus allows the simultaneous measurement 
of the adsorption isotherm and the corresponding differential enthalpies. Gas is
introduced into the system using a step-by-step method and each dose is allowed to stabilize in a
reference volume before being brought into contact with the
adsorbent located in the microcalorimeter. The introduction of the adsorbate to the sample is 
accompanied by an exothermic thermal signal, measured by the
thermopiles of the microcalorimeter. The peak in the calorimetric signal is integrated over time 
to give the total energy released during this adsorption step.
At low coverage the error in the signal can be estimated to around \( \pm \) \SI{0.2}
{\kilo\joule\per\mol}. Around \SI{0.4}{\gram} of sample is used in each experiment. 
For each injection of gas, equilibrium was assumed to have
been reached after 90 minutes. This was confirmed by the return of the calorimetric signal to its
baseline (<\SI{5}{\micro\watt}). The gases used for the
adsorption were obtained from Air Liquide and were of minimum N47 quality (99.997 \% purity).