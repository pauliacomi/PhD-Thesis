% !TEX root = ../../main.tex

\section{Results and discussion}


\subsection{Routine characterization of a MOF sample}




New materials are often screened for their ability to act as a \ce{CO2} capture material. 
A good predictor of performance 
in this application are the enthalpies of adsorption, which are an indication of
host-guest interactions. Here, we first measure the differential heats of adsorption
directly through the use of adsorption microcalorimetery at \SI{303}{\kelvin}. 
Then, to determine the isosteric heats of adsorption,
two isotherms have been measured at \SI{303}{\kelvin} and \SI{323}{\kelvin} respectively. 
The complete set of isotherms is loaded into \texttt{pyGAPS} and plotted by the 
\lstinline{pygaps.plot_iso()} function as seen in Figure~\ref{fig:uioisostericiso}. 
To calculate the isosteric heat
of adsorption, the two isotherms measured for this purpose are passed through the 
\lstinline{pygaps.isosteric_heat()} function. The results from the calculation are overlaid 
on top of the measured calorimetric data in 
Figure~\ref{fig:uioisostericheat}. The two datasets are
overlap for the most part but diverge at low loadings and near complete coverage.
At low loading the small changes in pressure amount introduce large errors in the 
Clausius-Clapeyron equation. This, together with the breakdown of the 
assumption of equilibrium due to active sites in the MOF lead to the calorimetric
measurement providing more valid results. At higher loadings, where the isotherm reaches
a plateau and the change in adsorbed amount is small from point to point, errors are
introduced in the direct calculation of the heat of adsorption. The two techniques are thus 
complementary.  

\begin{figure}[htb]

    \centering
    \begin{subfigure}[b]{.5\textwidth}
        \centering
        \includegraphics[width=.8\linewidth]{uio-isosteric-iso}
        \caption{}%
        \label{calo:fig:uioisostericiso}
    \end{subfigure}%
    \begin{subfigure}[b]{.5\textwidth}
        \centering
        \includegraphics[width=.8\linewidth]{uio-isosteric-heat}
        \caption{}%
        \label{calo:fig:uioisostericheat}
    \end{subfigure}
    \caption{Calculation of enthalpy of adsorption: 
    (\protect\subref{calo:fig:uioisostericiso}) 
    the dataset of isotherms used and 
    (\protect\subref{calo:fig:uioisostericheat}) the calculated
    isosteric heat (red line) together with the measured 
    differential enthalpy of adsorption (blue triangles)}%
    \label{calo:fig:uioisosteric}

\end{figure}

\subsection{Analysis of a carbon sample for gas separation applications}

A sample of reference carbon Takeda 5A is to be investigated for an in-depth characterisation of
the adsorption behaviour of pure gases, with a focus on describing the pore environment.
Afterwards, the performance of different binary separations is evaluated, 
such as \ce{CO2}/\ce{N2} and propane/propylene.

Pure gas adsorption data has been recorded at \SI{303}{\kelvin} in 
conjunction with 
microcalorimetry on \ce{N2}, \ce{CO}, \ce{CO2}, \ce{CH4}, \ce{C2H6},
\ce{C3H6} and \ce{C3H8}. The complete dataset is plotted with the 
\lstinline{pygaps.plot_iso()} function and can be seen in Figure~\ref{fig:takedadataset}.

Nitrogen and carbon monoxide are similar in their adsorption behaviour,
with a nearly linear isotherm and low capacities.
Hydrocarbons are adsorbed with higher loadings, with both propane and propylene 
reaching a plateau at low pressures. Propylene is seen to have a 
higher capacity than propane, with packing effects as a likely cause.
Carbon dioxide has the highest loading capacity of the entire dataset.

Two parameters can be useful in characterising the local pore environment
before guest-guest interactions come into effect: the Henry constant at 
low loadings as well as the initial enthalpy of adsorption. Both can 
be calculated with \texttt{pyGAPS}, with several options in 
regard to the methodology. Here, Henry's constant is calculated using the 
\lstinline{pygaps.initial_henry_virial()} function, which fits a virial model to 
the isotherm and then takes the limit at loading approaching zero. The initial enthalpy of
adsorption is obtained through the \lstinline{pygaps.initial_enthalpy_comp()} function.
This fits the enthalpy curve to a compound contribution from 
guest-host interaction, defects, guest-guest attraction and repulsion using a minimization 
algorithm. The results of the calculations are plotted versus the polarizability of the 
gas used, which can be obtained from the respective \texttt{Adsorbate} class.
Figure~\ref{fig:takedatrends} shows that both the parameters fall on a linear 
trend, which suggests that the interactions between those guests and the pore walls are 
mostly due to Lennard-Jones interactions. Carbon dioxide has a higher enthalpy 
of adsorption than the baseline due to the contribution from its quadrupole moment. 
There is almost a complete overlap between propane and propylene, which leads to
the conclusion that the unsaturated double bond does not interact in a specific way
with the carbon surface.
The difference between the two isotherms is due exclusively to steric and packing effects.

\begin{figure}[ht]

    \centering
    \begin{subfigure}[b]{.5\textwidth}
        \centering
        \includegraphics[width=.95\linewidth]{takeda-dataset}
        \caption{}%
        \label{fig:takedadataset}
    \end{subfigure}%
    \begin{subfigure}[b]{.5\textwidth}
        \centering
        \includegraphics[width=.8\linewidth]{takeda-enth-henry}
        \caption{}%
        \label{fig:takedatrends}
    \end{subfigure}
    \caption{Takeda 5A dataset processing: (\protect\subref{fig:takedadataset}) the
    experimental dataset all recorded gases and (\protect\subref{fig:takedatrends}) the calculated
    trends of initial heat of adsorption and Henry's constant}%
    \label{fig:takedaanalysis}

\end{figure}
