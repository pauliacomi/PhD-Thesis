% !TEX root = ../../main.tex

\section{Introduction}

The previous chapter focused on a physical description
of adsorption and its use for the characterisation of
metal organic frameworks and other porous materials. It was found that 
a large variation in surface areas exists in scientific literature.
Therefore, a requirement for a more rigorous type of characterisation
is highlighted, a requirement which could be fulfilled through the 
simultaneous measurement of the energetics of adsorption. 
A determination of the energetic aspects of adsorption 
and the interactions in the studied system allows for an understanding 
of the kinds of physical and chemical processes occurring
during adsorption. Through a direct or indirect measurement
of the \textbf{differential} enthalpy of adsorption 
(\gls{dh}), information about surface composition, 
the strength of adsorbate-guest and guest-guest interactions, 
phase change phenomena and, in some cases, transitions in the
adsorbing material itself can be obtained. As such, it is a useful
parameter for fundamental characterisation.

Furthermore, the enthalpy of adsorption is a crucial parameter
in an industrial setting. Owing to the exothermic nature of
adsorption, in combination with the strong influence of temperature
on the performance of adsorbent materials, the total or 
\textbf{integral} enthalpy of adsorption (\gls{dH}) is
often regarded as one of the most important parameters in the
design of beds and columns, alongside working capacity and
adsorption/catalytic selectivity. The integral enthalpy of adsorption
is also a measure of the sum of all guest-host interactions, which
have to be overcome for material regeneration. As such, it
represents an important metric of the energy efficiency
of a cyclic process.

Therefore the insight afforded through measurement of
the energetic components of adsorption can prove invaluable
for investigating the subtle changes in adsorbent materials
which lead to the kind of variability encountered in
\autoref{pyg}. The thoroughness and suitability of the activation
procedure, presence of surface functionalisations, defect
formation such as inclusion of counterions and vacancies
and activation-driven phenomena such as gate opening,
state switching or flexibility can be qualitatively and,
with careful methodology, even quantitatively analysed.

\subsection*{Chapter summary}

First, an overview of the theoretical aspects underpinning
the energetics of adsorption is presented. The methods
section will go into detail in the available methodology
to determine the enthalpy of adsorption on surfaces and in
pores. The final part of the chapter will explore the use
of combined adsorption and calorimetric measurements
as a way of extending characterisation of \glspl{MOF}.

\subsection*{Contributions}

All calorimetric measurements and data processing were
performed by Paul Iacomi in the Madirel Laboratory, Marseille.
The sample of Zr Fumarate \gls{MOF} was synthesised in the group
of Prof.\ Peter Behrens, from the University of Hannover,
Germany. Gravimetric isotherms referenced in this chapter were
recorded by Andrew Wiersum during his thesis.