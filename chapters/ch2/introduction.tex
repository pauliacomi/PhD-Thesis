% !TEX root = ../../main.tex

\section{Introduction}

While the previous chapter focused on a physical description
of adsorption and its use for the characterisation of
porous materials, a rigorous thermodynamic background
was omitted. However, an understanding of the energetic
aspects of adsorption allows for one of the most powerful
applications of adsorption methodology, namely a representation
of the kinds of physical and chemical processes occurring
during adsorption. Through a direct or indirect measurement
of the so-called enthalpy of adsorption (\(h_{ads}\)),
information about surface composition, the strength of
adsorbate-guest and guest-guest interactions, phase change
phenomena and, in some cases, transitions in the
adsorbing material itself can be obtained.

Furthermore, the enthalpy of adsorption is a crucial parameter
in an industrial setting. Owing to the exothermic nature of
adsorption in combination with the strong influence of temperature
on the performance of adsorbent materials, \(h_{ads}\) is
often regarded as the most important parameter in the
design of beds and columns, alongside working capacity and
adsorption/catalytic selectivity. The enthalpy of adsorption
is also a measure of the guest-host interactions, which
have to be overcome for material regeneration. As such, it
represents an important metric of the energy efficiency
of the process.

As such the insight afforded through measurement of
the energetic components of adsorption can prove invaluable
for investigating subtle changes in adsorbent materials
which lead to the kind of variability encountered in
\autoref{pyg}. The thoroughness and suitability of the activation
procedure, presence of surface functionalisations, defect
formation such as inclusion of counterions and vacancies
and activation-driven phenomena such as gate opening,
state switching or flexibility all be qualitatively and,
with careful methodology, even quantitatively analysed.

\subsection*{Chapter summary}

First, an overview of the theoretical aspects underpinning
the energetics of adsorption is presented. The methods
section will go into detail in the available methodology
for studying the enthalpy of adsorption on surfaces and in
pores. The final part of the chapter will explore the use
of combined adsorption and calorimetric measurements
as a way of extending characterisation of porous materials.

\subsection*{Contributions}

The isotherms on the Takeda 5A carbon were recorded by
Andrew Wiersum during his thesis.
The sample of Zr Fumarate MOF was synthesised in the group
of Prof.\ Peter Behrens, from the University of Hannover,
Germany. All measurements on this sample and data processing were
performed by Paul Iacomi in the Madirel Laboratory, Marseille.