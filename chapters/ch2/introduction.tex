% !TEX root = ../../main.tex

\section{Introduction}

While the previous chapter focused on a physical description
of adsorption and its use for the characterisation of 
porous materials, a rigorous thermodynamic background
was omitted. However, an understanding of the energetic
aspects of adsorption allows for one of the most powerful
applications of adsorption methodology, namely a representation
of the kinds of physical and chemical processes occurring 
during adsorption. Through a direct or indirect measurement
of the so-called enthalpy of adsorption (\(\Delta H_{ads}\)),
information about surface composition, the strength of 
adsorbate-guest and guest-guest interactions, phase change
phenomena and, in some cases, transitions in the 
adsorbing material itself can be obtained.

Furthermore, the enthalpy of adsorption is a crucial parameter
in an industrial setting. As adsorption is usually exothermic,
as well as a temperature 

\subsection*{Chapter summary}

First, an overview of the theoretical aspects underpinning 
the energetics of adsorption is presented. The methods 
section will go into detail in the available methodology
for studying the heat of adsorption on surfaces and in 
pores. The final part of the chapter will explore the use
of heat of adsorption measured in conjunction with isotherms
as a useful way of extending characterisation of porous
materials.

\subsection*{Contributions}

The sample of Zr Fumarate MOF was synthesised in the group
of Prof.\ Peter Behrens, from the University of Hannover, 
Germany. All presented measurements were performed 
by Paul Iacomi in the Madirel Laboratory, Marseille.