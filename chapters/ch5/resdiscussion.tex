% !TEX root = ../../main.tex

\section{Results and discussion}

The first part of this section will present the characteristics of the
\gls{NGA} step in DUT-49 as recorded by ambient calorimetry, 
and what a cursory examination of the isotherm and enthalpy
curves at \SI{303}{\kelvin} can reveal about the energetics of the 
system and the stability of the two phases.

The following subsection contains the results of the studies on 
DUT-49 analogues, highlighting the effects that linker elongation
and functionalization have on the adsorption behaviour and \gls{NGA}
extent at ambient temperature.

Finally, an in-depth study of the DUT-49 \gls{NGA} mechanism is performed,
employing a variety of gas probes (\ce{N2, Ar, O2, CO}) at low temperature.
The time-resolved data obtained through continuous adsorbate introduction
sheds light on the influence of the host-guest and guest-guest interactions
on transition mechanics, the energetic barrier of the transition state
and on system kinetics. A comparison with non-flexible analogues allows
an assessment of the role of the pore filling mechanism on the structural
transition to be made.

% !TEX root = ../../main.tex

\subsection{The structural transition leading to NGA in DUT-49}

In the time-resolved pressure and calorimetry signal recorded when
performing butane adsorption at \SI{303}{\kelvin}, the NGA step 
is clearly visible as the pressure in the cell suddenly increases 
after a dosing point. An example can be seen 
in \autoref{dut:fig:dut-49-transient}, next to a regular dosing step.
A short time after the adsorbate is introduced in the measurement
cell, the material contracts. The calorimeter signal is
positive, therefore the overall process is still exothermic --- even
though desorption and structural transition, both endothermic
processes, are taking place.

\begin{figure}[htb]
    \includegraphics[width=\textwidth]{rtc/dut-49-transient}%
    \caption{The pressure and calorimeter signal from butane adsorption
    on DUT-49, highlighting a typical step (left) and the 
    observed NGA step (right)}%
    \label{dut:fig:dut-49-transient}
\end{figure}

The signals are then processed to yield the combined isotherm and
differential heat of adsorption. \autoref{dut:fig:dut-49-butane} shows
two isotherms recorded on different DUT-49 samples. 
Several observations can be made. First, a clear transition takes place
around \(0.15~p/p_0\). This corresponds to the \textit{op}/\textit{cp} 
transition which \textit{decreases} the amount adsorbed per gram 
of material. After NGA, the material is in its \textit{cp} form,
where it remains throughout the remainder of the measurement. 
Unlike in the methane experiments performed at 
\SI{111}{\kelvin}, the structure does not re-open. A secondary 
transition to the \textit{op} form is expected at higher pressures,
close to the saturation pressure of the adsorbate. However, this 
pressure range could not be reached within the experimental conditions.
\todo{Need simulated isotherms from Simon}

\begin{figure}[htb]
    \centering
    \begin{subfigure}{0.33\linewidth}
        \includegraphics[width=\linewidth]{rtc/dut-49-butane-reg}%
        \caption{}\label{dut:fig:dut-49-butane-reg}
    \end{subfigure}%
    \begin{subfigure}{0.33\linewidth}
        \includegraphics[width=\linewidth]{rtc/dut-49-butane-log}%
        \caption{}\label{dut:fig:dut-49-butane-log}
    \end{subfigure}%
    \begin{subfigure}{0.33\linewidth}
        \includegraphics[width=\linewidth]{rtc/dut-49-butane-enth}%
        \caption{}\label{dut:fig:dut-49-butane-enth}
    \end{subfigure}%
    \caption{Butane adsorption experiments on two samples of 
    DUT-49, DUT-49(o) and DUT-49(m) shown as (a) regular 
    isotherms (b) logarithmic isotherms 
    and (c) enthalpy as a function of pressure.}%
    \label{dut:fig:dut-49-butane}
\end{figure}

Due to the steep knee in the adsorption/desorption branch of the 
\textit{cp} form, a complete desorption branch cannot be obtained,
with the minimum attained loading of \SI{4}{\milli\mol}. As such,
framework collapse is not seen in the experiments. Another 
shortcoming is that the isotherm
and energy landscape of the \textit{op} form in its metastable 
region is inaccessible, since by definition the material will 
undergo a transition in this pressure range.

A key feature visible in the two recorded isotherms is that,
while adsorption on the open pore form fully overlaps between the 
two experiments, both the location and extent of NGA differ 
slightly. In the DUT-49(o) isotherm, a smaller NGA step occurs,
with the resulting structural contraction apparently unable to 
achieve a complete closing of the pores. Upon desorption, a 
secondary step occurs in the same \(0.15-0.2~p/p_0\) pressure range,
after which the material is fully in its \textit{cp} state,
as evidenced by the overlap with the DUT-49(m) isotherm. The 
reason behind the dissimilar behaviour is the contribution
of crystal size to the energy barrier of transition. A thorough
analysis can be found in the paper published 
by \citet{krauseEffectCrystalliteSize2018} where it shown that 
a high surface to volume ratio has a negative impact on NGA, with 
smaller crystallites unable to achieve a complete contraction,
instead accessing an intermediated or \textit{ip}, with the \textit{``a''}
lattice parameter 5--7\% smaller than the \textit{op} phase
rather than 24\% in the \textit{cp} phase.

The enthalpy curves in \autoref{dut:fig:dut-49-butane-enth},
paint a picture of the energetic landscape of both phases.
First it should be noted that the differential enthalpy of
the NGA step itself does not appear on the graph as it is a
negative value. This is not because the transition step is 
endothermic, as proven in \autoref{dut:fig:dut-49-transient}. 
However, as the net change in adsorbed amount between the two
points where the transition occurs is negative, the calculated
enthalpy \textit{per mole of gas adsorbed} takes the same sign.

In the adsorption branch, the initial enthalpy of adsorption 
can be observed to be around \SIrange{38}{40}{\kilo\joule\per\mol}.
This value is higher than observed for butane in HKUST-1 in the 
order of \SI{35.6}{\kilo\joule\per\mol}~\cite{farrussengHeatsAdsorptionSeven2009} 
though smaller than in another similar copper paddlewheel flexible MOF 
(\SI{50}{\kilo\joule\per\mol})~\cite{langeStructuralFlexibilityCopperbased2014},
and suggest that the interaction of the pore wall with the adsorbate is 
relatively low, probably due to the large pore size. 
After \SI{2}{\milli\mol\per\gram}, the enthalpy
curve slopes downwards until a local minima around 
\SI{30}{\kilo\joule\per\mol} before NGA. 
This can be assumed to be the multilayer adsorption region,
where the field gradient of the pore wall decreases and guest-guest 
interactions dominate.
The enthalpy calculated for the desorption curve is essentially the 
differential enthalpy of adsorption of butane on the \textit{cp} form.
A large difference, in the range of 
\SIrange{10}{20}{\kilo\joule\per\mol}, exists
between the enthalpy of adsorption in the two states. This can be 
attributed to the more confined nature of the \textit{cp} form, with the
smaller pore walls increasing the interaction of the framework backbone
with an adsorbate molecule. The energy required to drive the 
transition and generate the observed thermal effect can be accounted 
for by the increased total interactions of remaining adsorbed 
molecules with the \textit{cp} state of the framework.
A more in-depth analysis of the NGA energetics will be presented 
in \autoref{dut:indepth}.

\subsubsection{A brief mention of measurement uncertainty}

In order to assess whether results obtained through calorimetry
are within acceptable accuracy, uncertainty calculations have been
carried out. The method used here is laid out by the International 
Organisation for Standardisation (ISO) in the Guide to the expression
of Uncertainty in Measurements (GUM). The method is fully detailed 
in \autoref{appx:errors}, and consists of an identification of 
all variables used in the calculation of the final result, and 
estimation of the uncertainty in the final value as a function of
the uncertainty in each such variable. The result is multiplied
by a factor of confidence, which has been chosen as 95\% in the 
figures presented in \autoref{dut:fig:dut-49-butane-err}.


\begin{figure}[htb]
    \centering
    \begin{subfigure}{0.33\linewidth}
        \includegraphics[width=\linewidth]{rtc/dut-49-butane-err-iso}%
        \caption{}\label{dut:fig:dut-49-butane-err-iso}
    \end{subfigure}%
    \begin{subfigure}{0.33\linewidth}
        \includegraphics[width=\linewidth]{rtc/dut-49-butane-err-enth}%
        \caption{}\label{dut:fig:dut-49-butane-err-enth}
    \end{subfigure}%
    \begin{subfigure}{0.33\linewidth}
        \includegraphics[width=\linewidth]{rtc/dut-49-butane-err-enthp}%
        \caption{}\label{dut:fig:dut-49-butane-err-enthp}
    \end{subfigure}%
    \caption{Estimated errors at a 95\% confidence range for 
    (a) loading as a function of pressure, 
    (b) differential enthalpy as a function of pressure 
    (c) differential enthalpy as a function of loading.}%
    \label{dut:fig:dut-49-butane-err}
\end{figure}

The use of the manometric method for isotherm measurement entails a 
cumulative error in the pressure measurement that leads to an increase
in uncertainty with each measured point. It can be seen in 
\autoref{dut:fig:dut-49-butane-err-iso} that in spite of the margin of 
error, there is perfect overlap between the two measured isotherms
in the adsorption branch and in the desorption branch. It is reasonable
to assume that the error is therefore much smaller than the calculation 
would suggest.

The uncertainty in the differential enthalpy of adsorption 
(\autoref{dut:fig:dut-49-butane-err-enth}) is a 
function of \(\Delta n\), the amount adsorbed in each step, as well
as pressure, with the same cumulative error applicable. The former 
variable accounts for the large uncertainty in flat sections of 
the isotherm where almost no adsorption takes place, while the latter
results in the spread seen at the end of the desorption curves. 
However, when observing \(\Delta_{ads} \dot{h}\) as a function of 
loading in \autoref{dut:fig:dut-49-butane-err-enthp}, the uncertainty 
is confined to high values on the x axis, with a clear separation
of isotherm branches.

As the display of uncertainty ranges clutter the isotherm graphs and 
makes it hard to distinguish features, the remainder of this chapter 
will only display them if the error range is significant.
% !TEX root = ../../main.tex

\subsection{Impact of framework structure on transition mechanics}%
\label{dut:comparison}

While the discovery of the NGA transition in DUT-49 was a result 
of serendipity, it opens the door to a rational design approach 
to modify the extent and the location of the phenomenon. In principle,
there are several avenues that could be taken in order to tune the 
contraction mechanics.
\begin{itemize}
    \item Changing the range of metastability of the open pore state.
    \item Decreasing the porosity of the closed pore state.
    \item Increasing the capacity of the open pore form.
\end{itemize}
These factors are bound to be tightly interlinked, with a slight 
alteration in one possibly leading a shift in all. For example, the 
addition of a stabilizing group which would increase the tensile 
strength of the linker is also likely to decrease the porosity 
of the entire system.

There are a range of physicochemical modifications available to 
tune the properties of the framework, many already employed in the
MIL-53 and MIL-47 family of flexible MOFs.
Through functionalisation or modification of the linker, the strength of the guest-guest
and guest-host interactions is affected, as evidenced by the different
gate opening behaviour with nitrogen and water on several functionalised
versions on MIL-53~\cite{biswasNewFunctionalizedFlexible2011}.
In the case of DUT-49, moieties grafted to the central strut or changes
in the linker backbone are also likely to affect its buckling behaviour.
The use of a different metal as the node, has succeeded in changing the 
mechanical response of MIL-53~\cite{yotImpactMetalCentre2016}. This 
approach is likely have less impact on DUT-49, as the mechanism of 
contraction is due to linker flexibility.
A common rational design methodology is the so-called isoreticular design,
where topologically isomorphic MOFs are synthesised through progressive elongation
of the linker. Another path to controlling flexibility is manipulation of crystal 
size, as it has already been shown by \citet{krauseEffectCrystalliteSize2018}.
Finally, structural defects, of which a description was given 
in \autoref{def}, are another degree of freedom to consider for NGA 
tunability. Approaches such as mixed linkers synthesis and vacancy defects
may allow for fine-grain influence of framework stiffness.

\subsubsection{Behaviour of isoreticular materials}

Adsorption 

\autoref{dut:fgr:dut-reticular} shows the dataset recorded on the 
non-interpenetrated versions of the material.

\begin{figure}[htb]
    \centering
    \begin{subfigure}{0.33\linewidth}
        \includegraphics[width=\linewidth]{butane/dut-reticular-reg}%
        \caption{}\label{dut:fgr:dut-reticular-reg}
    \end{subfigure}%
    \begin{subfigure}{0.33\linewidth}
        \includegraphics[width=\linewidth]{butane/dut-reticular-log}%
        \caption{}\label{dut:fgr:dut-reticular-log}
    \end{subfigure}%
    \begin{subfigure}{0.33\linewidth}
        \includegraphics[width=\linewidth]{butane/dut-reticular-enth}%
        \caption{}\label{dut:fgr:dut-reticular-enth}
    \end{subfigure}%
    \caption{(a) Experimental adsorption isotherms for DUT-48, DUT-46, DUT-49 and 
    DUT-50. Enthalpy points are omitted for clarity. (b) A logarithmic plot of 
    isotherms and enthalpy curves, to highlight the low pressure region. 
    (c) Enthalpy as a function of loading for these materials.}%
    \label{dut:fgr:dut-reticular}
\end{figure}


The isotherms in \autoref{dut:fgr:dut-reticular-reg} show that only DUT-49 
and DUT-50 undergo NGA, one around \(0.15~p/p_0\) and the other at a higher 
relative pressure around \(0.25~p/p_0\).

Indeed, the results follow a predictable trend. With an increase of linker 
size, a higher overall capacity is available from the 

In DUT-50, the structure is seen to re-open around \(0.5~p/p_0\), although 
the pressure is not high enough to obtain a fully \textbf{op} form.
The transition to the \textbf{cp} phase is 

\autoref{dut:fgr:dut-reticular-interp} presents isotherms and enthalpy on 
DUT-151 and DUT-152, both interpenetrated nets.

\begin{figure}[htb]
    \centering
    \begin{subfigure}{0.5\linewidth}
        \includegraphics[width=\linewidth]{butane/dut-reticular-interp-reg}%
        \caption{}\label{dut:fgr:dut-reticular-interp-reg}
    \end{subfigure}%
    \begin{subfigure}{0.5\linewidth}
        \includegraphics[width=\linewidth]{butane/dut-reticular-interp-enth}%
        \caption{}\label{dut:fgr:dut-reticular-interp-log}
    \end{subfigure}%
    \caption{The (a) isotherms and (b) enthalpy curves of the
    interpenetrated materials DUT-151 and DUT-152. Shaded regions
    are guides for the eye.}%
    \label{dut:fgr:dut-reticular-interp}
\end{figure}


\subsubsection{Behaviour of ``reinforced'' linker analogues}

% !TEX root = ../../main.tex

\subsection{An in-depth look at the NGA mechanism}%
\label{dut:indepth}

As it has been shown in the previous section, there are 
several avenues NGA tuning in the DUT-49 framework.
However, a fundamental understanding of the factors and 
energetics of the process itself may
lend itself to prediction of when the phenomenon occurs without 
experimental input, and even lead to the rational design of 
materials.

To this end, the adsorption of multiple probes was investigated 
at \SI{77}{\kelvin} with \textit{in situ} continuous microcalorimetry,
in order to observe the influence of the guest on the mechanism of
adsorption and NGA. In order to obtain a baseline of adsorption in 
the \textbf{op} phase, a non-flexible alternative is used. 
DUT-149 is the most similar material out of all previously studied
analogues, as it has the same linker length and nearly identical pore
size and surface area.





A common feature to almost all measured isotherms is the identical
adsorption mechanism before pore size effects come into play.
Indeed if enthalpy curves recorded with different adsorbates, at different
temperatures and even on different DUT analogues are compared at low 
loading, they may be mistaken for the same data.

This typical enthalpy curve starts at relatively low values compared 
to adsorption in the same conditions on a similar paddlewheel based 
MOF such as HKUST-1, then increases to a local maxima followed
by a gentle downward slope.
This feature is typical of cooperative adsorption, and suggests that
the interactions with the framework are generally low. The only 
exception to this case are adsorbates which can act with the 
Cu paddlewheel in some way, either through adduct formation, 
electron donation or \(\pi\) backbonding interactions. 


\begin{figure}[htb]
    \centering
    \begin{subfigure}{0.5\linewidth}
        \includegraphics[width=\linewidth]{butane/dut-48-prop-log}%
        \caption{}\label{dut:fgr:dut-48-prop-log}
    \end{subfigure}%
    \begin{subfigure}{0.5\linewidth}
        \includegraphics[width=\linewidth]{butane/dut-48-prop-enth}%
        \caption{}\label{dut:fgr:dut-48-prop-enth}
    \end{subfigure}%
    \caption{The (a) isotherms and (b) enthalpy curves of propane 
    and propylene adsorption on DUT-48, highlighting the high energy
    of \ce{C3H6} on specific sites at low pressure, likely to be 
    \(\pi\)-Cu interactions.}%
    \label{dut:fgr:dut-48-prop}
\end{figure}