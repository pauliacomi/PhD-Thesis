% !TEX root = ../../main.tex

\section{Introduction}

Until this chapter, it has been assumed that the porous materials are 
static when adsorbing a gas. Differences in pore size, crystalinity 
or structure may exist, but these properties do not change
as the host fluid enters the pores.
In most cases this is a reasonable assumption. However, it is not 
universally applicable, as as the forces and interactions exerted 
during adsorption may induce changes in solid itself.

Such effects in porous inorganic materials like zeolites take the form
of pore window vibrations or counterion 
displacement~\cite{coudertMolecularInsightCO2017}.
It is only recently that flexibility was discovered in coordination
polymers, such as MOFs. A feature which arises from 
their comparatively weak coordination bonds or pliant organic components,
it allows for a systematic deflection of bonds throughout the 
entire crystal lattice.
As such, the term ``soft porous crystals'' defines porous solids that are 
both highly ordered and possess the ability to reversibly transform
their structure upon external stimuli. Part of the so-called
third generation of crystalline porous compounds, they represent 
some of the latest developments in the field of MOFs.

The unique properties of flexible materials can result in
their application in fields such as sensing, micromechanical
devices and highly efficient gas storage. It is these perspectives
that make their synthesis and design a key research interest.
However, their flexible nature introduces new challenges in
their characterisation, as factors such as temperature and
thermal history~\cite{liuReversibleStructuralTransition2008},
crystal size~\cite{zhangCrystalSizeDependentStructuralTransitions2014, %
krauseEffectCrystalliteSize2018}, external 
pressure~\cite{itoReversiblePoreSize2013, %
chanutUsingExternalFactors2016}, structural defects~\cite{bennettInterplayDefectsDisorder2016} 
and even adsorption kinetics play a role in their compliance.
This type of variability goes beyond what has been insofar discussed
in this thesis and it is here where a combined characterisation approach
becomes essential in understanding the fundamental physics
governing flexibility and potential prediction of adsorption behaviour. 

\subsection*{Summary}

After a brief introduction of the background of soft porous 
materials, this chapter will present the characterization of a novel
flexible MOF (DUT-49) and its analogues. This material undergoes
a sudden collapse of its pore network into a closed form state
upon adsorption, resulting in the expulsion of gas from its pores.
This phenomena was coined ``negative gas adsorption'' (NGA).
The text will focus on characterisation through calorimetric methods
performed by Paul Iacomi, together with references
of results obtained by collaborating groups included in order
obtain a complete story of the underlying mechanism behind NGA.

\subsection*{Contribution}

The synthesis of all MOFs was performed by Simon Krause
(TU Dresden), together with their initial characterization through 
nitrogen adsorption at \SI{77}{\kelvin}.
Ambient and low temperature calorimetry was carried out by 
Paul Iacomi. Computer simulations of adsorption isotherms
are the result of work from Jack Evans and Prof.\ F.X Coudert.
Mechanical compression experiments were performed in 
the group of Prof.\ Guillaume Maurin in Montpellier.
Prof.\ Philip Llewellyn and Prof.\ Stefan Kaskel were 
instrumental in the analysis of the results obtained.
