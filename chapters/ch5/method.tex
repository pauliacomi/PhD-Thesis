% !TEX root = ../../main.tex

\section{Materials and characterisation methods}

\subsection{Materials}

Several DUT materials have been synthesised in order to 
study the effect of different parameters on the switching 
behaviour. From the point of view of the criterion of interest,
the materials can be divided into the several categories.
The material name, together with the central part of the linker, 
which was modified to change the flexible behaviour of the framework,
is presented in \autoref{dut:tbl:materials}.

\begin{itemize}
    \item Series dedicated to studying the influence of isoreticular
    design through variation of linker length in the order of 
    theoretical increasing porosity: DUT-48, DUT-46, DUT-49, 
    DUT-50, DUT-151/DUT-152. These materials are designed 
    with a linker of increasing size by using differently 
    structured phenyl rings. A corresponding increase in 
    porosity is expected, however, starting from a 4-linear
    phenyl chain (DUT-151), the internal voids are large enough to 
    allow for a secondary interpenetrated network to develop.
    An attempt to prevent this by grafting bulky naphthalene 
    rings was made in the synthesis of DUT-152, but the resulting 
    structure was still found to be interpenetrated.
    
    \item Series assessing the impact of steric hindrance of the 
    central linker bond on NGA, in the order of connectivity:
    DUT-49, DUT-149, DUT-148, DUT-147. The rationale behind this
    approach is to improve the tensile strength of the strut by 
    the addition of sterically hindering side connections.

    \item Series investigating the effect of heterocycles on compliant
    behaviour, using thiophene as replacement for the benzene rings, 
    in the order of increasing linker size: DUT-170, DUT-171, DUT-172,
    DUT-173. If interactions with the framework plays a role in NGA,
    the addition of potentially stronger host-guest sites.

    \item Series aiming to possess a progressively more labile central 
    strut through the use of different degrees of saturation,
    in order of central bond hybridization: DUT-160, DUT-161, DUT-163.
    It was found that the removal of solvent from DUT-162 could not be 
    performed without structure collapse. The softness of the saturated
    backbone lends itself to an unstable \textit{op} state.

    \item Series of increasing crystallite size to study the effect 
    of the crystal surface to volume ratio on NGA. Different sizes
    of DUT-49 were synthesised either through the addition of a 
    acid modulator for obtaining large crystals or through the addition
    of a base to inhibit crystal growth. A series of 4 DUT-49 materials
    was received, of \SI{800}{\nano\metre}, \SI{1}{\micro\metre},
    \SI{4}{\micro\metre} and \SI{10}{\micro\metre} average size 
    respectively.

\end{itemize}

\begin{table}[p]
    \newcolumntype{Y}{>{\centering\arraybackslash}X}
    \newcolumntype{M}[1]{>{\centering\arraybackslash}m{#1}}
    \centering %
	\caption{Flexible materials analogous to DUT-49}
    \footnotesize
    \begin{tabularx}{\linewidth}{M{1cm} | Y >{\centering}m{5cm} Y p{2.5cm}}
		\toprule
        \textbf{Study}
	    & \textbf{Name}
        & \textbf{Linker center}
        & \textbf{Pore volume (geometric)}
        & \textbf{Observations} \\
        & & & \tiny{\si{\centi\metre^3\per\gram}} & \\
		\midrule
        \multirow{1}{*}{\rotatebox[origin=c]{90}{---}} &
        DUT-49  & 
            \includegraphics[width=4cm]{structures/DUT-49}
            & 2.24 &
            \makecell[l]{
            Multiple crystal sizes \\ %
            \tiny\llap{\textbullet}~~DUT-49(o)-\SI{4.4}{\micro\metre}\\%
            \tiny\llap{\textbullet}~~DUT-49(s)-\SI{1.0}{\micro\metre}\\%
            \tiny\llap{\textbullet}~~DUT-49(m)-\SI{4.0}{\micro\metre}\\%
            \tiny\llap{\textbullet}~~DUT-49(l)-\SI{10.0}{\micro\metre}} \\
        \midrule
        \multirow{5}{*}{\rotatebox[origin=c]{90}{Linker size}} &
        DUT-48  & 
            \includegraphics[width=2.5cm]{structures/DUT-48}
            & 1.41 & --- \\
        & DUT-46  & 
            \includegraphics[width=3cm]{structures/DUT-46}
            & 1.75 & --- \\
        & DUT-50  & 
            \includegraphics[width=5.3cm]{structures/DUT-50}
            & 3.23 & --- \\
        & DUT-151  & 
            \includegraphics[width=5cm]{structures/DUT-151} 
            & 4.36 & Interpenetrated \\
        & DUT-152  & 
            \includegraphics[width=5cm]{structures/DUT-152}
            & 4.13 & Interpenetrated \\
        \midrule
        \multirow{3}{*}{\rotatebox[origin=c]{90}{Linker stiffening}} &
        DUT-149  & 
            \includegraphics[width=3.8cm]{structures/DUT-149}
            & 2.15 & --- \\
        & DUT-148  & 
            \includegraphics[width=3.8cm]{structures/DUT-148}
            & 2.12 & --- \\
        & DUT-147  & 
            \includegraphics[width=3.8cm]{structures/DUT-147}
            & 2.11 & --- \\
        \midrule
        \multirow{4}{*}{\rotatebox[origin=c]{90}{Heterocycle}} &
        DUT-170  & 
            \includegraphics[width=3.2cm]{structures/DUT-170}
            & 1.59 & --- \\
        & DUT-171  & 
            \includegraphics[width=4cm]{structures/DUT-171}
            & 1.67 & --- \\
        & DUT-172  & 
            \includegraphics[width=4cm]{structures/DUT-172}
            & 2.06 & Not measured \\
        & DUT-173  & 
            \includegraphics[width=5.5cm]{structures/DUT-173}
            & 2.70 & Not measured \\
        \midrule
        \multirow{3}{*}{\rotatebox[origin=c]{90}{Strut saturation}} &
        DUT-160  & 
            \includegraphics[width=5cm]{structures/DUT-160}
            & 2.77 & --- \\
        & DUT-161  & 
            \includegraphics[width=5cm]{structures/DUT-161}
            & n.\ a. & Not measured \\
        & DUT-162  & 
            \includegraphics[width=5cm]{structures/DUT-162}
            & 2.72 & No stable \textit{op} state \\
        \bottomrule
	\end{tabularx}%
	\label{dut:tbl:materials}
\end{table}%

\subsection{Characterisation methods}

In order to examine the energetic components of both adsorption and 
NGA, the combined manometry and calorimetry setup first 
presented in \autoref{calo} was used. Ambient temperature calorimetry
was conducted at \SI{303}{\kelvin} with probes such as butane, propane
and propylene using the step-by-step gas introduction method. 
The apparatus description and exact procedure for such an experiment 
can be found in \autoref{calo:rtc} of \autoref{calo}.
For the experiments at \SI{77}{\kelvin}, a high resolution continuous
introduction method was employed using the low temperature calorimeter
described in \autoref{calo:ltc} of \autoref{calo}.

Extreme care has to be taken during sample preparation, as all materials 
studied are sensitive to heat and water vapour, which attack
the Cu paddlewheel and lead to material degradation. To prevent any 
decomposition, the samples were stored under an inert argon 
atmosphere in a glovebox. The loading of ambient temperature 
cells was performed inside the glovebox while filling and sealing 
of low temperature glass cells was done in an argon flow. After 
sample cell preparation, the materials are activated under 
dynamic vacuum at \SI{120}{\degreeCelsius}. 

It is worth noting that if the sample undergoes an \textit{op}/\textit{cp}
phase transition during the experiment and cannot be reopened with
the application of high pressure, as is the case for butane at 
\SI{303}{\kelvin}, the material cannot be reused. Any attempt to
activate it under vacuum leads to structural breakdown of the 
unstable \textit{cp} phase. 
Since a limited amount of sample is available, this played a large
role in data acquisition, with several isotherms only recorded once.
