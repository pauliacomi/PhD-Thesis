% !TEX root = ../../main.tex

\section{Conclusion}

In this chapter, an extensive study of the unique phase transitions
in the flexible framework of DUT-49 and its analogues was performed
through combined adsorption and microcalorimetry.
The results offer insight into the mechanism of phase transition
and the influence of structural modifications on the contraction step,
as well as corroborate the ongoing multi-pronged approach to understand
this type of transition.

An exploration of several approaches for tuning the NGA step through 
synthesis is explored using adsorption microcalorimetry has revealed
several interesting findings regarding what linker properties can
influence the barrier to structural transition and stability of the
two phases. These can be generally summarised as follows.

\begin{itemize}
    \item Stiffening of the central linker can be achieved through 
    linker shortening or side functionalisation, usually accompanied 
    by a reduction in total pore volume. Structural contraction may 
    still take place through adsorption in conditions of strong 
    guest-guest interactions or through mechanical pressure.
    \item Lengthening of the linker can increase porosity,
    but is only effective before framework interpenetration 
    occurs, after which complete structural contraction is 
    prevented.
    \item Different linker hinging systems may change 
    the mechanism of contraction, but a ligand backbone with 
    deformable bonds is essential for phase transition.
\end{itemize}

Extending the characterisation through microcalorimetry at \SI{77}{\kelvin}
with other probes allowed an in-depth exploration of the guest-host and 
guest-guest interactions in the DUT-49 system. First, it is confirmed 
that at the experimental conditions, the adsorbates can be assumed to 
be in a liquid state in the pores. DUT-149 is selected as a non-flexible 
analogue, which allows for complete adsorption curves on the \textit{op}
state of the framework to be recorded, except during oxygen adsorption,
when the increased energy barrier induced through linker functionalisation
is also overcome. The enthalpy curves of all the probes used are seen to 
be similar in the low pressure region. Even in adsorbates which interact 
strongly with the copper atoms in the MOP, such as carbon monoxide and 
\(\pi\)-bond containing hydrocarbons, the increased interactions with the 
framework affect only the initial 10\% of all adsorbed guests. Together
with the location of NGA in the secondary pore filling step,
guest-guest interactions are determined to be the dominant contribution 
to the transition mechanism. The pore filing process itself is likened
to a condensation step, which is confirmed through \textit{in situ}
powder neutron diffraction results.
The probes used reveal the lower limits of the DUT-49
and DUT-149 phase transitions at \SI{77}{\kelvin} with nitrogen and 
oxygen, respectively. The calorimetric signal during the NGA step
is seen to be an indication of the energy required or generated 
during contraction. The emerging trend of \ce{N2}<\ce{Ar}<\ce{O2}
can then be related to the strength of temperature and probe dependence
of NGA through the enthalpy of vaporisation of the fluid.

However, many questions and research opportunities still remain about the 
system as a whole. From an adsorption standpoint, a rigorous theoretical 
model of the influence of the adsorbed phase on the forces acting on the
framework is still lacking at the current time.
From a material point of view, the particular molecular properties that 
can affect the activation barrier between the two phases without 
decreasing the stability of the closed pore from have only been 
subject of a cursory examination.
Furthermore, the influence of crystal size and surface on the
barrier of transition is yet to be quantified. The use of a copper 
paddlewheel leads to instability to humidity and it is currently
not known if the metal has an influence on transition mechanics.
Finally, tantalising applications, unique to such systems can be 
envisaged, like micromechanical switches or pressure amplifiers.

Beyond the properties of this particular framework, it is shown that 
the assumption of a static porous solid during adsorption can often
diverge significantly from the behaviour of a real system. In particular
with soft materials such as MOFs, flexibility may introduce 
unexpected and counterintuitive phenomena which can only be elucidated 
through the use of the use of complementary techniques such as 
\textit{in situ} microcalorimetry, NMR, electron and neutron diffraction 
and advanced \textit{in silico} simulations.

\FloatBarrier%
\pagebreak