% !TEX root = ../../main.tex

\subsection{Adsorption isotherms at 77K and room temperature}

Nitrogen sorption isotherms measured at \SI{77}{\kelvin} have been
measured on both powder and \(\rho\)-alumina pellets, with the dataset
presented in Figure~\ref{fgr:shaping:n2adsorption}.

\begin{figure}[p!]
    \centering
    
    \begin{subfigure}{\linewidth}
        \centering
        \parbox{0.1\linewidth}{\caption{}\label{fgr:shaping:n277kuio66}}%
        \includegraphics[width=0.3\textwidth]{n277k/UiO-66(Zr)-physisorption}%
        \includegraphics[width=0.3\textwidth]{n277k/UiO-66(Zr)-physisorption-log}%
    \end{subfigure}

    \begin{subfigure}{\linewidth}
        \centering
        \parbox[c]{0.1\linewidth}{\caption{}\label{fgr:shaping:n277kuio66}}%
        \includegraphics[width=0.3\textwidth]{n277k/MIL-100(Fe)-physisorption}%
        \includegraphics[width=0.3\textwidth]{n277k/MIL-100(Fe)-physisorption-log}%
    \end{subfigure}

    \begin{subfigure}{\linewidth}
        \centering
        \parbox[c]{0.1\linewidth}{\caption{}\label{fgr:shaping:n277kmil127}}%
        \includegraphics[width=0.3\textwidth]{n277k/MIL-127(Fe)-physisorption}%
        \includegraphics[width=0.3\textwidth]{n277k/MIL-127(Fe)-physisorption-log}%
        \label{fgr:shaping:n277kmil127}
    \end{subfigure}
    
    \caption{Nitrogen isotherms at 77K for (a) UiO-66(Zr), 
    (b) MIL-100(Fe) and (c) MIL-127(Fe). The powder sample is in light
    blue while the \(\rho\)-alumina sample in dark blue. Logarithmic
    graphs of the isotherms are on the right for clarity of the low
    pressure region.}%
    \label{fgr:shaping:n2adsorption}
\end{figure}

The isotherms can be processed to yield properties such as specific 
surface area and pore volume for the powders and 
the pellets.

\begin{table}[htbp]
    \centering
    \caption{Properties of the studied powders and pellets}
    \begin{tabular}{lcccc}
        \toprule
        \thead{\textbf{MOF}}
        & \thead{\textbf{form}}
            & \thead{\textbf{BET surface area}}
                & \thead{\textbf{Pore volume}}
                    & \thead{\textbf{Bulk density}} \\
        \midrule
        \multirow{2}{*}{UiO-66(Zr)} & powder & 903 & 0.38 & 0.3192 \\
            & \(\rho\)-alumina & 619 & 0.24 & 0.4724 \\
        \multirow{2}{*}{MIL-100(Fe)} & powder & 1928 & 0.78 & 0.2165 \\
            & \(\rho\)-alumina & 1451 & 0.6 & 0.3512 \\
        \multirow{2}{*}{MIL-127(Fe)} & powder & 1400 & x & 0.412 \\
            & \(\rho\)-alumina & 1266 & 0.49 & 0.526 \\
        \bottomrule
    \end{tabular}%
    \label{tab:shaping:propertiestable}
\end{table}%
  
As expected, the specific surface area of the shaped samples is decreased compared to the
corresponding powder. While in the case of MIL-127(Fe) the BET area is only 10\% lower, 
for the MIL-100(Fe) and UiO-66(Zr) materials a larger drop is seen, of 25\% and 31\%,
respectively. 
Another property which can be used to corroborate the effects of shaping is micropore volume,
calculated as the volume of nitrogen adsorbed at a \(p/p^0\) of 0.2. A similar capacity
decrease can be seen here with a 36\%, 23\% and x\% seen in UiO-66(Zr), MIL-100(Fe) and 
MIL-127(Fe) respectively.

\todo{some micropore volumes missing + density}

The decrease in both surface area and micropore volume is too large for it to be a
consequence of the presence of non-porous binder. It is therefore theorised that some 
structure degradation must have occurred in the pelletisation process.

Observation of the physisorption curves sheds light on the further impact of the 
alumina binder on the materials chosen.
The shapes of all isotherms are visually similar, with the pellet ones shifted
downwards due to the aforementioned structure degradation.
In both powders and pellets, the increased uptake after 0.9 \(p/p^0\) is a sign
of condensation in very large pores or voids, which can be attributed to intra-pellet
spaces and crystal agglomeration.
In the MIL-127(Fe) pellets, a narrow hysteresis curve is seen, which closes at a 
\(p/p^0\) of 0.5. This curve corresponds to capillary condensation in a pore size
of around \SI{4}{\nano\metre}. This pore width is too small to be a sign of 
inter-pellet voids and therefore must be a consequence of the shaping process.