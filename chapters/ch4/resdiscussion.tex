% !TEX root = ../../main.tex

\section{Results and discussion}

% !TEX root = ../../main.tex

\subsection{Thermal stability}

In order to check if the shaped samples have not undergone bulk
structural changes, as well as find a suitable activation temperature,
the powder and shaped samples underwent thermogravimetric analysis
under an argon atmosphere.

\begin{figure}
	\centering
	\begin{subfigure}{0.85\textwidth}
		\parbox[c]{0.1\linewidth}{\caption{}\label{shaping:fgr:tgauio66}}%
		\parbox[b]{0.7\linewidth}{%
			\includegraphics[width=\linewidth]{tga/uio66}%
		}%
	\end{subfigure}%

	\begin{subfigure}{0.85\textwidth}
		\parbox[c]{0.1\linewidth}{\caption{}\label{shaping:fgr:tgamil100}}%
		\parbox[b]{0.7\linewidth}{%
			\includegraphics[width=\linewidth]{tga/mil100}%
		}%
	\end{subfigure}%
	
	\begin{subfigure}{0.85\textwidth}
		\parbox[c]{0.1\linewidth}{\caption{}\label{shaping:fgr:tgamil127}}%
		\parbox[b]{0.7\linewidth}{%
			\includegraphics[width=\linewidth]{tga/mil127}%
		}%
	\end{subfigure}%

	\caption{High resolution TGA curves recoded under argon
		on (a) UiO-66(Zr), (b) MIL-100(Fe) and (c) MIL-127(Fe). The
		original powders are depicted in red and the shaped material
		in blue.}%
	\label{shaping:fgr:tgacurves}

\end{figure}

The process of shaping did not have any impact on the thermal stability of
the investigated MOFs, as evidenced by the TGA curves in
\autoref{shaping:fgr:tgacurves}. The primary mass loss occurs
in a \SI{10}{\kelvin} range for all powder-pellet pairs.
Shaped samples are also seen to have a smaller mass loss
at high temperatures. This is expected, as after the addition of
temperature inert alumina, the MOF makes up a lower percentage of
the material.
% !TEX root = ../../main.tex

\subsection{Nitrogen sorption at 77K}

Isotherms have been recorded on samples activated at both
\SI{200}{\degreeCelsius} and \SI{320}{\degreeCelsius}.
It is important to note that, as seen from the TGA curves,
not all capping agents leave the structure when thermal treating
at a lower temperature. The modulator which is still present
within the pore surface will likely influence the adsorption
behaviour. Several example isotherms can be found in
\autoref{def:fgr:n2phys-dataset}, with the complete dataset
present in \autoref{appx:def}, \autoref{appx:def:n2phys}.

There are a few isotherm features which can be analysed to
assess the type of modifications introduced in the structure
and their preponderence.

\begin{itemize}
	\item The slope of the isotherm at low \(p/p_0\) is representative
	      of the first interactions with the pore surface, which can be
	      quantified using the initial Henry constant. It will be influenced
	      by any changes in pore environment such as functionalised defect
	      sites.
	\item Missing liker defects will lead to an increase of the
	      apparent surface area and perhaps lead to a more extensive
	      pore network.
	\item The pore size distribution and total pore volume give
	      indications on the presence of missing cluster defects and/or
	      of the formation of mesoporous voids within the structure.
	\item A steep step at high \(p/p_0\) is indicative of intercrystal
	      condensation and suggests particle aggregation due to lower average
	      crystal size.
\end{itemize}

\begin{figure}[htbp]
	\centering

	\begin{subfigure}{0.45\linewidth}
		\includegraphics[width=\textwidth]{n2phys/dmf-fa}%
		\caption{}%
		\label{def:fgr:n2phys-dmf-fa}
	\end{subfigure}
	\begin{subfigure}{0.45\linewidth}
		\includegraphics[width=\textwidth]{n2phys/h2o-tfa}%
		\caption{}%
		\label{def:fgr:n2phys-h2o-tfa}
	\end{subfigure}

	\begin{subfigure}{0.45\linewidth}
		\includegraphics[width=\textwidth]{n2phys/meoh-ba}%
		\caption{}%
		\label{def:fgr:n2phys-meoh-ba}
	\end{subfigure}
	\begin{subfigure}{0.45\linewidth}
		\includegraphics[width=\textwidth]{n2phys/dmso-aa}%
		\caption{}%
		\label{def:fgr:n2phys-dmso-aa}
	\end{subfigure}

	\caption{A selection of nitrogen sorption isotherms as measured on the
		leached samples: (a) formic acid in DMF, (b) trifluoroacetic
		acid in water, (c) benzoic acid in methanol and (d) acetic acid
        in DMSO. The curve for the parent material is in dark blue.
        The x axis is logarithmic for clarity of low pressure 
        points.}%
	\label{def:fgr:n2phys-dataset}
\end{figure}

It is immediately apparent that the leaching process had an influence 
on the adsorption characteristics of UiO-66. Isotherms of acid treated
samples diverge from the parent material, at often at different pressures.
In general, the total molar capacity at full loading is seen to increase,
a telltale sign of an increase in pore volume through defect generation.
On the other hand, other particularities exist, such as the benzoic acid
samples in methanol and DMSO where the low and high
concentration has opposite effect on the total loading. Predictors 
of defects and trends will be analysed in the following section.
% !TEX root = ../../main.tex

\subsection{Vapour adsorption}

The effects of shaping with \(\rho\)-alumina are 
so far more subtle than the changes encountered when using 
PVA, as shown in the corresponding 
study~\cite{chanutObservingEffectsShaping2016}.
The influence of the binder on hydrophobic character of the
material may be of interest for tuning the properties of the 
beads. Adsorption of water and methanol vapour can serve as 
a probe for small changes is surface properties.
To this end, the same PVA samples which were 
studied in the previous study were studied alongside 
the MRA-shaped MOF.

Due to its surface charges, alumina is a 
hydrophilic substance, with a contact 
angle of \SI{10}{\degree}. It is expected that its 
addition will therefore increase the affinity 
of the resulting pellet towards water. On the other hand,
the PVA binder is more hydrophilic, with a water contact
angle of \SI{51}{\degree}. The medium affinity for water
is due to the surface hydroxyl functionalizations, which
can lead to hydrogen bonding.

Two indicators may highlight 
changes in material hydrophilicity: adsorption in the 
low relative pressure region (\(p/p^0 < 0.3\)) and 
condensation steps in the isotherm. The adsorption at low
pressures is representative of the first interactions with the 
surface, as explained in the previous sections. The pressure 
at which condensation occurs in the pores of the material, or 
where a sharp isotherm increase is seen depends on the 
size of the pore, but also on the pore environment and 
guest-guest interactions.

The measured isotherms on water and methanol can be found
in Figure~\ref{fgr:shaping:wateradsorption} 
and~\ref{fgr:shaping:methanoladsorption} respectively.
The isotherms measured on the original powder materials 
show the normal adsorption behaviour which can then be 
compared with the MRA and PVA shaped samples. 

On UiO-66(Zr), the water isotherm shows a slow uptake at the 
start, corresponding to a hydrophobic material and then shows 
a small step at \(p/p^0 = 0.3\). Complete pore filling happens 
at \(p/p^0 = 0.8\).

Water isotherms on MIL-100(Fe) show a more hydrophilic environment,
with a higher initial uptake, and two condensation steps at 
\(p/p^0 = 0.3\) and \(p/p^0 = 0.5\). There is pronounced 
hysteresis on the desorption branch.
The methanol adsorption isotherms also show a two step adsorption
isotherm, but without any hysteresis present. The shaped samples 
also show a lower 

It can be therefore concluded that overall, the addition of either 
binder has not influenced the behaviour towards water or 
methanol for the materials studied. The full water adsorption isotherms
can be found in Figure~\ref{fgr:shaping:wateradsorption}.

\begin{figure}[p!]
    \centering

    \begin{subfigure}{\linewidth}
        \centering
        \parbox{0.1\linewidth}{\caption{}\label{fgr:shaping:wateruio66}}%
        \includegraphics[width=0.4\textwidth]{water/UiO-66(Zr)-water}%
        \includegraphics[width=0.4\textwidth]{water/UiO-66(Zr)-water-log}%
    \end{subfigure}

    \begin{subfigure}{\linewidth}
        \centering
        \parbox{0.1\linewidth}{\caption{}\label{fgr:shaping:watermil100}}%
        \includegraphics[width=0.4\textwidth]{water/MIL-100(Fe)-water}%
        \includegraphics[width=0.4\textwidth]{water/MIL-100(Fe)-water-log}%
    \end{subfigure}

    \begin{subfigure}{\linewidth}
        \centering
        \parbox{0.1\linewidth}{\caption{}\label{fgr:shaping:watermil127}}%
        \includegraphics[width=0.4\textwidth]{water/MIL-127(Fe)-water}%
        \includegraphics[width=0.4\textwidth]{water/MIL-127(Fe)-water-log}%
    \end{subfigure}
    
    \caption{Water adsorption isotherms (a) UiO-66(Zr), 
    (b) MIL-100(Fe) and (c) MIL-127(Fe). The powder samples are in light
    blue, while the \(\rho\)-alumina and poly-vinyl alcohol samples are in red
    and dark blue respectively. Logarithmic graphs of the isotherms are
    on the right for clarity of the low
    pressure region.}%
    \label{fgr:shaping:wateradsorption}
\end{figure}


\begin{figure}[p!]
    \centering

    \begin{subfigure}{\linewidth}
        \centering
        \parbox{0.1\linewidth}{\caption{}\label{fgr:shaping:methanoluio66}}%
        \includegraphics[width=0.4\textwidth]{methanol/UiO-66(Zr)-methanol}%
        \includegraphics[width=0.4\textwidth]{methanol/UiO-66(Zr)-methanol-log}%
    \end{subfigure}

    \begin{subfigure}{\linewidth}
        \centering
        \parbox{0.1\linewidth}{\caption{}\label{fgr:shaping:methanolmil100}}%
        \includegraphics[width=0.4\textwidth]{methanol/MIL-100(Fe)-methanol}%
        \includegraphics[width=0.4\textwidth]{methanol/MIL-100(Fe)-methanol-log}%
    \end{subfigure}

    \begin{subfigure}{\linewidth}
        \centering
        \parbox{0.1\linewidth}{\caption{}\label{fgr:shaping:methanolmil127}}%
        \includegraphics[width=0.4\textwidth]{methanol/MIL-127(Fe)-methanol}%
        \includegraphics[width=0.4\textwidth]{methanol/MIL-127(Fe)-methanol-log}%
    \end{subfigure}
    
    \caption{Methanol adsorption isotherms (a) UiO-66(Zr), 
    (b) MIL-100(Fe) and (c) MIL-127(Fe). The powder samples are in light
    blue, while the \(\rho\)-alumina and poly-vinyl alcohol samples are in red
    and dark blue respectively. Logarithmic graphs of the isotherms are
    on the right for clarity of the low
    pressure region.}%
    \label{fgr:shaping:methanoladsorption}
\end{figure}

\subsection{Room temperature gas adsorption and microcalorimetry}

Combining microcalorimetry with adsorption manometry is a powerful technique which can
give an insight into the strength of the interactions during the adsorption process,
by directly measuring the differential heat. Even though the different contributions 
to the overall enthalpy curve cannot be decoupled from the individual sources, 
such as gas-adsorbent interactions, gas-gas interactions or confinement effects, it 
is well suited for observing the effect of a process or treatment such as shaping
on the properties of a MOF.

A wide variety of probe gasses has been chosen for adsorption at \SI{303}{\kelvin}:
\ce{N2}, \ce{CO}, \ce{CO2}, \ce{CH4}, \ce{C2H6}, \ce{C3H6}, \ce{C3H8} and \ce{C4H10}.
The range of adsorbates chosen allows different effects to be investigated.
The adsorption of saturated hydrocarbons with an increasing carbon number (C1-C4) can be
assumed to be driven strictly by Van-der-Waals forces, due to the shielding effects 
of the hydrogen atoms. Differences in the uptakes of these gasses will point to 
loss of porosity or crystalinity. An assymetric capacity loss with the larger molecules 
will point to size exclusion effects induced by the binder, such as particle coating,
pore filling or pore obstruction.
The other probes have been chosen for their 
individual properties which can shine light on other specific interaction types 
present during the adsorption. 
Carbon monoxide is a slightly dipolar molecule which has the ability to interact with
other charges in the pores. It also can highlight CUS (coordinatively unsaturated sites)
generated through defects, reduction or open metal sites due to its
propensity for \( \pi \) backbonding coordination.
This electron transfer process also can result in complexation with molecular 
orbitals in systems with \( \pi \) bonds such as alkenes and alkynes. Propylene is used 
as a probe gas for this purpose.
Carbon dioxide is a highly quadrupolar molecule which will be strongly adsorbed in 
polar pore environments. Changes in the adsorption behaviour of \ce{CO2} will shed 
light on such surface changes and can even be used as a predictor of 
hydrophobicity~\cite{chanutScreeningEffectWater2017}.
Finally, \ce{N2} is a staple adsorbent for material characterisation when used at
\SI{77}{\kelvin}. The molecule is a slight quadrupole and has also been shown to 
chelate to some transitional metals in an analogue fashion to \ce{CO}.

To eliminate the influence of kinetic and diffusion effects on the experiments,
care has been taken to allow time for complete equilibration of both pressure
and calorimeter signal.

After collecting the combined isotherm enthalpy data, three indicators have been chosen
to best represent the effects of shaping: initial enthalpy of adsorption, initial 
Henry constant and maximum capacity. These numeric performance indicators have been
calculated programmatically in Python using a specially developed high throughput
package.
The initial enthalpy of adsorption extrapolated at zero coverage is a measure of the 
interaction with highest energetic sites on the MOF surface. Conversely, the 
\(K_H\) obtained by applying the linear Henry model at the lowest loading points
is also an indication of adsorption in the pores before any 
layering or adsorbate-adsorbate interaction comes into effect. The \(K_H\) was 
calculated by fitting the virial adsorption model (Equation~\ref{eqn:virial})
to the isotherm.
The last indicator, maximum capacity, was taken as the loading attained when 
the isotherm reached a plateau. In the case of probes where the plateau was outside 
the range of pressure of the instrumentation (>\SI{50}{\bar}), the loading at the 
highest available pressure was considered as a suitable approximation.
The three KPIs (key performance indicators) have then been compared side by 
side on both the powder and shaped samples. 

\begin{equation}
    \label{eqn:virial}
    \ln{\frac{n}{P}} = \ln{K_H} + An + Bn^2 \cdots
\end{equation}

The complete dataset of adsorption isotherms, in the basis of both mass and volume 
can be found in the Supplementary Information (Figures~\ref*{fgr:calouio66}~,
\ref*{fgr:calomil100}~~and~\ref*{fgr:calomil127}~).

\subsubsection{UiO-66(Zr)}

A visual inspection of the enthalpy curves on UiO-66(Zr) show it to have a 
relatively homogenous surface, with flat profiles being common.

The KPI graphs show very similar values for both Henry's constant and initial 
enthalpy of adsorption. It is therefore apparent that the shaping process did not 
change the interaction of the adsorbate with the MOF surface.

The maximum capacity graphs show a more interesting trend. When using small adsorbates
such as \ce{N2}, \ce{CO2} and \ce{CH4}, the shaped samples have a similar performance 
on a mass basis and, due to the densification process, better capacities on a volume
basis. Starting with ethane, the maximum capacity starts to decrease, with the performance
worsening with increasing molecule size. On hydrocarbons with a carbon number of 3 and 4,
both mass basis and volume basis capacity is decreased compared to the original powder.
Carbon monoxide is an apparent outlier to this trend, with a decreased maximum capacity
and a small molecular size. 
This size exclusion effect could be explained by the coating of crystal surfaces with 
the alumina binder.
It could be argued that instead of size exclusion, the decrease is due to
a decrease in pore volume, and that the isotherms of the low molecular 
weight gasses will diverge at higher pressures as the pores are filled. 
A counterargument for this hypothesis is that on \ce{CO} a divergence 
is observed while the isotherm is still in the Henry's region and in the 
case of \ce{CO2}, the plateau is reached with no differences between the 
powder and the pellet.

Overall, the shaping performance of UiO-66(Zr) is 
reasonable, as long as only small adsorbates are used.

\begin{figure}
    \centering
    \begin{subfigure}{0.8\textwidth}
        \parbox[c]{0.1\linewidth}{\caption{}\label{fig:analysisuio66henry}}%
        \parbox[b]{0.7\linewidth}{%
        \includegraphics[width=\textwidth]{UiO-66(Zr)-henry-distribution}%
        }%
    \end{subfigure}

    \begin{subfigure}{0.8\textwidth}
        \parbox[c]{0.1\linewidth}{\caption{}\label{fig:analysisuio66enth}}%
        \parbox[b]{0.7\linewidth}{%
        \includegraphics[width=\textwidth]{UiO-66(Zr)-enthalpy-distribution}%
        }%
    \end{subfigure}

    \begin{subfigure}{0.8\textwidth}
        \parbox[c]{0.1\linewidth}{\caption{}\label{fig:analysisuio66basis}}%
        \parbox[b]{0.7\linewidth}{%
        \includegraphics[width=\textwidth]{UiO-66(Zr)-mass-volume}%
        }%
    \end{subfigure}
    
    \caption{UiO-66(Zr) analysis}%
    \label{fig:analysisuio66}
\end{figure}

\subsubsection{MIL-100(Fe)}

The enthalpy profiles on MIL-100(Fe) are less homogenous than the ones on UiO-66(Zr). 
Some effects can
be seen with probes which can interact with the partially reduced Fe(II) atom, such as 
carbon monoxide and propylene. 
Indeed, when comparing both the initial Henry constants and enthalpy of adsorption,
for \ce{CO} and \ce{C3H6}, these are higher than the values obtained on UiO-66(Zr).
With initial enthalpy of adsorption for \ce{CO} of around \SI{45}{\kilo\joule\per\mol},
the value falls into the range of previous~\cite{yoonControlledReducibilityMetalOrganic2010}
results for interactions with such Fe(II) CUS.

Comparing the powder and shaped variants, there are no 
apparent differences between the two. The only discrepancy, which can be seen on the 
nitrogen \(K_{H, init}\) follows as a result of an ill-fitting virial parameter,
and can be assumed an error after observing the isotherm overlap directly. It could be 
theorised that by activation at a higher temperature (\SI{250}{\degreeCelsius}),
the percentage of iron trimers which would undergo reduction will increase and 
a larger adsorption enthalpy could be observed. However, the
activation temperature was chosen to maintain comparability with a previous 
study~\cite{chanutObservingEffectsShaping2016}.

The maximum loading differences of the MIL-100(Fe) show a very homogenous distribution.
On all probes tested, a fixed capacity loss of between 10-20\% can be seen on a mass 
basis. However, the increase in density afforded by the compression during pelletisation
leads to a compensation in performance as can be seen in Figure~\ref{fig:analysismil100basis}.

We can conclude that MIL-100(Fe) is almost unaffected by alumina shaping. A slight loss
in maximum capacity on a mass basis is compensated by a pronounced densification, 
which is desirable in an industrial setting.

\begin{figure}
    \centering
    \begin{subfigure}{0.8\textwidth}
        \parbox[c]{0.1\linewidth}{\caption{}\label{fig:analysismil100henry}}%
        \parbox[b]{0.7\linewidth}{%
        \includegraphics[width=\textwidth]{MIL-100(Fe)-henry-distribution}%
        }%
    \end{subfigure}

    \begin{subfigure}{0.8\textwidth}
        \parbox[c]{0.1\linewidth}{\caption{}\label{fig:analysismil100enth}}%
        \parbox[b]{0.7\linewidth}{%
        \includegraphics[width=\textwidth]{MIL-100(Fe)-enthalpy-distribution}%
        }%
    \end{subfigure}

    \begin{subfigure}{0.8\textwidth}
        \parbox[c]{0.1\linewidth}{\caption{}\label{fig:analysismil100basis}}%
        \parbox[b]{0.7\linewidth}{%
        \includegraphics[width=\textwidth]{MIL-100(Fe)-mass-volume}%
        }%
    \end{subfigure}
    
    \caption{MIL-100(Fe) analysis}%
    \label{fig:analysismil100}
\end{figure}

\subsubsection{MIL-127(Fe)}

The isotherms on MIL-127(Fe) show similar behaviour as the MIL-100(Fe) material,
although with a sharper uptake as a result of the smaller pores. Enthalpy 
profiles are also influenced by the similar interactions with the iron 
trimers leading to higher initial heats of adsorption on \ce{CO} and \ce{C3H6}.
An unexpected increase in the heat of adsorption is seen with 
butane adsorption (Figure~\ref{fgr:mil127c4h10ads}). Due to the dual pore 
type in the MIL-127(Fe) structure,
it is likely that adsorption first commences in the small (\( \sim \)\SI{6}{\angstrom})
channels and then, at higher pressures, intrusion into the larger cage-type
pores is possible through the narrow apertures of \( \sim \)\SI{3}{\angstrom}. 
The confined cages have an increased interaction with the molecule which 
leads to the higher enthalpy values.

When observing the comparison between the powder and the pellet variant, a large
difference in the initial \(K_H\) on \ce{CO} stands out as the only major change.
The value of the initial enthalpy of adsorption does not follow the same pattern.
However, visual inspection of the isotherm (Figure~\ref{fgr:mil127coads})
and the enthalpy curve shows that the energy of adsorption corresponding to
interactions with the more active sites is maintained for a larger pressure 
range. This points to the higher preponderence of such sites in the powder
variant. A similar offset can be seen in the propylene enthalpy at very low
pressures, but this is not reflected in the shape of the isotherm. The weaker 
complexation strength and the larger size of the molecule likely limits the effect 
seen in the carbon monoxide isotherm. As for the underlying reason behind the 
isotherm divergence, it could be that the alumina binder acts as a protector against the 
generation of iron(II) during thermal activation.
No other differences are seen between the two forms on either Henry constant and
initial enthalpy of adsorption.

\begin{figure}[ht]
    \begin{subfigure}{0.5\textwidth}
        \includegraphics[width=\textwidth]{calo/MIL-127(Fe)/c4h10-mass-basis-log-iso}
        \caption{\ce{C4H10} isotherms on MIL-127(Fe)}%
        \label{fgr:mil127c4h10ads}
    \end{subfigure}
    \begin{subfigure}{0.5\textwidth}
        \includegraphics[width=\textwidth]{calo/MIL-127(Fe)/co-mass-basis-iso}
        \caption{CO isotherms on MIL-127(Fe)}%
        \label{fgr:mil127coads}
    \end{subfigure}%
    \label{fgr:mil127isotherms}
\end{figure}

The capacity comparison in Figure~\ref{fig:analysismil127basis} paints an interesting 
picture. For most probes there is no change in maximum loading showing that there is no
structure degradation or pore filling. Two outliers are apparent: carbon monoxide and 
butane. The decrease in capacity on \ce{CO} can be explained through the 
aforementioned changes in active site prevalence. The drop in butane cannot be a 
consequence of the same effect as there is a perfect overlap in the enthalpy curves.
Therefore it best explained through a size exclusion effect as seen on UiO-66(Zr).

Overall, MIL-127(Fe) shows excellent performance when undergoing alumina shaping, with 
almost no capacity loss, as long as the adsorbent is not carbon monoxide or butane, 
where specific effects come into play.

\begin{figure}
    \centering
    \begin{subfigure}{0.8\textwidth}
        \parbox[c]{0.1\linewidth}{\caption{}\label{fig:analysismil127henry}}%
        \parbox[b]{0.7\linewidth}{%
        \includegraphics[width=\textwidth]{MIL-127(Fe)-henry-distribution}%
        }%
    \end{subfigure}
    
    \begin{subfigure}{0.8\textwidth}
        \parbox[c]{0.1\linewidth}{\caption{}\label{fig:analysismil127enth}}%
        \parbox[b]{0.7\linewidth}{%
        \includegraphics[width=\textwidth]{MIL-127(Fe)-enthalpy-distribution}%
        }%
    \end{subfigure}

    \begin{subfigure}{0.8\textwidth}
        \parbox[c]{0.1\linewidth}{\caption{}\label{fig:analysismil127basis}}%
        \parbox[b]{0.7\linewidth}{%
        \includegraphics[width=\textwidth]{MIL-127(Fe)-mass-volume}%
        }%
    \end{subfigure}
    
    \caption{MIL-127(Fe) analysis}%
    \label{fig:analysismil127}
\end{figure}