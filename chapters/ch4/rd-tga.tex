% !TEX root = ../../main.tex

\subsection{Thermal stability}

In order to check if the shaped samples have not undergone bulk
structural changes, as well as find a suitable activation temperature,
the powder and shaped samples underwent thermogravimetric analysis
under an argon atmosphere.

The process of shaping did not have any impact on the thermal stability of
the investigated MOFs, as evidenced by the TGA curves in
\autoref{fgr:shaping:tgacurves}. The primary mass loss occurs
in a \SI{10}{\degreeCelsius} range for all powder-pellet pairs.
Shaped samples are also seen to have a smaller mass loss
at high temperatures. This is expected, as after the addition of
temperature inert alumina, the MOF makes up a lower percentage of
the material.

\begin{figure}
	\centering
	\begin{subfigure}{0.85\textwidth}
		\parbox[c]{0.1\linewidth}{\caption{}\label{fgr:shaping:tgauio66}}%
		\parbox[b]{0.7\linewidth}{%
			\includegraphics[width=\linewidth]{tga/uio66}%
		}%
	\end{subfigure}
	\begin{subfigure}{0.85\textwidth}
		\parbox[c]{0.1\linewidth}{\caption{}\label{fgr:shaping:tgamil100}}%
		\parbox[b]{0.7\linewidth}{%
			\includegraphics[width=\linewidth]{tga/mil100}%
		}%
	\end{subfigure}
	\begin{subfigure}{0.85\textwidth}
		\parbox[c]{0.1\linewidth}{\caption{}\label{fgr:shaping:tgamil127}}%
		\parbox[b]{0.7\linewidth}{%
			\includegraphics[width=\linewidth]{tga/mil127}%
		}%
	\end{subfigure}

	\caption{High resolution TGA curves recoded under argon
		on (a) UiO-66(Zr), (b) MIL-100(Fe) and (c) MIL-127(Fe). The
		original powders are depicted in red and the shaped material
		in blue.}%
	\label{fgr:shaping:tgacurves}

\end{figure}
