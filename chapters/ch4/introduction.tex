% !TEX root = ../../main.tex

\section{Introduction}

An adsorbent cannot be used in an industrial process in its raw powder
form as the small particles or crystals which are normally obtained during
synthesis are not suited for direct use. Therefore, in order to use a 
material in a setting such 
as in beds or columns common in catalysis, \gls{PSA} and \gls{TSA},
a structuring into a hierarchically porous structure form is 
needed~\cite{akhtarStructuringAdsorbentsCatalysts2014}.

To this end adsorbents are usually shaped into pellets, a process
which introduces a range of benefits, such as improved flow regimes,
better thermal management and material containment. The shaping process
is needed not just for stabilisation of small particles, but also to
impart the resulting pellet with a high enough mechanical resistance 
to withstand the stresses imposed by the high flow encountered in 
an industrial bed. Ideally, forming would have limited effects on
the properties of the material. However, this is often not the case, 
as shaping can dramatically change adsorption performance.

Shaping is therefore a crucial step towards the large-scale
use of an adsorbent material. Even for commonplace adsorbents such as
carbons and zeolites, the optimum binding additives and the shaping
process itself are the subject of extensive research. Often, the
procedure is tailored for an individual material and application.

As metal organic frameworks with properties that make them suited 
for industrial applications emerge, a push towards obtaining shaped
versions of the best-performing materials is desired, first for 
pilot studies and perhaps large-scale use.

Due to the wide range of materials, potential shaping methods, binders 
and effects introduced through the process itself, a high-throughput
methodology is often the kind of approach that is best suited to 
exhaustively explore the resulting space. The data processing methodologies
discussed in \autoref{pyg} combined with calorimetric methods
presented in \autoref{calo} are put to use in a study on the 
shaping performance of \glspl{MOF}.

\subsection*{Chapter summary}

After a short introduction to shaping, this chapter explores the 
variability introduced by an \( \rho \)-alumina binder in three 
topical \glspl{MOF}: UiO-66(Zr), MIL-100(Fe) and MIL-127(Fe). These \glspl{MOF} 
have been selected for their known chemical and thermal stability 
and well-studied adsorption behaviour.
The alumina shaped variant of these \glspl{MOF} is compared to the original
powder material with regard to the adsorption of a series of common
gases and vapours. Microcalorimetry in conjunction with 8 gas probes
has been used to get an in-depth picture of the change in surface
energetics.
Finally, a previous study on the same materials shaped with a \gls{PVA}
binder is extended to vapour adsorption and the entire
dataset is processed to obtain an overview of the impact of a
hydrophobic and a hydrophilic binder on adsorption performance.

\subsection*{Contributions}

Both UiO-66 and MIL-100 synthesis and shaping through granulation was performed
in the group of Prof.\ Jong-San Chang from the Research Group for Nanocatalysts,
Korea Research Institute of Chemical Technology (KRICT). Synthesis 
of MIL-127 was performed in the group of Dr.\ Christian Serre in 
the Institut Lavoisier, Versailles. Calorimetry,
thermogravimetry, nitrogen and vapour adsorption were carried out by
Paul Iacomi. Nicolas Chanut and Dr.\ Philip Llewellyn were of great help
in interpretation of results.

