% !TEX root = ../../main.tex

\section{Introduction}

An adsorbent cannot be used in an industrial process in its raw powder
form. The small crystals which are normally the synthesis product of
such a material would not be suited for direct use and need to
be transformed into hierarchically porous structures.

Therefore, adsorbents are usually shaped into pellets, a process
which introduces a range of benefits, such as improved flow regimes,
better thermal management and material containment. The shaping process
is needed not just for stabilising the small particles, but also to
impart the resulting pellet with a high
enough mechanical resistance to withstand the stresses imposed
by the high flow encountered in an industrial bed.

The shaping process is therefore a crucial step towards the large-scale
use of an adsorbent material. Even for commonplace adsorbents such as
carbons and zeolites, the optimum binding compound and the shaping
process itself are the subject of extensive research. Often, the
procedure is tailored for individual material and application.

In an ideal setting, the process has limited effects on the properties
of the material. However, this is often not the case, as shaping can degrade
or improve adsorption performance. After a short introduction to shaping,
this chapter explores the variability introduced by the binder in three
topical MOFs: UiO-66(Zr), MIL-100(Fe) and MIL-127(Fe), which have been
selected for their good stability and well-studied adsorption behaviour.
The alumina shaped variant of these MOFs is compared to the original
powder material, with regard to the adsorption of a series of common
gasses and vapours. Microcalorimetry in conjunction with 8 gas probes
has been used to get an in-depth picture of the change in surface
energetics, with a separate study of adsorption of water and methanol
vapour to examine changes in hydrophobicity.
Particular behaviours are then highlighted and discussed.

Finally, a previous study on the same materials shaped with a poly-vinyl
alcohol (PVA) binder is extended to vapour adsorption and the entire
dataset is processed to obtain an overview of the impact of a
hydrophobic and a hydrophilic binder on adsorption performance.
