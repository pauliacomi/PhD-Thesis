% !TEX root = ../../main.tex

\section{Synthesis, shaping and characterisation}

\subsection{Material Synthesis}

The synthesis of these MOFs is well described in literature, 
as it has been previously reported for 
UiO-66(Zr)~\cite{cavkaNewZirconiumInorganic2008} and
MIL-100(Fe)~\cite{YangWaterStableMetalOrganic2013}.

The UiO-66(Zr) and MIL-100(Fe) powders have been synthesised at the
Korea Research Institute of Chemical Technology (KRICT) according 
to the methods previously referenced. Complete details of the 
synthesis method can be found in the related 
publication~\cite{valekarShapingPorousMetal2017}
and in Appendix~\ref{appx:synthesis}.

\subsection{Shaping Procedure}

The shaping of the samples also took place at KRICT and was done
using a wet granulation method. In the case of the alumina binder,
the MOF powder was was mixed with the previously prepared mesoporous
\(\rho\)-alumina with water added as the dispersing medium. For the 
PVA binder, the MOF powder was instead added to a solution of 
ethanol solution containing a polymer mixture of polyvinyl groups
such as polyvinyl alcohol and polyvinyl butyral. The resulting 
mixture was shaped into beads using a hand-made pan granulator.
During the process, the spheres were sprayed with the respective 
solvent in order to achieve desired size. The beads were then sieved
and rolled using a roller machine to enhance their spherical
shape. Finally, the prepared samples were dried at \SI{303}{\kelvin}
for \SI{12}{\hour} to remove all residual solvent.
The resulting beads were near spherical in shape, with a diameter
between \SIrange{2}{2.5}{\milli\metre}.

\subsection{Characterisation of powders and pellets}

The primary interest of the study was observing differences 
in adsorption properties between the powder and the shaped materials.

Thermogravimetric analysis was used to verify that the binder 
did not change the thermal stability of the materials and, in the 
case of the PVA variant, to ensure that the activation temperature
chosen did not induce polymer decomposition. The TGA method is described 
in detail in Appendix~\ref{appx:char:TGA}.

\subsection{Sample activation for adsorption}

The materials were pre-treated before all adsorption experiments by activation at high 
temperature under secondary vacuum for 16 hours. The activation temperature was specific
to each solid: \SI{200}{\degreeCelsius} for UiO-66(Zr), \SI{150}{\degreeCelsius}
for MIL-100(Fe) and \SI{150}{\degreeCelsius} for MIL-127(Fe).
