% !TEX root = ../../main.tex

\section{Conclusion}

It can be concluded that the process of alumina shaping does not, in 
general, have an impact on the surface chemistry of the three MOF
materials. Interestingly, in regards to the changes in maximum capacity,
each material has its own particular behaviour: UiO-66(Zr) has a higher loss in 
capacity with larger molecular probes, MIL-100(Fe) has a decrease 
of 10-20\% across all gasses and MIL-127(Fe) matches loading on 
a mass basis except on \ce{CO} and \ce{C4H10}.

The shaping also induces a densification which, in almost all cases,
leads to a better performance on a volumetric basis. However, the 
influence on the mass transport effects of the alumina binder is 
not known and should be investigated further. 

Overall, the process of alumina shaping is a promising method of 
preparing MOFs for gas-related applications in separation and 
storage, but care should be taken to not generalise the effects
present on one material to another. 
