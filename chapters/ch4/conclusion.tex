% !TEX root = ../../main.tex

\pagebreak
\FloatBarrier%

\section{Conclusion}

It can be concluded that the process of alumina shaping does not, in 
general, have an impact on the surface chemistry of the three MOF
materials. Interestingly, in regards to the changes in maximum capacity,
each material has its own particular behaviour: UiO-66(Zr) has a 
higher loss in capacity with larger molecular probes, MIL-100(Fe) 
has a decrease of 10--20\% across all gasses and MIL-127(Fe) matches
loading on a mass basis except on \ce{CO} and \ce{C4H10}.

Water and methanol adsorption also further the hypothesis that 
the granulation process resulted in different
effects on the MOF samples studied. The surface 
characteristics of neither pellet type appear changed from the 
the powder variant, judging by the overlapping low pressure
areas of all isotherms. When compared to the parent material,
the alumina shaping method reduces 
the crystalinity of the samples studied, leading to lower uptakes.
On the other hand, the polymer shaped version is less prone to
amorphization, with the total adsorbed volume actually increasing
in the case of UiO-66(Zr).

The shaping also induces a densification which, in almost all cases,
leads to a better performance on a volumetric basis. However, the 
influence on the mass transport effects of the alumina binder is 
not known and should be investigated further. 

Overall, the process of alumina shaping is a promising method of 
preparing MOFs for gas-related applications in separation and 
storage, but care should be taken to not generalise the effects
present on one material to another. It is also worth noting that,
from an industrial point of view, alumina shaping holds promise 
as a method which does not cause the surface properties of 
the underlying material to change. Even if some loss of capacity
is observed, the volumetric densification counterbalances this effect. 

In the context of this thesis, the post-processing of adsorbent
materials is shown to be highly relevant to their adsorption behaviour,
generally negatively impacting properties such as pore volume and surface
area. 

\pagebreak
