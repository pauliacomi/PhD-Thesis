
Calculates the pore size distribution using a 'classical' model which attempts to
describe the adsorption in a pore as a combination of a statistical thickness and
a condensation/evaporation behaviour described by surface tension

Currently, the methods provided are:

    - the BJH or Barrett, Joyner and Halenda method
    - the DH or Dollimore-Heal method, an extension of the BJH method

A common mantra of data processing is: "garbage in = garbage out". Only use methods
when you are aware of their limitations and shortcomings.

According to Rouquerol, in adopting this approach, it is assumed that:

    - The Kelvin equation is applicable over the pore range (mesopores). Therefore
    in pores which are below a certain size (around 2.5 nm), the granularity
    of the liquid-vapour interface becomes too large for classical bulk methods
    to be applied.
    - The meniscus curvature is controlled be the pore size and shape. Ideal shapes
    for the curvature are assumed.
    - The pores are rigid and of well defined shape. They are considered
    open-ended and non-intersecting
    - The filling/emptying of each pore does not depend on its location.
    - The adsorption on the pore walls is not different from surface adsorption.

\subsection{Mesoporous size distribution}


The BJH or Barrett, Joyner and Halenda method for calculation of pore size distribution
is based on a classical description of the adsorbate behaviour in the adsorbent pores.
Under this method, the adsorbate is adsorbing on the pore walls in a predictable way,
and decreasing the apparent pore volume until condensation takes place, filling the
entire pore. The two variables, layer thickness and radius where 
condensation takes place can be modelled by a thickness model
(such as Halsey, Harkins \& Jura, etc.) and a
critical radius model for condensation/evaporation, based on a form of the Kelvin equation.

\begin{equation}
    r_p = t + r_k
\end{equation}

The original model used the desorption curve as a basis for calculating pore size distribution.
Between two points of the curve, the volume desorbed can be described as the volume contribution
from pore evaporation and the volume from layer thickness decrease as per the equation
above. The computation is done cumulatively, starting from the filled pores and 
calculating for each point the volume adsorbed in a pore from the following equation:

\begin{align}
    V_p & = \Big( \frac{\bar{r}_p}{\bar{r}_k + \Delta t_n} \Big)^2 \\
    (\Delta V_n - \Delta t_n \sum_{i=1}^{n-1} \Delta A_p \\
    + \Delta t_n \bar{t}_n \sum_{i=1}^{n-1} \frac{\Delta A_p}{\bar{r}_p})
    A & = 2 \Delta V_p / r_p
\end{align}

Where:

\begin{itemize}
    
    \item \(\Delta A_p\) is the area of the pores
    \item \(\Delta V_p\) is the adsorbed volume change between two points
    \item \(\bar{r}_p\) is the average pore radius calculated as a sum of the
    kelvin radius and layer thickness of the pores at pressure p between two
    measurement points
    \item \(\bar{r}_k\) is the average kelvin radius between two measurement points
    \item \(\bar{t}_n\) is the average layer thickness between two measurement points
    \item \(\Delta t_n\) is the average change in layer thickness between two measurement points
    
\end{itemize}

Then, by plotting \(\Delta V / (2*\Delta r_p)\) versus the width of the pores calculated
for each point, the pore size distribution can be obtained.

“The Determination of Pore Volume and Area Distributions in Porous Substances.
   I. Computations from Nitrogen Isotherms”, Elliott P. Barrett, Leslie G. Joyner and
   Paul P. Halenda, J. Amer. Chem. Soc., 73, 373 (1951)

"Adsorption by Powders \& Porous Solids", F. Roquerol, J Roquerol
   and K. Sing, Academic Press, 1999


\subsection{Microporous size distribution}

The H-K method attempts to describe the adsorption within pores by calculation
of the average potential energy for a pore. The method starts by assuming the
relationship between the gas phase as being:

\begin{equation}
    R_g T ln(\frac{p}{p_0}) = U_0 + P_a
\end{equation}

Here \(U_0\) is the potential function describing the surface to adsorbent
interactions and \(P_a\) is the potential function describing the adsorbate-
adsorbate interactions. This equation is derived from the equation of the free energy
of adsorption at constant temperature where term \(T \Delta S^{tr}(w/w_{\infty})\)
is assumed to be negligible.

If a Lennard-Jones-type potential function describes the interactions between the
adsorbate molecules and the adsorbent molecules then the two contributions to the
total potential can be replaced by the extended function. The resulting equation becomes:

\begin{align}
    RTln(p/p_0) =   & N_A\frac{n_a A_a + n_A A_A}{2 \sigma^{4}(l-d)} \\
                    & \times \int_{d/_2}^{1-d/_2}
                        \Big[
                        - \Big(\frac{\sigma}{r}\Big)^{4}
                        + \Big(\frac{\sigma}{r}\Big)^{10}
                        - \Big(\frac{\sigma}{l-r}\Big)^{4}
                        + \Big(\frac{\sigma}{l-r}\Big)^{4}
                        \Big] \mathrm{d}x
\end{align}

Where:

\begin{itemize}
    
    \item \(R\) -- gas constant
    \item \(T\) -- temperature
    \item \(l\) -- width of pore
    \item \(d\) -- defined as \(d=d_a+d_A\) the sum of the diameters of the adsorbate and
    \item adsorbent molecules
    \item \(N_A\) -- Avogadro's number
    \item \(n_a\) -- number of molecules of adsorbent
    \item \(A_a\) -- the Lennard-Jones potential constant of the adsorbent molecule defined as
    
        \begin{equation}
            A_a = \frac{6mc^2\alpha_a\alpha_A}{\alpha_a/\varkappa_a + \alpha_A/\varkappa_A}
        \end{equation}
        
    \item \(A_A\) -- the Lennard-Jones potential constant of the adsorbate molecule defined as
    
    \begin{equation}
            A_a = \frac{3mc^2\alpha_A\varkappa_A}{2}
    \end{equation}
        
    \item \(m\) -- mass of an electron
    \item \(c\) -- speed of light in vacuum
    \item \(\alpha_a\) -- polarizability of the adsorbate molecule
    \item \(\alpha_A\) -- polarizability of the adsorbent molecule
    \item \(\varkappa_a\) -- magnetic susceptibility of the adsorbate molecule
    \item \(\varkappa_A\) -- magnetic susceptibility of the adsorbent molecule
    
\end{itemize}

*Limitations*

The assumptions made by using the H-K method are:

\begin{itemize}
    
    \item It does not have a description of capillary condensation. This means that the
    pore size distribution can only be considered accurate up to a maximum of 5 nm.
    \item Each pore is uniform and of infinite length. Materials with varying pore
    shapes or highly interconnected networks may not give realistic results.
    \item The wall is made up of single layer atoms. Furthermore, since the HK method
    is reliant on knowing the properties of the surface atoms as well as the
    adsorbent molecules the material should ideally be homogenous.
    \item Only dispersive forces are accounted for. If the adsorbate-adsorbent interactions
    have other contributions, the Lennard-Jones potential function will not be
    an accurate description of pore environment.
    
\end{itemize}

References
----------
K. Kutics, G. Horvath, Determination of Pore size distribution in MSC from
Carbon-dioxide Adsorption Isotherms, 86


\subsection{Multiscale size distribution - DFT fitting}

The function will take the data in the form of pressure and loading. It will
then load the kernel either from disk or from memory and define a minimization
function as the sum of squared differences of the sum of all individual kernel
isotherm loadings multiplied by their contribution as per the following function:

\begin{equation}
    f(x) = \sum_{p=p_0}^{p=p_x} (n_{p,exp} - \sum_{w=w_0}^{w=w_y} n_{p, kernel} X_w )^2
\end{equation}