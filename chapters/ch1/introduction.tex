% !TEX root = ../../main.tex

\section{Introduction}

Historically, the processing of isotherms was done by hand, with large
worksheets being used for the calculations. As an example,
we point out that one of the initial limitations of the
Barrett-Joyner and Halenda method for determining pore size distribution
was that each point had to be estimated with an approximation of critical
pore radius, resulting in tedious work involved in the
calculation~\cite{barrettDeterminationPoreVolume1951}.

The advent of computers meant that calculations could be performed
quickly and reliably and led to the development of more computationally
demanding methods for adsorption modelling and isotherm processing, such as the 
density functional theory (DFT) and grand canonical monte carlo (GCMC) kernel fitting
methods for pore size distribution~\cite{seatonNewAnalysisMethod1989,%
	tarazonaPhaseEquilibriaFluid1987}.
Commercial adsorption equipment which offers the users
a complete software solution for any isotherm calculations is now
commonplace, and makes obtaining reports of desired properties
for measured materials a matter of seconds.

Given the current ubiquitousness of adsorption as a characterisation method,
particularly for investigating surfaces and porous compounds,
there is a large pool of data published in the scientific community.
Recent efforts have focused on building a database of adsorption
isotherms~\cite{sideriusNISTARPAEDatabase2015}, to offer a searchable
pool of standardised behaviours on different materials. This serves as
both a useful reference for comparing synthesised compounds, as well as a
method for quickly finding suitable materials which have the
desired properties for a particular application.

The abundance of data reflects a trend in recent years where, due
to the prevalence of digital records and cheap storage, more
information is available than ever before. As such, challenges
now lie in making sense of the so called ``big data''.
Key performance indicators (KPI) such as specific surface area, working
capacity and pore volume are commonly reported in scientific literature
and used as benchmarking tools. Furthermore,
several studies have attempted to find a reliable indicator for
the suitability of an adsorbent for a specific
application~\cite{regeSimpleParameterSelecting2001, %
	ackley2000psa,%
	wiersumAdsorbentPerformanceIndicator2013} such as pressure swing
adsorption.

However, we feel that a critical step in the process has yet
to be fully addressed. Transitioning from isotherms published in
literature to the calculation of such KPIs requires individual
implementation of previously mentioned techniques. As such,
not only can errors arise in such implementations, but
a critical survey of the results is tedious and time consuming.
In conclusion, a standardised method of high throughput processing
of such data would be invaluable not only for reference purposes,
but also for in-depth studies of data reliability,
structure-property relationships and effect of variables on
the adsorption properties of a material.

\subsection*{Chapter summary}

In this chapter an open-source software package is presented, which
is released under an MIT licence and written in Python, intended to be
used for manipulation, storage, visualisation and processing of
adsorption isotherms. The software is aimed to give users a powerful
yet easy to use package that can perform the kind of processing
common to adsorption methodology, but using large datasets of hundreds
or thousands of isotherms. The code is then used to process a
dataset of 26000 isotherms from the NIST adsorption database in order
to highlight the variability of adsorption data reported in literature
on MOFs.

\subsection*{Contribution}

Paul Iacomi wrote the python code for the pyGAPS framework and is
responsible for its publication and maintenance as an open source
package. The IAST functionality is a modified version of
the pyIAST code, published by Cory Simon. Example data on reference 
materials was recorded by Paul Iacomi for Takeda 5A and UiO-66(Zr) 
and by Emily Bloch on MCM-41 and porous silica. The bulk processing
example data comes from the NIST Adsorption Database, 
maintained by Dan Siderius.
