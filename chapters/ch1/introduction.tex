% !TEX root = ../../main.tex

\section{Introduction}

Historically, the processing of isotherms was done by hand, with large 
worksheets being used for the calculations. As an example, 
we point out that one of the initial limitations of the BJH method was 
that each point had to 
be determined with an approximation of critical pore radius, due to the 
tedious work involved in the 
calculation~\cite{barrettDeterminationPoreVolume1951}.

The advent of computers meant that the calculations could be performed
quickly and reliably and led to the introduction of more complex
methods for isotherm processing, such as the DFT method for pore size
distribution~\cite{seatonNewAnalysisMethod1989,%
tarazonaPhaseEquilibriaFluid1987}. 
Commercial adsorption equipment which offers the users
a complete software solution for any isotherm calculations is now
commonplace and makes obtaining reports of desired properties
for measured materials a matter of minutes.

Given the current ubiquitousness of adsorption as a characterisation method,
particularly for investigating surfaces and porous compounds,
there is a large pool of data published in the scientific community.
Key performance indicators such as specific surface area, working 
capacity and pore volume are commonly reported in scientific literature
and used as benchmarking tools for comparing performance.

Recent efforts have also focused on building a database of adsorption 
isotherms~\cite{sideriusNISTARPAEDatabase2015}, to offer a searchable
pool of standardised behaviours on different materials. This serves as both a
useful reference for comparing synthesised compounds, as well as a
method for quickly finding suitable materials which have the
desired properties for a particular application.

\subsection*{Chapter summary}

In this chapter an open-source software package is presented, which 
is released under an MIT licence and written in Python, intended to be
used for manipulation, storage, visualisation and processing of
adsorption isotherms. Developed internally at the MADIREL Laboratory in
Marseilles, the software is aimed to give users a powerful yet easy to
use package that can perform the kind of processing usually offered by
commercial software, but using large datasets of hundreds or thousands
of isotherms.

\subsection*{Contribution}

Paul Iacomi wrote the python code for the pyGAPS framework and is 
responsible for its publication and maintenance as an open source
package. The IAST functionality is a modified version of
the pyIAST code, published by Cory Simon. The data used in this chapter 
is available from the NIST Adsorption Database, maintained by Dan Siderius.

