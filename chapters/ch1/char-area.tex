% !TEX root = ../../main.tex

\subsection{Specific surface area and pore volume calculation}

\subsubsection{\gls{BET} surface area}\label{pyg:charac:betarea}

The \gls{BET} method~\cite{brunauerAdsorptionGasesMultimolecular1938}
for determining surface area is the recommended \gls{IUPAC}
method~\cite{thommesPhysisorptionGasesSpecial2015}
to calculate the surface area of a porous material.
It is generally applied on isotherms obtained through \ce{N2}
adsorption at \SI{77}{\kelvin}, although other adsorbates
(\ce{Ar} at \SI{77}{\kelvin} or \SI{87}{\kelvin},
\ce{Kr} at \SI{77}{\kelvin}, \ce{CO2} at \SI{293}{\kelvin})
have been used. In principle, any probe with an adsorption isotherm
which can be described through the \gls{BET} equation in the low pressure regime
can be used.

As previously mentioned, \autoref{pyg:models:bet} the \gls{BET} model assumes
that adsorption takes place in incremental layers on the material
surface. Even if the adsorbent is porous, the initial amount adsorbed
(usually between 0.05--0.4 \(p/p_0\)) can be
described through the equation written in its linear form:
%
\begin{equation}
	\frac{p/p_0}{n_{ads} (1-p/p_0)} = \frac{1}{n_{ads}^m C} + \frac{C - 1}{n_{ads}^m C}(p/p_0)
\end{equation}

If a plot of the isotherm points as \({(p/p_0)}/{n_{ads}(1-p/p_0)}\)
versus \(p/p_0\) is generated, a linear region
can usually be found. The slope and intercept of this line
can then be used to calculate \gls{namon}, the amount adsorbed at the
statistical monolayer, as well as \gls{C}, the \gls{BET} constant.
%
\begin{align}
	n_{ads}^{m} & = \frac{1}{s+i} & C & = \frac{s}{i} + 1
\end{align}

From the \gls{BET} monolayer capacity, the specific surface area can be
calculated if the area taken up by one of the adsorbate molecules
on the surface is known. The calculation uses the following equation
together with Avogadro's number:
%
\begin{equation}
	A_{BET} = n_{ads}^m N_A \sigma
\end{equation}

While commonly used for surface area determination, the \gls{BET} area
should be used with care, as the assumptions made in
its calculation may not hold. To augment the validity of the \gls{BET}
method, Rouquerol~\cite{rouquerolAdsorptionPowdersPorous2013} proposed
several checks to ensure that the selected \gls{BET} region is valid:

\begin{itemize}

	\item The \gls{BET} (\gls{C}) obtained should be positive;
	\item In the corresponding Rouquerol plot where \(n_{ads}(1-p/p_0)\)
	      is plotted with respect to \(p/p_0\), the points chosen for \gls{BET}
	      analysis should be strictly increasing;
	\item The loading at the statistical monolayer should be
	      situated within the limits of the \gls{BET} region.

\end{itemize}

Regardless, the \gls{BET} surface area should still be interpreted carefully.
Since adsorption takes place on the pore surface, microporous materials
which have pores of similar or smaller size as the probe molecule used
will not give a realistic surface area. In the micropore range
it is difficult to separate pore condensation behaviour from
multilayer adsorption. Furthermore, the cross-sectional
area of the molecule on the surface cannot be guaranteed. For example,
nitrogen has been known to adopt a different conformation on the surface
of some materials due to inter-molecular forces, which effectively
lowers its cross-sectional area~\cite{rouquerolAdsorptionPowdersPorous2013}
from \SI{0.168}{\nano\metre} to \SI{0.138}{\nano\metre}.

\subsubsection{Langmuir surface area}\label{pyg:charac:langmuirarea}

The Langmuir equation (\autoref{pyg:eqn:langmuir}) can be also
be expressed in a linear form by rearranging it as:
%
\begin{equation}
	\frac{p}{n_{ads}} = \frac{1}{K n_{ads}^m} + \frac{p}{n_{ads}^m}
\end{equation}
%
Assuming the data can be fitted with a Langmuir model, by plotting
\({p}/{n_{ads}}\) against pressure, a straight line will be obtained.
The slope and intercept of this line can then be used to calculate
\gls{namon}, the amount adsorbed at the monolayer capacity, as well
as \(K\), the Langmuir constant.
%
\begin{align}
	n_{ads}^m & = \frac{1}{s} & K & = \frac{1}{i \times n_{ads}^m}
\end{align}
%
The surface area can then be calculated by using the monolayer
capacity in a manner analogous to the \gls{BET} surface area method.
%
\begin{equation}
	A_{Langmuir} = n_{ads}^m N_A \sigma
\end{equation}

The Langmuir method for determining surface area assumes that
adsorption takes place until all active sites on the material
surface are occupied, or until monolayer formation in the
case of complete surface coverage. Most adsorption processes
(except chemisorption) don't follow this behaviour, although 
high temperature isotherms can often be estimated through this 
model. Therefore it is important to regard the Langmuir surface 
area as an estimate.

\subsubsection{Ideal isotherms or thickness functions}\label{pyg:charac:tcurve}

The initial part of an isotherm (the Henry regime) can be seen to
be very dependent on the interactions between the adsorbate and the
surface. However, subsequent layers are less influenced by the
material and can often be assumed, like in the \gls{BET} model,
to have an energy of adsorption identical to the enthalpy of liquefaction
of the bulk liquid. Therefore, their formation will essentially depend
only on partial pressure.

With this assumption, several studies have been focused on obtaining
an ``ideal'' isotherm of adsorption on a non-porous material which can
then, if the cross-sectional area of the molecule is known, be transformed
in a function capable of predicting the thickness of the adsorbed layer
as a function of pressure. This kind of empirical function,
also referred to as a \textit{thickness function} or \textit{t-curve},
can be used as an alternative method for surface area determination,
as explained in the following section.
These curves are also used in the classical methods
for calculating mesoporous size distribution. It is important to
clarify that the function is only applicable for a single adsorbent
and temperature.

For example, two common thickness functions applicable for nitrogen
at \SI{77}{\kelvin} are the Halsey~\cite{halseyPhysicalAdsorptionNon1948}
(\autoref{pyg:eqn:halsey}) and the
Harkins and Jura~\cite{harkinsSurfacesSolidsXIII1944}
(\autoref{pyg:eqn:harkinsjura}) curves.
%
\begin{align}
	t_{Halsey}        & = 0.354 {\Big(\frac{-5}{\log(p/p_0)}\Big)}^{1/3}            %
	\label{pyg:eqn:halsey}                                                          \\
	t_{Harkins\&Jura} & = {\Big(\frac{0.1399}{0.034 - \log_{10}(p/p_0)}\Big)}^{1/2} %
	\label{pyg:eqn:harkinsjura}
\end{align}

\subsubsection{t-plot method}\label{pyg:charac:tplot}

The t-plot method is an empirical method developed as a
tool to determine the surface area of porous materials,
which can also be used for other calculations, such as
external surface area and pore volume~\cite{lippensStudiesPoreSystems1965}.
A plot is constructed, where the isotherm loading
data is plotted versus the ideal thickness of the adsorbate layer,
obtained through a t-curve (\autoref{pyg:charac:tcurve}).
It stands to reason that, in the case when the experimentally measured
isotherm conforms to the model, a straight line will be obtained with its
intercept through the origin. However, as in most cases there
are differences between adsorption in pores and surface multilayer
adsorption, the t-plot will deviate and reveal features which can
be analysed to obtain material characteristics. For example, a sharp
vertical deviation will indicate condensation in a type of mesopore
while a gradual slope will indicate adsorption on a specific surface.

The slope of a linear section can be used to calculate the area where
adsorption is taking place. If the linear region occurs at low loadings,
it will represent the total surface area of the material.
If at the end of the curve, it will instead represent adsorption on
external surface area. The formula to calculate an area starting
from the t-plot slope is presented in \autoref{pyg:fig:areatplot},
where \(\rho_{l}\) is the liquid density of the adsorbate at experimental
conditions.
%
\begin{equation}\label{pyg:fig:areatplot}
	A_{surface} = \frac{s M_m}{\rho_{l}}
\end{equation}

If the linear region selected is after a vertical deviation,
the intercept of the calculated line will no longer pass through
the origin. This intercept can be used to calculate the volume of
the filled pore through the following equation:
%
\begin{equation}
	V_{ads} = \frac{i M_m}{\rho_{l}}
\end{equation}

As the t-plot method compares the experimental isotherm
with an ideal model, care must be taken to ensure that the t-curve
is an accurate representation of the thickness of the adsorbate layer.
Since there is no such thing as a universal thickness curve,
a reliable model which is applicable to both material and adsorbate
should be chosen. It should also be noted that, features on the t-plot
found at loadings lower than the monolayer thickness may not have any
physical meaning.

\subsubsection{\(\alpha_s\) Method}\label{pyg:charac:alphasplot}

In order to extend the t-plot analysis with other adsorbents and
non-standard thickness curves, the \(\alpha_s\) method was
devised~\cite{atkinsonAdsorptivePropertiesMicroporous1984}.
Instead of attempting to find an ideal isotherm that describes the
thickness of the adsorbed layer, a reference isotherm is used.
This isotherm is measured on a non-porous version of the same material,
which is assumed to have identical surface characteristics.
The dimensionless \(\alpha_s\) values are obtained from this isotherm by
dividing the loading values by the amount adsorbed at a specific relative
pressure, usually taken as \(p/p_0=0.4\) since nitrogen hysteresis loops
theoretically close at this point.
%
\begin{equation}
	\alpha_s = \frac{n_{ads}}{n_{ads}^{0.4}}
\end{equation}

The analysis then proceeds as in the t-plot method, with the
same explanation for observed features. The only difference is
that the surface area calculation from linear regions observed
uses the known specific area of the reference material.
%
\begin{equation}
	A_{\alpha_s} = \frac{s A_{ref}}{{(n_{ads, ref}^{0.4}}
\end{equation}

The reference isotherm chosen for the \(\alpha_s\) method must
be a description of adsorption on a completely non-porous sample
of the same material. It is often impossible to obtain such
a version. Furthermore, an adsorption isotherm on a non-porous
solid may be a challenging endeavour, due to the small ratio
of surface area to volume.