% !TEX root = ../../main.tex

\section{Characterisation of materials through adsorption}

Adsorption is a powerful tool for characterisation of materials.
As it can give insight into the surface properties of solids, it
is most often used for the investigation of porous or finely
divided materials where the surface area to volume ratio is
high enough for adsorption to be detected with common laboratory
measurement techniques such as volumetry and gravimetry (see
\autoref{appx:char}). The most common properties which can
be obtained through adsorption methods are probe accessible
surface area, pore volume and pore size distribution. Other
characteristics such as surface chemistry or material response
can be probed through adsorption. Finally, pure component adsorption
isotherms can be used to predict multicomponent adsorption through
computational models such as IAST. The following section will go into
detail on how such methods are applied.

% !TEX root = ../../main.tex

\subsection{Specific surface area and pore volume calculation}

\subsubsection{BET surface area}\label{pyg:charac:betarea}

The BET method~\cite{brunauerAdsorptionGasesMultimolecular1938}
for determining surface area is the recommended IUPAC
method~\cite{thommesPhysisorptionGasesSpecial2015}
to calculate the surface area of a porous material.
It is generally applied on isotherms obtained through \ce{N2}
adsorption at \SI{77}{\kelvin}, although other adsorbates
(\ce{Ar} at \SI{77}{\kelvin} or \SI{87}{\kelvin},
\ce{Kr} at \SI{77}{\kelvin}, \ce{CO2} at \SI{293}{\kelvin})
have been used. In principle, any probe with an adsorption isotherm
which can be described through the BET equation in the low pressure regime
can be used.

As previously mentioned, \autoref{pyg:models:bet} the BET model assumes
that adsorption takes place in incremental layers on the material
surface. Even if the adsorbent is porous, the initial amount adsorbed
(usually between 0.05--0.4 \(p/p_0\)) can be
described through the equation written in its linear form:
%
\begin{equation}
	\frac{p/p_0}{n_{ads} (1-p/p_0)} = \frac{1}{n_{m} C} + \frac{C - 1}{n_{m} C}(p/p_0)
\end{equation}

If a plot of the isotherm points as \({(p/p_0)}/{n_{ads}(1-p/p_0)}\)
versus \(p/p_0\) is generated, a linear region
can usually be found. The slope and intercept of this line
can then be used to calculate \(n_{m}\), the amount adsorbed at the
statistical monolayer, as well as \(C\), the BET constant.
%
\begin{align}
	n_{m} & = \frac{1}{s+i} & C & = \frac{s}{i} + 1
\end{align}

From the BET monolayer capacity, the specific surface area can be
calculated if the area taken up by one of the adsorbate molecules
on the surface is known. The calculation uses the following equation
together with Avogadro's number:
%
\begin{equation}
	a_{BET} = n_m A_N \sigma
\end{equation}

While commonly used for surface area determination, the BET area
should be used with care, as the assumptions made in
its calculation may not hold. To augment the validity of the BET
method, Rouquerol~\cite{rouquerolAdsorptionPowdersPorous2013} proposed
several checks to ensure that the selected BET region is valid:

\begin{itemize}

	\item The BET (\(C\)) obtained should be positive;
	\item In the corresponding Rouquerol plot where \(n_{ads}(1-p/p_0)\)
	      is plotted with respect to \(p/p_0\), the points chosen for BET
	      analysis should be strictly increasing;
	\item The loading at the statistical monolayer should be
	      situated within the limits of the BET region.

\end{itemize}

Regardless, the BET surface area should still be interpreted carefully.
Since adsorption takes place on the pore surface, microporous materials
which have pores of similar or smaller size as the probe molecule used
will not give a realistic surface area. In the micropore range
it is difficult to separate pore condensation behaviour from
multilayer adsorption. Furthermore, the cross-sectional
area of the molecule on the surface cannot be guaranteed. For example,
nitrogen has been known to adopt a different conformation on the surface
of some materials due to inter-molecular forces, which effectively
lowers its cross-sectional area~\cite{rouquerolAdsorptionPowdersPorous2013}
from \SI{0.168}{\nano\metre} to \SI{0.138}{\nano\metre}.

\subsubsection{Langmuir surface area}\label{pyg:charac:langmuirarea}

The Langmuir equation (\autoref{pyg:eqn:langmuir}) can be also
be expressed in a linear form by rearranging it as:
%
\begin{equation}
	\frac{p}{n} = \frac{1}{K n_m} + \frac{p}{n_m}
\end{equation}
%
Assuming the data can be fitted with a Langmuir model, by plotting
\({p}/{n}\) against pressure, a straight line will be obtained.
The slope and intercept of this line can then be used to calculate
\(n_{m}\), the amount adsorbed at the monolayer capacity, as well
as \(K\), the Langmuir constant.
%
\begin{align}
	n_m & = \frac{1}{s} & K & = \frac{1}{i * n_m}
\end{align}
%
The surface area can then be calculated by using the monolayer
capacity in a manner analogous to the BET surface area method.
%
\begin{equation}
	a_{Langmuir} = n_m A_N \sigma
\end{equation}

The Langmuir method for determining surface area assumes that
adsorption takes place until all active sites on the material
surface are occupied, or until monolayer formation in the
case of complete surface coverage. Most adsorption processes
(except chemisorption) don't follow this behaviour, although 
high temperature isotherms can often be estimated through this 
model. Therefore it is important to regard the Langmuir surface 
area as an estimate.

\subsubsection{Ideal isotherms or thickness functions}\label{pyg:charac:tcurve}

The initial part of an isotherm (the Henry regime) can be seen to
be very dependent on the interactions between the adsorbate and the
surface. However, subsequent layers are less influenced by the
material and can often be assumed, like in the BET model,
to have an energy of adsorption identical to the enthalpy of liquefaction
of the bulk liquid. Therefore, their formation will essentially depend
only on partial pressure.

With this assumption, several studies have been focused on obtaining
an ``ideal'' isotherm of adsorption on a non-porous material which can
then, if the cross-sectional area of the molecule is known, be transformed
in a function capable of predicting the thickness of the adsorbed layer
as a function of pressure. This kind of empirical function,
also referred to as a \textit{thickness function} or \textit{t-curve},
can be used as an alternative method for surface area determination,
as explained in the following section.
These curves are also used in the classical methods
for calculating mesoporous size distribution. It is important to
clarify that the function is only applicable for a single adsorbent
and temperature.

For example, two common thickness functions applicable for nitrogen
at \SI{77}{\kelvin} are the Halsey~\cite{halseyPhysicalAdsorptionNon1948}
(\autoref{pyg:eqn:halsey}) and the
Harkins and Jura~\cite{harkinsSurfacesSolidsXIII1944}
(\autoref{pyg:eqn:harkinsjura}) curves.
%
\begin{align}
	t_{Halsey}        & = 0.354 {\Big(\frac{-5}{\log(p/p_0)}\Big)}^{1/3}            %
	\label{pyg:eqn:halsey}                                                          \\
	t_{Harkins\&Jura} & = {\Big(\frac{0.1399}{0.034 - \log_{10}(p/p_0)}\Big)}^{1/2} %
	\label{pyg:eqn:harkinsjura}
\end{align}

\subsubsection{t-plot method}\label{pyg:charac:tplot}

The t-plot method is an empirical method developed as a
tool to determine the surface area of porous materials,
which can also be used for other calculations, such as
external surface area and pore volume~\cite{lippensStudiesPoreSystems1965}.
A plot is constructed, where the isotherm loading
data is plotted versus the ideal thickness of the adsorbate layer,
obtained through a t-curve (\autoref{pyg:charac:tcurve}).
It stands to reason that, in the case when the experimentally measured
isotherm conforms to the model, a straight line will be obtained with its
intercept through the origin. However, as in most cases there
are differences between adsorption in pores and surface multilayer
adsorption, the t-plot will deviate and reveal features which can
be analysed to obtain material characteristics. For example, a sharp
vertical deviation will indicate condensation in a type of mesopore
while a gradual slope will indicate adsorption on a specific surface.

The slope of a linear section can be used to calculate the area where
adsorption is taking place. If the linear region occurs at low loadings,
it will represent the total surface area of the material.
If at the end of the curve, it will instead represent adsorption on
external surface area. The formula to calculate an area starting
from the t-plot slope is presented in \autoref{pyg:fig:areatplot},
where \(\rho_{l}\) is the liquid density of the adsorbate at experimental
conditions.
%
\begin{equation}\label{pyg:fig:areatplot}
	a_{surface} = \frac{s M_m}{\rho_{l}}
\end{equation}

If the linear region selected is after a vertical deviation,
the intercept of the calculated line will no longer pass through
the origin. This intercept can be used to calculate the volume of
the filled pore through the following equation:
%
\begin{equation}
	V_{ads} = \frac{i M_m}{\rho_{l}}
\end{equation}

As the t-plot method compares the experimental isotherm
with an ideal model, care must be taken to ensure that the t-curve
is an accurate representation of the thickness of the adsorbate layer.
Since there is no such thing as a universal thickness curve,
a reliable model which is applicable to both material and adsorbate
should be chosen. It should also be noted that, features on the t-plot
found at loadings lower than the monolayer thickness may not have any
physical meaning.

\subsubsection{\(\alpha_s\) Method}\label{pyg:charac:alphasplot}

In order to extend the t-plot analysis with other adsorbents and
non-standard thickness curves, the \(\alpha_s\) method was
devised~\cite{atkinsonAdsorptivePropertiesMicroporous1984}.
Instead of attempting to find an ideal isotherm that describes the
thickness of the adsorbed layer, a reference isotherm is used.
This isotherm is measured on a non-porous version of the same material,
which is assumed to have identical surface characteristics.
The dimensionless \(\alpha_s\) values are obtained from this isotherm by
dividing the loading values by the amount adsorbed at a specific relative
pressure, usually taken as \(p/p_0=0.4\) since nitrogen hysteresis loops
theoretically close at this point.
%
\begin{equation}
	\alpha_s = \frac{n_a}{n_{0.4}}
\end{equation}

The analysis then proceeds as in the t-plot method, with the
same explanation for observed features. The only difference is
that the surface area calculation from linear regions observed
uses the known specific area of the reference material.
%
\begin{equation}
	A = \frac{s A_{ref}}{{(n_{ref})}_{0.4}}
\end{equation}

The reference isotherm chosen for the \(\alpha_s\) method must
be a description of adsorption on a completely non-porous sample
of the same material. It is often impossible to obtain such
a version. Furthermore, an adsorption isotherm on a non-porous
solid may be a challenging endeavour, due to the small ratio
of surface area to volume.
% !TEX root = ../../main.tex

\subsection{Assessing porosity}

Characterization of pore sizes and their distribution in a porous material
is often the main goal when performing adsorption experiments.
When it comes to such materials, three kinds of pore sizes can be
defined, based on their lengthscale: micropores (\SI{< 2}{\nano\metre}),
mesopores (\SIrange{2}{50}{\nano\metre}), and macropores (\SI{> 50}{\nano\metre}).

Macropores are generally unable to be characterised through adsorption,
with methods such as mercury intrusion porosimetry as the standard for pore
size determination at this lengthscale, although other alternatives have
been suggested~\cite{rouquerolCharacterizationMacroporousSolids2012} due
to the high toxicity of mercury.

In the mesopore range, \textit{``classical''} methods are often
used, which are based on the application of Kelvin's equation
pertaining to capillary condensation. This equation calculates the
critical pressure at which the fluid completely files a pore of a
specific diameter. It is applicable to a range of geometries
is used in multiple approaches such as the Barrett-Joyner-Halenda
(BJH) method or the Dollimore-Heal (DH) method.

For microporous materials, the Kelvin equation, with its assumption
of continuous fluid properties and comparable density of the adsorbed
state to bulk liquid density breaks down. An atomistic approach
is required, addressing the interaction between solid-fluid
and fluid-fluid through potential functions. The Horvath-Kawazoe or
HK method is often used, alongside the older Dubinin-Radushkevich
model.

Finally, computational methods based on Grand Canonical Monte Carlo
(GCMC) and density functional theory (DFT) together with its
derivations such as non-local DFT (NLDFT) and quenched solid state DFT
(QSDFT) should be mentioned, as they can be used for multiscale
(micropore and mesopore) characterisation. These methods rely on
\textit{in-silico} simulation of isotherms on a range of pore sizes,
which can then be collated in a so-called \textit{kernel}, able to
be used to deconvolute an experimental isotherm to obtain a
pore size distribution.

It should be noted that all methods here described require knowledge of
the pore geometry and depend on the material having pores which
conform to a well-defined shape. Real adsorbents usually have
interconnected networks of irregular pores, which the ideal pores
used in these models merely approximate.

\subsubsection{Mesoporous size distribution}

\paragraph{The Kelvin equation}

Since the Kelvin equation is the basis of most mesoporous PSD 
calculations, such as the BJH and DH methods, it will be described
first. The original form of the equation (\autoref{pyg:eqn:kelvin})
gives the dependence of pressure on the radius of curvature of a
meniscus in a pore \(r\) by means of surface tension \(\gamma\), 
molar liquid volume, here expressed as \(v_l=M_m/\rho_l\) and the
fluid contact angle with the surface
\(\theta\). The fluid is often assumed to be fully wetting, with
\(\theta=0\) and \(\cos\theta=1\).

\begin{equation}\label{pyg:eqn:kelvin}
	\ln\Big(\frac{p}{p_0}\Big) = -\frac{\gamma M_m}{\rho_l RT}\frac{2 \cos\theta}{r}
\end{equation}

To apply the Kelvin equation to different types of pore systems, the
generalized form presented in \autoref{pyg:eqn:kelvin-general}
is used. It replaces the meniscus radius by a mean radius of curvature
\(r_m\).

\begin{equation}\label{pyg:eqn:kelvin-general}
	\ln\Big(\frac{p}{p_0}\Big) = -\frac{\gamma M_m}{\rho_l RT}\frac{2}{r_m}
\end{equation}

The mean radius of curvature is defined through the two principal
radii of the curved interface.

\begin{equation}\label{pyg:eqn:kelvin-mradius}
	\frac{1}{r_m} = \frac{1}{2}\Big(\frac{1}{r_1}+\frac{1}{r_2}\Big)
\end{equation}

The relationship of the kelvin radius \(r_m\) to the actual pore
radius is more subtle, as it depends on pore geometry and the
filling state of the pore~\cite{doAdsorptionAnalysisEquilibria1998}.
If considering a cylindrical pore open at both ends, the radius reduces
to the original Kelvin equation during the desorption phase, and takes
other values with different combinations of parameters
as can be seen in \autoref{pyg:tab:kelvin-meniscus}.

\begin{table}[htb]
	\centering
	\caption{Assumed relationship between pore geometry and meniscus geometry during adsorption and desorption}%
	\label{pyg:tab:kelvin-meniscus}
	\begin{tabular}{ccc}
		\toprule
		              & \multicolumn{2}{c}{Pore filling}              \\
		\cmidrule{2-3}
		Pore geometry & Adsorption                       & Desorption \\
		\midrule
		slit          & cylindrical                      & concave    \\
		cylinder      & cylindrical                      & spherical  \\
		sphere        & spherical                        & spherical  \\
		\bottomrule
	\end{tabular}
\end{table}

According to Rouquerol~\cite{rouquerolAdsorptionPowdersPorous2013},
in adopting this approach, it is assumed that:

\begin{itemize}

	\item The Kelvin equation is applicable over the pore
	      range (mesopores). Therefore in pores which are below a
	      certain size (around \SI{2.5}{\nano\meter}), the granularity
	      of the liquid-vapour interface becomes too large for classical
	      bulk methods to be applied.
	\item The meniscus curvature is controlled by pore size and
	      shape. Ideal shapes for the curvature are assumed.
	\item Pores are rigid and of well defined shape. Their geometry is
	      considered to be invariant across the entire adsorbate.
	\item The filling/emptying of each pore does not depend on its location.
	\item Adsorption on the pore walls is not different from
	      surface adsorption.

\end{itemize}

\paragraph{The Barrett, Joyner and Halenda (BJH) method}

The BJH method for calculating pore size distribution
is based on a classical description of the adsorbate behaviour
in the adsorbent pores~\cite{barrettDeterminationPoreVolume1951}.
Under this method, the guest is adsorbing on the pore walls
following an ideal model, and decreasing the apparent pore volume until
condensation takes place, filling the entire pore. The critical radius
is a sum of two radii, the adsorbed layer thickness, which can be
modelled by a thickness model (such as Halsey, Harkins \& Jura or similar
as presented in \autoref{pyg:charac:tcurve})
and a critical radius model for condensation/evaporation,
based on a form of the Kelvin equation.

\begin{equation}\label{pyg:eqn:condensradius}
	r_p = t + r_k
\end{equation}

The original model uses the desorption curve as a basis for calculating
pore size distribution. Between two points of the curve, the volume
desorbed can be described as the volume contribution
from pore evaporation and the volume from layer thickness decrease as
per \autoref{pyg:eqn:condensradius}. The computation is done
cumulatively, starting from the filled pores and calculating the volume
adsorbed in a pore for each point using the following equation:

\begin{align}
	V_p & = \Big( \frac{\bar{r}_p}{\bar{r}_k + \Delta t_n} \Big)^2
	\Big(\Delta V_n - \Delta t_n \sum_{i=1}^{n-1} \Delta A_p
	+ \Delta t_n \bar{t}_n \sum_{i=1}^{n-1} \frac{\Delta A_p}{\bar{r}_p}\Big)\intertext{where}
	A   & = 2 \Delta V_p / r_p
\end{align}

In the BJH equation:

\begin{itemize}

	\item \(\Delta A_p\) is the area of the pores
	\item \(\Delta V_p\) is the adsorbed volume change between two points
	\item \(\bar{r}_p\) is the average pore radius calculated as a sum of the
	      kelvin radius and layer thickness of the pores at pressure p between two
	      measurement points
	\item \(\bar{r}_k\) is the average kelvin radius between two
	      measurement points
	\item \(\bar{t}_n\) is the average layer thickness
	      between two measurement points
	\item \(\Delta t_n\) is the average change in layer thickness
	      between two measurement points

\end{itemize}

Then, by plotting \(\Delta V / (2*\Delta r_p)\) versus the width
of the pores calculated for each point, the pore size distribution
can be obtained.

\paragraph{The Dollimore-Heal (DH) method}

The DH or Dollimore-Heal
method~\cite{dollimorePoresizeDistributionTypical1970}
is an extension of the BJH method which takes into account the
geometry of the pores by introducing a length component.
Like the BJH method, it is based on a classical description of
the adsorbate behaviour in the adsorbent pores and uses the 
same assumptions. The modified equation becomes:

\begin{align}
	V_p & = \Big(\frac{\bar{r}_p}{\bar{r}_k + \Delta t_n}\Big)^2
	\Big(\Delta V_n - \Delta t_n \sum_{i=1}^{n-1} \Delta A_p
	+ 2 \pi \Delta t_n \bar{t}_n \sum_{i=1}^{n-1} L_p\Big)       \\
	%
	A   & = 2 \Delta V_p / r_p                                   \\
	%
	L   & = \Delta A_p / 2 \pi r_p
\end{align}

The meaning of each symbol is the same as in the BJH equation.
As before, a plot of \(\Delta V/(2*\Delta r_p)\) versus the width of
the pores calculated for each point yields the pore size distribution.
The Dollimore-Heal method is used on the desorption branch,
of the isotherm but often also applied on the adsorption branch.

\subsubsection{Microporous size distribution}

When it comes to micropores (width \SI{< 2}{\nano\metre}), classical
fluid methods stop being viable. Adsorption of molecules in these pores
of comparable scale is highly dependent on the surface properties and
on guest-host interaction and leads to adsorbate phase densities
which are often very different than those in the bulk liquid state.
In order to model adsorption in such pores, a good description of
both solid-fluid and fluid-fluid potential functions is required.

\paragraph{The Horvath-Kawazoe (HK) method}

The H-K method attempts to describe the adsorption within pores
by calculation of the average potential energy for a
pore~\cite{horvathMethodCalculationEffective1983}.
The method starts by assuming the relationship between the gas
phase as being:

\begin{equation}
	R_g T \ln\Big(\frac{p}{p_0}\Big) = U_0 + P_a
\end{equation}

Here \(U_0\) is the potential function describing the surface to adsorbent
interactions and \(P_a\) is the potential function describing
guest-guest interactions. This equation is derived from the
equation of the free energy of adsorption at constant temperature where
the term \(T \Delta S^{tr}(w/w_{\infty})\) is assumed to be negligible.

If it is assumed that a Lennard-Jones-type potential function can 
accurately describe the interactions between adsorbate and surface
molecules, then the two contributions to the total potential can be
replaced by the extended function in \autoref{pyg:eqn:hk}.

\begin{multline}\label{pyg:eqn:hk}
	RTln(p/p_0) =   N_A\frac{n_a A_a + n_A A_A}{2 \sigma^{4}(l-d)} \\
	\times \int_{d/_2}^{1-d/_2}
	\Big[
		- \Big(\frac{\sigma}{r}\Big)^{4}
		+ \Big(\frac{\sigma}{r}\Big)^{10}
		- \Big(\frac{\sigma}{l-r}\Big)^{4}
		+ \Big(\frac{\sigma}{l-r}\Big)^{4}
		\Big] \mathrm{d}x
\end{multline}

Here \(l\) is the width of the pore, \(d\) defined as \(d=d_a+d_A\) is
the sum of the diameters of the adsorbate and adsorbent molecules,
\(n_a\) is number of molecules of adsorbent
and \(A_a\) and \(A_A\) the Lennard-Jones potential constant of the
fluid molecule and solid molecule respectively. They are defined as

\begin{align}
	A_a & = \frac{6mc^2\alpha_a\alpha_A}{\alpha_a/\varkappa_a + \alpha_A/\varkappa_A}
	%
	\intertext{and}
	%
	A_a & = \frac{3mc^2\alpha_A\varkappa_A}{2}
\end{align}

Where \(m\) is the mass of an electron, \(\alpha_a\) and \(\alpha_A\) are
the polarizability of the adsorbate and adsorbate molecule
and \(\varkappa_a\) and \(\varkappa_A\) the magnetic susceptibility of
the adsorbate molecule and adsorbent molecule, respectively.

The HK method is applicable to slit pores, and it can be extended
through modification to cylindrical and spherical pores. It is worth noting
that there are several assumptions which limit its applicability.

\begin{itemize}

	\item The HK method is reliant on knowledge of the properties of
	      the surface atoms. This assumption is true only if the
	      material surface is homogenous. Furthermore,
	      longer range interactions with multiple surface layers are
	      not considered.
	\item Each pore is modelled as uniform and of infinite length.
	      Materials with varying pore shapes or highly interconnected
	      networks may not give realistic results.
	\item Only dispersive forces are accounted for.
	      If the adsorbate-adsorbent interactions
		  have other specific contributions, as is the case 
		  for dipole-dipole or \(pi\)-backbonding interactions,
		  the Lennard-Jones potential function will not be
	      an accurate description.
	\item The model does not have a description of capillary condensation.
	      This means that the pore size distribution can only
		  be considered accurate up to a maximum of 
		  \SIrange{3}{5}{\nano\metre}.

\end{itemize}

\subsubsection{Multiscale computational methods}

DFT theory emerged as a rigorous description of molecular
adsorption in pores~\cite{seatonNewAnalysisMethod1989}. 
It calculates the properties of the fluid
directly from the forces acting between constituent molecules
through a statistical mechanical approach. Latter developments,
like non-local DFT (NLDFT)~\cite{tarazonaPhaseEquilibriaFluid1987}, 
which makes an account for short range molecule correlation
and therefore for the changes in the density profile
around the pore walls, and quenched solid state DFT 
(QSDFT)~\cite{neimarkQuenchedSolidDensity2009}, which allows for 
heterogeneity of pore walls to be incorporated in the model,
have improved the accuracy of the method.

The density functional theory approach can therefore simulate 
adsorption isotherms on pores of different geometries and sizes. 
By defining a pore geometry and running the simulation with a 
range of pore radii, a collection of isotherms is obtained. 
If an experimental isotherm is thought of as a sum of 
adsorption isotherms in different size pores, then it stands 
to reason that the preponderence of those pores can be calculated
through deconvolution.

Since the DFT method can model adsorbate condensation behaviour,
as well as micropore filling and multilayer adsorption, it can 
be used for multiscale pore size distribution. The downside is
that DFT kernels are temperature, probe, pore-geometry and
adsorbent specific and as such are not universally applicable.

% !TEX root = ../../main.tex

\subsection{Predicting multicomponent adsorption}\label{pyg:iast}

Until now, we have only referred to isotherms pertaining to the adsorption
of a single adsorbate. However, besides gas storage, most if not all
industrial applications of adsorbents involve multiple chemical
species undergoing competitive adsorption. Experiments involving
several adsorbents are generally difficult and time consuming.
There is therefore a need to predict such multicomponent systems
in order to rapidly screen for potentially interesting separations
starting from pure component data.

To this end, several methods have been devised, with perhaps
the most common approach the assumption of the adsorbed phase as
an analogue to a fluid mixture, in a similar manner to the
vacancy solution theory presented in \autoref{pyg:models:vst}.
This method, also known as \gls{IAST}
will be presented here. Other multicomponent theories
such as \gls{RAST} or the Nitta model
exist, but usually require more information such as specific
surface binary activity coefficients, and are therefore less suited to
a general approach. Furthermore, the \gls{IAST} method has been
proven~\cite{cessfordEvaluationIdealAdsorbed2012,%
	vanheestIdentificationMetalOrganic2012} to be successful in the
prediction of multicomponent adsorption even when considering
adsorption in microporous materials such as \glspl{MOF}.

Originally derived by \citet{myersThermodynamicsMixedgasAdsorption1965},
the \gls{IAST} model is based on three main assumptions. The first is that the
same surface area is available for all components. Then it is
assumed that the mixture behaves like an ideal solution at constant
spreading pressure and temperature. This means the spreading pressure \(\pi\) of
each component is the same as the surface potential of the mixture
and therefore the mean strength of interaction is equal between all
molecules of solution. Finally the surface adsorbed phase is assumed
to be at equilibrium with the gas phase. \autoref{pyg:eqn:iast}
can then be written, where \(p_{i,g}\) is the partial pressure of
the component in the gas phase, \( p_i^0(\pi)\) the pressure of
that component which will give the same spreading pressure on the
surface and \(x_i\) is its fraction in the adsorbed phase.
%
\begin{align}
	%
	p_{i,g}             & = p_i^0(\pi)x_i                              %
	\label{pyg:eqn:iast}                                               \\
	%
	\frac{\pi A}{R_g T} & = \int_{0}^{P_{i}^{0}} n_{ads,i} d\ln{(p_i)} %
	\label{pyg:eqn:iast-2}                                             \\
	%
	\sum_{1=1}^{N} x_i  & = 1                                          %
	\label{pyg:eqn:iast-3}
\end{align}

In order to account for non-ideality of the gas phase
fugacity can be used instead of pressure,
with a suitable equation of state required to relate it to bulk pressure.
The pressure and spreading pressure are then calculated through
the integrated Gibbs isotherm (\autoref{pyg:eqn:iast-2}). As the sum of all
fractions in the adsorbed phase is equal to unity (\autoref{pyg:eqn:iast-3}),
the equations can be solved for a given pressure and gas composition to obtain
the fractions of each component in the adsorbed phase at the hypothetical
pressure \(P_{i}^{0}\). The total amount adsorbed can then be calculated.
%
\begin{equation}
	\frac{1}{n_{ads,t}} = \sum_{1=1}^{N} \frac{x_i}{n_{ads,i}^0}
\end{equation}
%
Finally, the number of moles adsorbed for each component is given
by multiplying the total amount adsorbed by the fraction of each
component.
%
\begin{equation}
	n_{ads, i} = x_i n_{ads,t}
\end{equation}

In order to calculate the spreading pressure, an adsorption model
such as the ones presented in \autoref{pyg:models} is required.
By substituting the amount adsorbed (\gls{nads}) in \autoref{pyg:eqn:iast-2}
and calculation of the resulting integral, an expression for
spreading pressure can be determined. The integral can only be
evaluated analytically in the case of a few models, for example in
the previously mentioned Henry, Langmuir and multi-site Langmuir,
\gls{BET}, Quadratic and Temkin models. In other cases, a numerical approach
must be used for its calculation. An alternative to using a
model is to use an interpolation method between recorded experimental
points to approximate the amount adsorbed at a pressure \gls{p}. Numerical
quadrature is then used to evaluate the integral and obtain the
spreading pressure.