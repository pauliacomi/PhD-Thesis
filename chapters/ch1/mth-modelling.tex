\subsection{Mathemathical descriptions of isotherms}

A lot of effort was put into attempting to describe 
the phenomenon of adsorption using a simple model.
Through a mathematical understanding, the underlying 
mechanisms of adsorption can be understood and 
compared.
Unfortunately, the plethora of isotherm 
features and shapes can only be truly recreated
by molecular simulation, which requires an exact
knowledge of the adsorbent structure and its interaction
with the adsorbed gas.
Nevertheless, the available models can be useful
when fitted to measured isotherms
for obtaining simple parameters which are 
representative of physical factors such as 
guest-host interaction, surface area, pore size,
total pore volume and others.
More importantly, it allows us to numerically 
compare measured isotherms. 


\subsubsection{The Henry model}

The simplest method of describing adsorption on a 
surface is Henry’s law. It assumes only interactions
with the adsorbate surface and is described by a 
linear dependence of adsorbed amount with
increasing pressure.

\begin{equation}\label{eqn:pyg:henry}
    n_a(p) = K_H p
\end{equation}

Physically, Henry's law is unrealistic as adsorption sites
will saturate at higher pressures. However, the constant \(K_H\),
or Henry’s constant, can be thought of as a measure of the strength
of the interaction of the probe gas with the surface. At very 
low concentrations of gas there is a 
theoretical requirement for the applicability of Henry's law.
Therefore, most models reduce to equation~\ref{eqn:pyg:henry}
as \(\lim_{p \to 0} n(p)\).

\subsubsection{Langmuir and multi-site langmuir model}

The Langmuir theory~\cite{langmuirAdsorptionGasesPlane1918a}, 
proposed at the start of the 20th century, states that
adsorption takes place on specific sites on a surface, in a single layer. 
It is derived based on several assumptions:

\begin{itemize}
    
    \item All sites are equivalent and have the same chance of being occupied.
    \item Each adsorbate molecule can occupy one adsorption site.
    \item There are no interactions between adsorbed molecules.
    \item The rates of adsorption and desorption are proportional to the number
    of sites currently free and currently occupied, respectively.
    \item Adsorption is complete when all sites are filled.
    
\end{itemize}

Using these assumptions we can define rates for both adsorption and
desorption. The adsorption rate (equation~\ref{eqn:pyg:langmuir_ads}) 
will be proportional to the number of sites available on the surface, 
as well as the number of molecules in the gas, which is given by pressure.
The desorption rate, on the other hand, will be proportional to the 
number of occupied sites and the energy of adsorption (equation~\ref{eqn:pyg:langmuir_des}).
It is also useful to define \(\theta = n_a/n_a^m\) as the surface coverage,
the number of sites occupied divided by the total sites. At equilibrium, 
the rate of adsorption and the rate of
desorption are equal, therefore the two equations can be combined.
The equation can then be arranged to obtain an expression for the 
loading called the Langmuir equation (\ref{eqn:pyg:langmuir}).

\begin{gather}
    v_a = k_a p (1 - \theta) \label{eqn:pyg:langmuir_ads} \\
    v_d = k_d \theta \exp{(-\frac{E}{RT})} \label{eqn:pyg:langmuir_des} \\
    k_a p (1 - \theta) = k_d \theta \exp{(-\frac{E}{RT})} \\
    n_a(p) = n_a^m \frac{Kp}{1+Kp} \label{eqn:pyg:langmuir}
\end{gather}

The Langmuir constant \(K\)is the product of the individual desorption 
and adsorption constants \(k_a\) and \(k_d\) and exponentially 
related to the energy of adsorption \(\exp{(-{E}/{RT})}\).

A common extension to the Langmuir model is to consider 
the experimental isotherm to be the sum of several Langmuir-type
isotherms, each with specific maximum coverage and affinities.
The underlying assumption is that the adsorbent has several distinct
types of homogeneous adsorption sites and applying a Langmuir
equation to each. This is particularly
applicable in cases where the structure of the adsorbent 
suggests that different types of sites are present, 
such as in crystalline materials of variable chemistry like 
zeolites and MOFs. The resulting isotherm equation is:

\begin{equation}\label{eqn:pyg:langmulti}
    n_a(p) = \sum_i n_{a,i}^m\frac{K_i p}{1+K_i p}
\end{equation}

In practice, only up to three adsorption sites are usually considered.


\subsubsection{BET model}

The BET model~\cite{brunauerAdsorptionGasesMultimolecular1938}
assumes that adsorption happens on the surface of the material in
incremental layers according to several assumptions:

\begin{itemize}
    \item The adsorption sites are equivalent, and therefore the surface is heterogeneous.
    \item There are no lateral interactions between adsorbed molecules.
    \item The adsorption takes place in layers, with adsorbed molecules 
    acting as sites for the next layer.
    \item The adsorption energy of a molecule on the second and higher layers
    is the same and equals the condensation energy of the adsorbent \(E_L\).
\end{itemize}

A particular surface percentage \(\theta_x\) is occupied with x layers.
For each layer at equilibrium, the adsorption and desorption rates must be equal. 
We can then apply the Langmuir theory for each layer.

\begin{align}
    k_{a_1} p \theta_0 &= k_{d_1} \theta_1 \exp{(-\frac{E_1}{RT})} \\
    k_{a_2} p \theta_1 &= k_{d_2} \theta_2 \exp{(-\frac{E_L}{RT})} \\
    \vdots \nonumber \\
    k_{a_i} p \theta_{i-1} &= k_{d_i} \theta_i \exp{(-\frac{E_L}{RT})}
\end{align}

Since we are assuming that all layers beside the first have the same properties,
we can define \(g= {k_{d_2}}{k_{a_2}} = {k_{d_3}}{k_{a_3}} = \cdots\).
The coverage for each layer \(\theta\) can now be expressed in terms of \(\theta_0\).

\begin{align}
    \theta_1 &= y \theta_0 \quad where \quad y = \frac{k_{a_1}}{k_{d_1}} p \exp{(-\frac{E_1}{RT})} \\
    \theta_2 &= x \theta_1 \quad where \quad x = \frac{p}{g} \exp{(-\frac{E_L}{RT})} \\
    \theta_3 &= x \theta_2 = x^2 \theta_1 \\
    \vdots \nonumber \\
    \theta_i &= x^{i-1} \theta_1 = y x^{i-1} \theta_0
\end{align}

A constant C may be defined such that

\begin{equation}
    C = \frac{y}{x} = \frac{k_{a_1}}{k_{d_1}} g \exp{(\frac{E_1 - E_L}{RT})}
    \theta_i = C x^i \theta_0
\end{equation}

For all the layers, the equations can be summed:

\begin{equation}
    \frac{n}{n_m} = \sum_{i=1}^{\infty} i \theta^i = C \sum_{i=1}^{\infty} i x^i \theta_0
\end{equation}

And since

\begin{equation}
    \theta_0 = 1 - \sum_{1}^{\infty} \theta_i
    \sum_{i=1}^{\infty} i x^i = \frac{x}{(1-x)^2}
\end{equation}

Then we obtain the BET equation

\begin{equation}\label{eqn:pyg:bet}
    n_a(p) = n_a^m \frac{K_a p}{(1-K_b p)(1-K_b p+ K_a p)}
\end{equation}

The equation reduces to the Langmuir equation (\ref{eqn:pyg:langmuir}) when

\subsubsection{Toth model}

The Toth model is an empirical modification to the Langmuir equation
(\ref{eqn:pyg:langmuir})
which introduces a power parameter for the denominator leading to
the following formula:

\begin{equation}\label{eqn:pyg:toth}
    n_a(p) = n_a^m \frac{K p}{[1 + (K p)^t]^(1/t)}
\end{equation}

The parameter \(t\) is a measure of the system heterogeneity. Thanks to this
additional parameter, the Toth equation can accurately describe a
large number of adsorbent/adsorbate systems and is recommended as the first
choice of isotherm equation for fitting isotherms of many adsorbents such as
hydrocarbons, carbon oxides, hydrogen sulphide and alcohols on activated carbons
but also zeolites.
It is worth noting that the equation no longer reduces to the Henry law 
at low loading and therefore is theoretically inconsistent.


\subsubsection{Temkin model}

The Temkin adsorption isotherm~\cite{temkinKineticsAmmoniaSynthesis1940}, 
like the Langmuir model, considers
a surface with \(n_a^m\) identical adsorption sites, but takes into account adsorbate-
adsorbate interactions by assuming that the heat of adsorption is a linear
function of the coverage. The Temkin isotherm is derived using a
mean-field argument and used an asymptotic approximation
to obtain an explicit equation for the 
loading~\cite{simonOptimizingNanoporousMaterials2014}.

\begin{equation}\label{eqn:pyg:temkin}
    n_a(p) = n_a^m \frac{Kp}{1+Kp} + n_a^m \theta (\frac{Kp}{1+Kp})^2 (\frac{Kp}{1+Kp} -1)
\end{equation}

Here, \(n_a^m\) and K have the same physical meaning as in the Langmuir model.
The additional parameter \(\theta\) describes the strength of the adsorbate-adsorbate
interactions (\(\theta < 0\) for attractions).

\subsubsection{Jensen-Seaton model}

When modelling adsorption in micropores, a requirement was highlighted by
Jensen and Seaton in 1996~\cite{jensenIsothermEquationAdsorption1996} 
that at sufficiently high pressures the adsorption
isotherm should not reach a horizontal plateau corresponding to saturation but
that this asymptote should continue to rise due to the compression 
of the adsorbate in the pores. They developed a semi-empirical equation
to describe this phenomenon based on a function that interpolates between
two asymptotes: the Henry’s law asymptote at low pressure and an
asymptote reflecting the compressibility of the adsorbate at
high pressure.

\begin{equation}\label{eqn:pyg:jseaton}
    n(p) = K_H p (1 + \frac{K_H p}{(a (1+(b p)))^c})^{(-1/c)}
\end{equation}

Here \(K_H\) is the Henry constant, \(b\) is the compressibility of the
adsorbed phase and \(c\) an empirical constant.

The equation can be used to model both absolute and excess adsorption as the pore
volume can be incorporated into the definition of \(b\), although this can lead
to negative adsorption slopes for the compressibility asymptote.
This equation has been found to provide a better fit for experimental data
from microporous solids than the Langmuir or Toth equation, in particular for
adsorbent/adsorbate systems with high Henry’s constants where the amount adsorbed
increases rapidly at relatively low pressures and then slows down dramatically.

\subsubsection{Quadratic model}

The quadratic adsorption isotherm~\cite{hillIntroductionStatisticalThermodynamics1986} 
exhibits an inflection point. The loading is convex at low 
pressures but changes concavity as it saturates, yielding
an S-shape. The S-shape can be explained by adsorbate-adsorbate attractive
forces; the initial convexity is due to a cooperative
effect of adsorbate-adsorbate attractions aiding in the recruitment of
additional adsorbate molecules.

\begin{equation}\label{eqn:pyg:quad}
    n(p) = n_a^m \frac{(K_a + 2 K_b p)p}{1+K_{ap} + K_{bp}^2}
\end{equation}

The parameter \(K_a\) can be interpreted as the Langmuir constant; the
strength of the adsorbate-adsorbate attractive forces is embedded in \(K_b\).


\subsubsection{Virial model}

A virial isotherm model attempts to fit the measured data to a factorized
exponent relationship between loading and 
pressure~\cite{myersThermodynamicsAdsorptionPorous2002}.

\begin{equation}\label{eqn:pyg:virial}
    p = n \exp{(K_1n^0 + K_2n^1 + K_3n^2 + K_4n^3 + \cdots + K_i n^{i-1})}
\end{equation}

It has been applied with success to describe the behaviour of standard as
well as supercritical isotherms. The factors are usually empirical,
although some relationship with physical can be determined:
the first constant is related to the Henry constant at zero loading, while
the second constant is a measure of the interaction strength with the surface.

\begin{equation}
    K_1 = -\ln{K_{H,0}}
\end{equation}

In practice, besides the first constant, only 2-3 factors are used.


\subsubsection{Vacancy solution theory models}


As a part of the Vacancy Solution Theory (VST) family of models, it is based on concept
of a “vacancy” species, denoted v, and assumes that the system consists of a
mixture of these vacancies and the adsorbate.

The VST model is defined as follows:

\begin{itemize}
    
    \item A vacancy is an imaginary entity defined as a vacuum space
    which acts as the solvent in both the gas and adsorbed phases.
    \item The properties of the adsorbed phase are defined as excess properties
    in relation to a dividing surface.
    \item The entire system including the adsorbent are in thermal equilibrium
    however only the gas and adsorbed phases are in thermodynamic equilibrium.
    \item The equilibrium of the system is maintained by the spreading pressure
    which arises from a potential field at the surface

\end{itemize}
    
It is possible to derive expressions for the vacancy chemical potential in both
the adsorbed phase and the gas phase, which when equated give the following equation
of state for the adsorbed phase:

\begin{equation}
    \pi = - \frac{R_g T}{\sigma_v} \ln{y_v x_v}
\end{equation}

where \(y_v\) is the activity coefficient and  \(x_v\) is the mole fraction of
the vacancy in the adsorbed phase.
This can then be introduced into the Gibbs equation to give a general isotherm equation
for the Vacancy Solution Theory where \(K_H\) is the Henry’s constant and
\(f(\theta)\) is a function that describes the non-ideality of the system based
on activity coefficients:

\begin{equation}
    p = \frac{n_{ads}}{K_H} \frac{\theta}{1-\theta} f(\theta)
\end{equation}

The general VST equation requires an expression for the activity coefficients.
The Wilson~\cite{suwanayuenGasAdsorptionIsotherm1980} equation can be used, 
which expresses the activity coefficient in terms
of the mole fractions of the two species (adsorbate and vacancy) and two constants
\(\Lambda_{1v}\) and \(\Lambda_{1v}\). The equation becomes:

\begin{equation}\label{eqn:pyg:wvst}
    p = \bigg( \frac{n_{ads}}{K_H} \frac{\theta}{1-\theta} \bigg)
    \bigg( \Lambda_{1v} \frac{1-(1-\Lambda_{v1})\theta}{\Lambda_{1v}+(1-\Lambda_{1v})\theta} \bigg)
    \exp{\bigg( -\frac{\Lambda_{v1}(1-\Lambda_{v1})\theta}{1-(1-\Lambda_{v1})\theta}
    -\frac{(1 - \Lambda_{1v})\theta}{\Lambda_{1v} + (1-\Lambda_{1v}\theta)} \bigg)}
\end{equation}

Cochran~\cite{cochranVacancySolutionTheory1985} 
developed a simpler, three parameter equation based on
the Flory–Huggins equation for the activity coefficient.
The equation for then becomes:

\begin{equation}\label{eqn:pyg:fhvst}
    p = \bigg( \frac{n_{ads}}{K_H} \frac{\theta}{1-\theta} \bigg)
        \exp{\frac{\alpha^2_{1v}\theta}{1+\alpha_{1v}\theta}} 
        \quad where \quad
    \alpha_{1v} = \frac{\alpha_{1}}{\alpha_{v}} - 1
\end{equation}

Here \(\alpha_{1}\) and \(\alpha_{v}\) are the molar areas of the adsorbate
and the vacancy respectively.