\subsection{Isotherm models}


\subsubsection{The Henry model}

Henry's law. Assumes a linear dependence of adsorbed amount with
pressure.

Usually, Henry's law is unrealistic because the adsorption sites
will saturate at higher pressures.
Only use if your data is linear.

\subsubsection{BET model}

The BET model assumes that adsorption happens on the surface of the material in
incremental layers according to several assumptions:

\begin{itemize}
    \item The adsorption sites are equivalent, and therefore the surface is heterogeneous
    \item There are no lateral interactions between adsorbed molecules
    \item The adsorption happens in layers, with adsorbed molecules acting as adsorption
    sites for new molecules
    \item The adsorption energy of a molecule on the second and higher layers equals
    the condensation energy of the adsorbent \(E_L\).
\end{itemize}

A particular surface percentage \(\theta_x\) is occupied with x layers.
For each layer at equilibrium, the adsorption and desorption rates must be equal. We can
then apply the Langmuir theory for each layer.

\begin{align}
    k_{a_1} p \theta_0 &= k_{d_1} \theta_1 \exp{(-\frac{E_1}{RT})} \\
    k_{a_2} p \theta_1 &= k_{d_2} \theta_2 \exp{(-\frac{E_L}{RT})} \\
    \cdots \\
    k_{a_i} p \theta_{i-1} &= k_{d_i} \theta_i \exp{-\frac{E_L}{RT}} \\
\end{align}

Since we are assuming that all layers beside the first have the same properties,
we can define \(g= \frac{k_{d_2}}{k_{a_2}} = \frac{k_{d_3}}{k_{a_3}} = \cdots\).
The coverages \(\theta\) can now be expressed in terms of \(\theta_0\).

\begin{align}
    \theta_1 &= y \theta_0 \quad where \quad y = \frac{k_{a_1}}{k_{d_1}} p \exp{(-\frac{E_1}{RT})} \\
    \theta_2 &= x \theta_1 \quad where \quad x = \frac{p}{g} \exp{(-\frac{E_L}{RT})} \\
    \theta_3 &= x \theta_2 = x^2 \theta_1 \\
    \cdots \\
    \theta_i &= x^{i-1} \theta_1 = y x^{i-1} \theta_0 \\
\end{align}

A constant C may be defined such that

\begin{equation}
    C = \frac{y}{x} = \frac{k_{a_1}}{k_{d_1}} g \exp{(\frac{E_1 - E_L}{RT})}
    \theta_i = C x^i \theta_0
\end{equation}

For all the layers, the equations can be summed:

\begin{equation}
    \frac{n}{n_m} = \sum_{i=1}^{\infty} i \theta^i = C \sum_{i=1}^{\infty} i x^i \theta_0
\end{equation}

Since

\begin{equation}
    \theta_0 = 1 - \sum_{1}^{\infty} \theta_i
    \sum_{i=1}^{\infty} i x^i = \frac{x}{(1-x)^2}
\end{equation}

Then we obtain the BET equation

\begin{equation}
    L(P) = \frac{n}{n_m} = M\frac{K_A P}{(1-K_B P)(1-K_B P+ K_A P)}
\end{equation}

“Adsorption of Gases in Multimolecular Layers”, Stephen Brunauer,
P. H. Emmett and Edward Teller, J. Amer. Chem. Soc., 60, 309(1938)

\subsubsection{Langmuir and multi-site langmuir model}

The Langmuir theory, proposed at the start of the 20th century, states that
adsorption happens on active sites on a surface in a single layer. It is
derived based on several assumptions.

\begin{itemize}
    
    \item All sites are equivalent and have the same chance of being occupied
    \item Each adsorbate molecule can occupy one adsorption site
    \item There are no interactions between adsorbed molecules
    \item The rates of adsorption and desorption are proportional to the number
    of sites currently free and currently occupied, respectively
    \item Adsorption is complete when all sites are filled.
    
\end{itemize}

Using the following assumptions we can define rates for both the adsorption and
desorption. The adsorption rate will be proportional to the number of sites available
on the surface, as well as the number of molecules in the gas, which is represented by
the pressure. The desorption rate, on the other hand, will be proportional to the
number of occupied sites and the energy of adsorption.
It is also useful to define \(\theta = \frac{n_a}{M}\) as the surface coverage,
the number of sites occupied divided by the total sites. Mathematically:

\begin{equation}
    v_a = k_a p (1 - \theta)
    v_d = k_d \theta \exp{(-\frac{E}{RT})}
\end{equation}

Here, \(M\) is the moles adsorbed at the completion of the monolayer, and therefore
the maximum possible loading. At equilibrium, the rate of adsorption and the rate of
desorption are equal, therefore the two equations can be combined.

\begin{equation}
    k_a p (1 - \theta) = k_d \theta \exp{(-\frac{E}{RT})}
\end{equation}

Rearranging to get an expression for the loading, the Langmuir equation becomes:

\begin{equation}
    L(P) = M\frac{KP}{1+KP}
\end{equation}

The Langmuir constant is the product of the individual desorption and adsorption constants
\(k_a\) and \(k_d\) and exponentially related to the energy of adsorption
\(\exp{(-\frac{E}{RT})}\).

An extension to the Langmuir model is to consider the experimental isotherm to be
the sum of several Langmuir-type isotherms with different monolayer capacities and affinities.
The assumption is that the adsorbent presents several distinct types of homogeneous adsorption
sites, and that separate Langmuir equations should be applied to each. This is particularly
applicable in cases where the structure of the adsorbent suggests that different types of
sites are present, such as in crystalline materials of variable chemistry like zeolites and MOFs.
The resulting isotherm equation is:

\begin{equation}
    L(P) = \sum_i M_i\frac{K_i P}{1+K_i P}
\end{equation}

In practice, up to three adsorption sites are considered.
This model is the dual-site model (\(i=2\))

\subsubsection{Jensen Seaton model}

When modelling adsorption in micropores, a requirement was highlighted by
Jensen and Seaton in 1996, that at sufficiently high pressures the adsorption
isotherm should not reach a horizontal plateau corresponding to saturation but
that this asymptote should continue to rise due to the compression of the adsorbate
in the pores. They came up with a semi-empirical equation to describe this phenomenon
based on a function that interpolates between two asymptotes: the Henry’s law asymptote
at low pressure and an asymptote reflecting the compressibility of the adsorbate at
high pressure.

Here \(K_H\) is the Henry constant, \(b\) is the compressibility of the
adsorbed phase and \(c\) an empirical constant.

The equation can be used to model both absolute and excess adsorption as the pore
volume can be incorporated into the definition of \(b\), although this can lead
to negative adsorption slopes for the compressibility asymptote.
This equation has been found to provide a better fit for experimental data
from microporous solids than the Langmuir or Toth equation, in particular for
adsorbent/adsorbate systems with high Henry’s constants where the amount adsorbed
increases rapidly at relatively low pressures and then slows down dramatically.

References
----------
Jensen, C. R. C.; Seaton, N. A., An Isotherm Equation for Adsorption to High
   Pressures in Microporous Adsorbents. Langmuir 1996, 12, (Copyright (C) 2012
   American Chemical Society (ACS). All Rights Reserved.), 2866-2867.

   
\subsubsection{Quadratic model}

The quadratic adsorption isotherm exhibits an inflection point; the loading
is convex at low pressures but changes concavity as it saturates, yielding
an S-shape. The S-shape can be explained by adsorbate-adsorbate attractive
forces; the initial convexity is due to a cooperative
effect of adsorbate-adsorbate attractions aiding in the recruitment of
additional adsorbate molecules.

The parameter \(K_a\) can be interpreted as the Langmuir constant; the
strength of the adsorbate-adsorbate attractive forces is embedded in \(K_b\).

References
----------
T. L. Hill, An introduction to statistical thermodynamics, Dover
   Publications, 1986.

\subsubsection{Temkin model}

The Temkin adsorption isotherm, like the Langmuir model, considers
a surface with M identical adsorption sites, but takes into account adsorbate-
adsorbate interactions by assuming that the heat of adsorption is a linear
function of the coverage. The Temkin isotherm is derived using a
mean-field argument and used an asymptotic approximation
to obtain an explicit equation for the loading.

Here, M and K have the same physical meaning as in the Langmuir model.
The additional parameter \(\theta\) describes the strength of the adsorbate-adsorbate
interactions (\(\theta < 0\) for attractions).

References
----------
V. P. M.I. Tempkin, Kinetics of ammonia synthesis on promoted iron
catalyst, Acta Phys. Chim. USSR 12 (1940) 327–356.
Phys. Chem. Chem. Phys., 2014,16, 5499-5513


\subsubsection{Temkin model}

The Toth model is an empirical modification to the Langmuir equation.
The parameter \(t\) is a measure of the system heterogeneity.

Thanks to this addition parameter, the Toth equation can accurately describe a
large number of adsorbent/adsorbate systems and is recommended as the first
choice of isotherm equation for fitting isotherms of many adsorbents such as
hydrocarbons, carbon oxides, hydrogen sulphide and alcohols on activated carbons
but also zeolites.


\subsubsection{Virial model}

A virial isotherm model attempts to fit the measured data to a factorized
    exponent relationship between loading and pressure.

\begin{equation}
    P = n \exp{(K_1n^0 + K_2n^1 + K_3n^2 + K_4n^3 + \cdots + K_i n^{i-1})}
\end{equation}

It has been applied with success to describe the behaviour of standard as
well as supercritical isotherms. The factors are usually empirical,
although some relationship with physical can be determined:
the first constant is related to the Henry constant at zero loading, while
the second constant is a measure of the interaction strength with the surface.

\begin{equation}
    K_1 = -\ln{K_{H,0}}
\end{equation}

In practice, besides the first constant, only 2-3 factors are used.


\subsubsection{Vacancy solution theory models}


As a part of the Vacancy Solution Theory (VST) family of models, it is based on concept
of a “vacancy” species, denoted v, and assumes that the system consists of a
mixture of these vacancies and the adsorbate.

The VST model is defined as follows:

\begin{itemize}
    
    \item A vacancy is an imaginary entity defined as a vacuum space
    which acts as the solvent in both the gas and adsorbed phases.
    \item The properties of the adsorbed phase are defined as excess properties
    in relation to a dividing surface.
    \item The entire system including the adsorbent are in thermal equilibrium
    however only the gas and adsorbed phases are in thermodynamic equilibrium.
    \item The equilibrium of the system is maintained by the spreading pressure
    which arises from a potential field at the surface

\end{itemize}
    
It is possible to derive expressions for the vacancy chemical potential in both
the adsorbed phase and the gas phase, which when equated give the following equation
of state for the adsorbed phase:

\begin{equation}
    \pi = - \frac{R_g T}{\sigma_v} \ln{y_v x_v}
\end{equation}

where \(y_v\) is the activity coefficient and  \(x_v\) is the mole fraction of
the vacancy in the adsorbed phase.
This can then be introduced into the Gibbs equation to give a general isotherm equation
for the Vacancy Solution Theory where \(K_H\) is the Henry’s constant and
\(f(\theta)\) is a function that describes the non-ideality of the system based
on activity coefficients:

\begin{equation}
    P = \frac{n_{ads}}{K_H} \frac{\theta}{1-\theta} f(\theta)
\end{equation}

The general VST equation requires an expression for the activity coefficients.
The Wilson equation can be used, which expresses the activity coefficient in terms
of the mole fractions of the two species (adsorbate and vacancy) and two constants
\(\Lambda_{1v}\) and \(\Lambda_{1v}\). The equation becomes:

\begin{equation}
P = \bigg( \frac{n_{ads}}{K_H} \frac{\theta}{1-\theta} \bigg)
    \bigg( \Lambda_{1v} \frac{1-(1-\Lambda_{v1})\theta}{\Lambda_{1v}+(1-\Lambda_{1v})\theta} \bigg)
    \exp{\bigg( -\frac{\Lambda_{v1}(1-\Lambda_{v1})\theta}{1-(1-\Lambda_{v1})\theta}
    -\frac{(1 - \Lambda_{1v})\theta}{\Lambda_{1v} + (1-\Lambda_{1v}\theta)} \bigg)}
\end{equation}

Cochran developed a simpler, three parameter equation based on
the Flory–Huggins equation for the activity coefficient.
The equation for then becomes:

\begin{align}
    P &= \bigg( \frac{n_{ads}}{K_H} \frac{\theta}{1-\theta} \bigg)
        \exp{\frac{\alpha^2_{1v}\theta}{1+\alpha_{1v}\theta}}
    \alpha_{1v} &= \frac{\alpha_{1}}{\alpha_{v}} - 1
\end{align}

where \(\alpha_{1}\) and \(\alpha_{v}\) are the molar areas of the adsorbate
and the vacancy respectively.

References
----------
Cochran, T. W.; Kabel, R. L.; Danner, R. P., Vacancy solution theory of
adsorption using Flory-Huggins activity coefficient equations. AIChE J. 1985, 31, 268-77.

Suwanayuen, S.; Danner, R. P., Gas-Adsorption Isotherm Equation Based On
Vacancy Solution Theory. AIChE Journal 1980, 26, (1), 68-76.
