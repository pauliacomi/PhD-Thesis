
\section{Methods implemented in pyGAPS}

\subsection{BET surface area}

The BET surface area is one of the first standardised methods to calculate the
surface area of a porous material. It is generally applied on isotherms obtained
through N2 adsorption at 77K, although other adsorbates (Ar, Kr) have been used.

It assumes that the adsorption happens on the surface of the material in
incremental layers according to the BET theory. Even if the adsorbent is porous,
the initial amount adsorbed (usually between 0.05 - 0.4 \(p/p_0\)) can be
modelled through the BET equation:

\begin{equation}
    \frac{p/p_0}{n_{ads} (1-p/p_0)} = \frac{1}{n_{m} C} + \frac{C - 1}{n_{m} C}(p/p_0)
\end{equation}


Therefore, if we plot the isotherm points as
\(\frac{p/p_0}{n_{ads}(1-p/p_0)}\) versus \(p/p_0\), a linear region
can usually be found. The slope and intercept of this line
can then be used to calculate \(n_{m}\), the amount adsorbed at the
statistical monolayer, as well as C, the BET constant.

\begin{equation}
    n_{m} = \frac{1}{s+i}
    C = \frac{s}{i} + 1
\end{equation}

The surface area can then be calculated by using the moles
adsorbed at the statistical monolayer. If the specific area taken
by one of the adsorbate molecules on the surface is known, it is
inserted in the following equation together with Avogadro's number:

\begin{equation}
    a(BET) = n_m A_N \sigma
\end{equation}


Limitations

While a standard for surface area determinations, the BET area
should be used with care, as there are many assumptions made in
the calculation. To augment the validity of the BET
method, Rouquerol~\cite{rouquerolAdsorptionPowdersPorous2013} proposed
several checks to ensure that the BET region selected is valid:

\begin{itemize}

	\item The BET constant (C) obtained should be positive
	\item In the corresponding Rouquerol plot where \(n_{ads}(1-p/p_0)\) is plotted with respect to \(p/p_0\), the points chosen for BET
	      analysis should be strictly increasing
	\item  The loading at the statistical monolayer should be
	      situated within the limits of the BET region

\end{itemize}


This module implements all these checks.

Regardless, the BET surface area should still be interpreted carefully.
The following assumptions are implicitly made in this approach:

\begin{itemize}

    \item Adsorption takes place on the pore surface. Microporous materials
            which have pores in similar size as the molecule adsorbed
            cannot have a realistic surface area
    \item The cross-sectional area of the molecule on the surface
            cannot be guaranteed. For example, nitrogen has been known to adopt
            a different conformation on the surface of some materials,
            due to inter-molecular forces, which effectively
            lowers its cross-sectional area.
    \item No account is made for adsorbate-adsorbent interaction in BET theory

\end{itemize}

References

“Adsorption of Gases in Multimolecular Layers”, Stephen Brunauer,
P. H. Emmett and Edward Teller, J. Amer. Chem. Soc., 60, 309(1938)


\subsection{Langmuir surface area}

The Langmuir theory, proposed at the start of the 20th century,
states that adsorption happens on active sites on a surface
in a single layer. It is derived based on several assumptions.

\begin{itemize}
    \item All sites are equivalent and have the same chance of being occupied
    \item Each adsorbate molecule can occupy one adsorption site
    \item There are no interactions between adsorbed molecules
    \item The rates of adsorption and desorption are proportional to the number
            of sites currently free and currently occupied, respectively
    \item Adsorption is complete when all sites are filled.
\end{itemize}

The Langmuir equation is then:

\begin{equation}
    n = n_m\frac{KP}{1+KP}
\end{equation}

The equation can be rearranged as:

\begin{equation}
	\frac{P}{n} = \frac{1}{K n_m} + \frac{P}{n_m}
\end{equation}

Assuming the data can be fitted with a Langmuir model, by plotting
\(\frac{P}{n}\) against pressure, a line will be obtained. The slope and
intercept of this line can then be used to calculate \(n_{m}\),
the amount adsorbed at the monolayer, as well as K, the Langmuir constant.

\begin{equation}
	n_m = \frac{1}{s}
	K = \frac{1}{i * n_m}
\end{equation}


The surface area can then be calculated by using the moles adsorbed at the
monolayer. If the specific area taken by one of the adsorbate molecules on
the surface is known, it is inserted in the following equation together with Avogadro's number:

\begin{equation}
	a(Langmuir) = n_m A_N \sigma
\end{equation}


Limitations

The Langmuir method for determining surface area assumes that only one single
layer is adsorbed on the surface of the material. As most adsorption processes
(except chemisorption) don't follow this behaviour, it is important to regard
the Langmuir surface area as an estimate.

References

I. Langmuir, J. American Chemical Society 38, 2219(1916); 40, 1368(1918)

\subsection{t-plot Method}

The t-plot method attempts to relate the adsorption on a material
with an ideal curve which describes the thickness of the adsorbed
layer on a surface. A plot is constructed, with the isotherm loading
data is plotted versus thickness values obtained through the model.
It stands to reason that, in the case that the experimental adsorption
curve follows the model, a straight line will be obtained with its
intercept through the origin. However, since in most cases there
are differences between adsorption in the pores and ideal surface
adsorption, the t-plot will deviate and form features which can
be analysed to describe the material characteristics.

\begin{itemize}

	\item a sharp vertical deviation will indicate condensation
	      in a type of pore
	\item a gradual slope will indicate adsorption on the
	      wall of a particular pore

\end{itemize}

The slope of the linear section can be used to calculate the area where
the adsorption is taking place. If it is of a linear region at the start
of the curve, it will represent the total surface area of the material.
If at the end of the curve, it will instead represent external surface
area of the sample. The formula to calculate the area is
where \(\rho_{l}\) is the liquid density of the adsorbate at experimental
conditions

\begin{equation}
	A = \frac{s M_m}{\rho_{l}}
\end{equation}


If the region selected is after a vertical deviation, the intercept of the line
will no longer pass through the origin. This intercept be used to calculate the
pore volume through the following equation:

\begin{equation}
	V_{ads} = \frac{i M_m}{\rho_{l}}
\end{equation}


Since the t-plot method is taking the differences between the
isotherm and a model, care must be taken to ensure that the model
actually describes the thickness of a layer of adsorbate on the
surface of the adsorbent. This is more difficult than it
appears as no universal thickness curve exists.
When selecting a thickness model, make sure that it is applicable
to both the material and the adsorbate.
Interactions at loadings that occur on the t-plot lower than the monolayer
thickness do not have any physical meaning.


“Studies on Pore Systems in Catalysts V. The t Method”,
B. C. Lippens and J. H. de Boer, J. Catalysis, 4, 319 (1965)

\subsection{\(\alpha_s\) Method}

In order to extend the t-plot analysis with other adsorbents and non-standard
thickness curves, the \(\alpha_s\) method was devised. Instead of
a formula that describes the thickness of the adsorbed layer, a reference
isotherm is used. This isotherm is measured on a non-porous version of the
material with the same surface characteristics and with the same adsorbate.
The \(\alpha_s\) values are obtained from this isotherm by regularisation with
an adsorption amount at a specific relative pressure, usually taken as 0.4 since
nitrogen hysteresis loops theoretically close at this value.

\begin{equation}
	\alpha_s = \frac{n_a}{n_{0.4}}
\end{equation}

The analysis then proceeds as in the t-plot method.

The slope of the linear section can be used to calculate the area
where the adsorption is taking place. If it is of a linear region
at the start of the curve, it will represent the total surface area
of the material. If at the end of the curve, it will instead
represent external surface area of the sample.
The calculation uses the known area of the reference material.
If unknown, the area will be calculated here using the BET method.

\begin{equation}
	A = \frac{s A_{ref}}{(n_{ref})_{0.4}}
\end{equation}


If the region selected is after a vertical deviation, the intercept of the line
will no longer pass through the origin. This intercept be used to calculate the
pore volume through the following equation:

\begin{equation}
	V_{ads} = \\frac{i M_m}{\\rho_{l}}
\end{equation}


The reference isotherm chosen for the \(\alpha_s\) method must be a description
of the adsorption on a completely non-porous sample of the same material. It is
often impossible to obtain such non-porous versions, therefore care must be taken how the reference isotherm is defined.

D.Atkinson, A.I.McLeod, K.S.W.Sing, J.Chim.Phys., 81,791(1984)


\subsection{Mesoporous size distribution}
\subsection{Microporous size distribution}
\subsection{Multiscale size distribution - DFT fitting}
\subsection{Thickness functions}
\subsection{Isosteric heat}