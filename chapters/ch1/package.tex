% !TEX root = ../../main.tex

\section{pyGAPS overview}

\subsection{Intended use cases}

The software was imagined for use in two types of scenario.
First, as a command line interface, in environments such as
\texttt{IPython} and \texttt{Jupyter}. The typical user working
in these environments is likely to be processing a small batch of
results at one time, and is interested in obtaining the results in
graphical form. For this type of application, the framework should
provide an unobtrusive way of importing the user data, as well as
present an API which does not require extensive knowledge of 
processing methods. Finally, a powerful graphing environment is
required which will allow the user to visualise their original
dataset and results.

The second envisaged application is related to bulk data processing.
Requirements here shift towards parameter control, scripting and
extensibility. The framework API should offer the option to change 
implicit parameters, select calculation limits and return the results
in a numerical form for further processing. This type of application 
is also likely to require storage of isotherms in a database or
under other types of data files.

\subsection{Core structure}

In order to offer a clear structuring of functionality, 
\texttt{pyGAPS} introduces several classes which abstract data and
functionality for facile interaction. The classes are
intuitively named: \texttt{Isotherm}, \texttt{Sample} and \texttt{Adsorbent}.

\subsubsection{The \texttt{Isotherm} class}

The \texttt{Isotherm} class is a representation of an adsorption 
isotherm i.e.\ a function of the amount adsorbed, or loading, 
with pressure at a fixed temperature. The class also contains
other information relating to the isotherm, such as the material
name and batch it describes, the adsorbate used and other 
user-defined properties.

Because the aforementioned relationship can be either a physical 
measurement defined by individual pressure-loading pairs or a model, 
describing the relationship as a function rather than discrete data, 
the Isotherm class is used as a parent class for two subclasses: 
\texttt{PointIsotherm}, describing datapoints and \texttt{ModelIsotherm} 
containing a model such as Henry, Langmuir etc., which encapsulate
the respective functionality. The two classes are interchangeable as they
share most methods and properties. Once an instance of an \texttt{Isotherm}
class is created, it can then be used for the processing, conversion 
and graphing capabilities of \texttt{pyGAPS}.

\subsubsection{The \texttt{Sample} class}

The isotherm classes contain the name and batch of the sample 
they are measured on in a string format. The user might want to 
specify other information about the material,
such as the date of synthesis or the material's density,
as well as store this information in the database.
For this case, \texttt{pyGAPS} provides the \texttt{Sample} class.
The framework uses the string values
in the isotherm to connect an \texttt{Isotherm} instance to a 
specific \texttt{Sample}.

\subsubsection{The \texttt{Adsorbate} class}

Finally, in order for many of the calculations included in 
\texttt{pyGAPS} to be performed, properties of the adsorbate 
used are needed e.g. \ liquid density, vapour pressure etc.
The \texttt{Adsorbate} class is provided for this purpose,
which is connected to an \texttt{Isotherm} class similarly to a 
\texttt{Sample}. The physical properties are either calculated 
in the background through an equation of state,
either the open source CoolProp library~\cite{bellPurePseudopureFluid2014} 
or the NIST-made REFPROP~\cite{lemmonNISTReferenceFluid1989}.
The properties can also be retrieved from the internal database 
or specified by the user.

\subsection{Creation of an Isotherm}

An \texttt{Isotherm} can be created either from the command line 
directly or through an import from a supported format. For the direct
creation, the code takes
two kinds of inputs: the data itself, in the form of a 
\texttt{pandas.DataFrame}, and the isotherm parameters describing it.
Only four parameters are strictly required:
the material name, the material batch, the adsorbate used and the 
experimental temperature. Other parameters can be passed as well 
and will be stored in the isotherm class.

\begin{python}[caption={Creating the \texttt{DataFrame}},%
    label={pyg:lst:isodata}]
isotherm_data = pandas.DataFrame({
    'pressure' : [1, 2, 3, 4, 5, 3, 2],
    'loading' : [1, 2, 3, 4, 5, 3, 2],
    'enthalpy' : [15, 15, 15, 15, 15, 15, 15],
    'xrd_peak_1' : [0, 0, 1, 2, 2, 1, 0],
})
\end{python}

The \texttt{DataFrame} must contain a column containing the pressure
points and one containing the corresponding loading points of the 
isotherm. Other columns can also be passed, when
secondary data such as enthalpy of adsorption is present at 
each measurement point. These columns will be saved in the case of
the \texttt{PointIsotherm} class and can be
plotted afterwards.

\begin{python}[float=htb, caption=Creating the \texttt{PointIsotherm},
    label={pyg:lst:isocreation}]
point_isotherm = pygaps.PointIsotherm(

    # First the pandas.DataFrame with the points
    # and the keys to what the columns represent.

    isotherm_data,

    loading_key='loading',          # The loading column
    pressure_key='pressure',        # The pressure column
    other_keys=['enthalpy',
                'xrd_peak_1'],      # The columns containing other data

    # Some of the unit parameters can be 
    # specified if desired.

    pressure_mode='absolute',       # absolute pressure
    pressure_unit='bar',            # with units of bar
    adsorbent_basis='mass',         # adsorbent mass basis
    adsorbent_unit='kg',            # with units of kg
    loading_basis='mass',           # loading mass basis
    loading_unit='g',               # with units of g

    # Finally the isotherm description
    # parameters are passed.

    'sample_name' : 'carbon',       # Required
    'sample_batch' : 'X1',          # Required
    'adsorbate' : 'nitrogen',       # Required
    't_exp' : 77,                   # Required
    't_act' : 150,                  # Recognised / named
    'user'  : 'Username',           # Recognised / named
    'DOI'   : '10.000/mydoi',       # Unknown / user specific
)
\end{python}

If no unit data is specified in the constructor, the framework will
assume that the isotherm is in units of \si{\milli\mole\per\gram} 
loading as a function of \si{\bar}. Both the units and the basis 
can be specified, as it is explained in a latter section.

Finally, the data is saved in the newly created class or used to
generate parameters for a model such as BET, Langmuir, etc.,
in the case of a \texttt{PointIsotherm} and
\texttt{ModelIsotherm} respectively. It should be noted that the creation of
\texttt{Sample} and \texttt{Adsorbate} instances is similar.

Alternatively, the isotherm can be imported from a file containing 
a format that is recognised by \texttt{pyGAPS}. Parsing from 
suitably structured JSON, CSV and Excel files is supported.

\subsection{Workflow}

Once an isotherm object is created, it will be used for all 
further processing. The class contains methods which can be 
used to inspect the data visually, or retrieve
parts of the isotherm such as the adsorption or desorption branches with
user-chosen limits or units. Singular values of pressure or 
loading can be calculated, either through interpolation in the
case of a \texttt{PointIsotherm} or by evaluation
of the internal model in the \texttt{ModelIsotherm}. For an isotherm with
datapoints, these can also be converted into different units or modes.

Characterisation functions take a single isotherm object as their
first parameter. This is the case for the BET area, Langmuir area, 
t-plot, \(\alpha_s\) plot, and pore size distribution methods.
These characterisation functions attempt to automate as much
of the process as possible. For example, the BET area limits are
automatically calculated using the 
Rouquerol~\cite{rouquerolAdsorptionPowdersPorous2013} method, 
with all the checks implemented into the code. In another example
the straight line sections of the t-plot are determined automatically
through a calculation of the second derivative of the transformed isotherm.
For detailed control, there is an option to specify
options for each individual method, such as manual BET limits, 
different thickness functions for the t-plot or Kelvin-based 
mesoporous pore distribution methods, custom parameters for the
Horvath-Kawazoe microporous pore distribution, custom
DFT or NLDFT kernels and more. The results are returned in a 
dictionary or can be directly graphed if the \texttt{verbose} 
parameter is passed.

\subsection{Units}

When computers work with physical data, units are often a matter 
that introduces confusion. Here we explain how \texttt{pyGAPS} 
handles units and other physical world concepts such as relative
pressure and mass or volume basis.

The following dimensions can be specified for an Isotherm: 
the measurement \textit{pressure}, the quantity of guest adsorbed
or \textit{loading} and the amount of adsorbent material
the loading is reported on, or \textit{adsorbent}.

Pressure can be reported either in an absolute value, in several 
common units such as \si{\bar}, torr, \si{\pascal}, or as 
\textit{relative pressure}, the pressure divided by the saturation 
vapour pressure of the adsorbate at the respective measurement
temperature. Conversions between the two modes are automatic 
and handled internally.

Both the \textit{loading} and \textit{adsorbent} can be reported
in three different bases: a molar basis, a mass basis or a volume
basis. Within each basis different units are recognised and can be
easily converted. The conversions between bases can also
easily performed if the required conversion factors (i.e.\ molar 
mass and density) are available. For \textit{loading}, these 
factors are automatically calculated internally, while
for the \textit{adsorbent} they should be provided by the user 
in the respective \texttt{Sample} class.