% !TEX root = ../../main.tex

\subsection{Predicting multicomponent adsorption}

Until now, we have only referred to isotherms pertaining to the adsorption
of a single adsorbate. However, besides gas storage, most if not all
industrial applications of adsorbents involve multiple chemical 
species undergoing competitive adsorption. Experiments involving
several adsorbents are generally difficult and time consuming.
There is therefore a requirement to predict such multicomponent systems
in order to rapidly screen for potentially interesting separations
starting from pure component data.

To this end, the several methods have been devised, with perhaps
the most common approach as considering the adsorbed phase as 
an analogue to a fluid mixture. This method, also known as the 
ideal adsorbed solution theory (IAST), has been implemented in
pyGAPS and will be presented here. Other multicomponent theories
such as real adsorbed solution theory (RAST) or the Nitta model
exist but usually require more information about specific
surface binary activity coefficients to be well suited to a
general high throughput approach.

\subsubsection{Ideal adsorbed solution theory}
