% !TEX root = ../../main.tex

\subsection{Assessing porosity}

Characterization of pore sizes and their distribution in a porous material
is often the main goal when performing adsorption experiments.
When it comes to such materials, three kinds of pore sizes can be
defined, based on their lengthscale: micropores (\SI{< 2}{\nano\metre}),
mesopores (\SIrange{2}{50}{\nano\metre}), and macropores (\SI{> 50}{\nano\metre}).

Macropores are generally unable to be characterised through adsorption,
with methods such as mercury intrusion porosimetry as the standard for pore
size determination at this lengthscale, although other alternatives have
been suggested~\cite{rouquerolCharacterizationMacroporousSolids2012} due
to the high toxicity of mercury.

In the mesopore range, \textit{``classical''} methods are often
used, which are based on the application of Kelvin's equation
pertaining to capillary condensation. This equation calculates the
critical pressure at which the fluid completely files a pore of a
specific diameter. It is applicable to a range of geometries
is used in multiple approaches such as the Barrett-Joyner-Halenda
(BJH) method or the Dollimore-Heal (DH) method.

For microporous materials, the Kelvin equation, with its assumption
of continuous fluid properties and comparable density of the adsorbed
state to bulk liquid density breaks down. An atomistic approach
is required, addressing the interaction between solid-fluid
and fluid-fluid through potential functions. The Horvath-Kawazoe or
HK method is often used, alongside the older Dubinin-Radushkevich
model.

Finally, computational methods based on Grand Canonical Monte Carlo
(GCMC) and density functional theory (DFT) together with its
derivations such as non-local DFT (NLDFT) and quenched solid state DFT
(QSDFT) should be mentioned, as they can be used for multiscale
(micropore and mesopore) characterisation. These methods rely on
\textit{in-silico} simulation of isotherms on a range of pore sizes,
which can then be collated in a so-called \textit{kernel}, able to
be used to deconvolute an experimental isotherm to obtain a
pore size distribution.

It should be noted that all methods here described require knowledge of
the pore geometry and depend on the material having pores which
conform to a well-defined shape. Real adsorbents usually have
interconnected networks of irregular pores, which the ideal pores
used in these models merely approximate.

\subsubsection{Mesoporous size distribution}

\paragraph{The Kelvin equation}

Since the Kelvin equation is the basis of most mesoporous PSD
calculations, such as the BJH and DH methods, it will be described
first. The original form of the equation (\autoref{pyg:eqn:kelvin})
gives the dependence of pressure on the radius of curvature of a
meniscus in a pore \(r\) by means of surface tension \( \gamma \),
molar liquid volume, here expressed as \(v_l=M_m/\rho_l\) and the
fluid contact angle with the surface
\( \theta \). The fluid is often assumed to be fully wetting, with
\(\theta=0\) and \(\cos\theta=1\).
%
\begin{equation}\label{pyg:eqn:kelvin}
	\ln\Big(\frac{p}{p_0}\Big) = -\frac{\gamma M_m}{\rho_l RT}\frac{2 \cos\theta}{r}
\end{equation}
%
To apply the Kelvin equation to different types of pore systems, the
generalized form presented in \autoref{pyg:eqn:kelvin-general}
is used. It replaces the meniscus radius by a mean radius of curvature
\(r_m\).
%
\begin{equation}\label{pyg:eqn:kelvin-general}
	\ln\Big(\frac{p}{p_0}\Big) = -\frac{\gamma M_m}{\rho_l RT}\frac{2}{r_m}
\end{equation}
%
The mean radius of curvature is defined through the two principal
radii of the curved interface.
%
\begin{equation}\label{pyg:eqn:kelvin-mradius}
	\frac{1}{r_m} = \frac{1}{2}\Big(\frac{1}{r_1}+\frac{1}{r_2}\Big)
\end{equation}

The relationship of the Kelvin radius \(r_m\) to the actual pore
radius is more subtle, as it depends on pore geometry and the
filling state of the pore~\cite{doAdsorptionAnalysisEquilibria1998}.
If considering a cylindrical pore open at both ends, the radius reduces
to the original Kelvin equation during the desorption phase, and takes
other values with different combinations of parameters
as can be seen in \autoref{pyg:tbl:kelvin-meniscus}.

\begin{table}[htb]
	\centering
	\caption{Assumed relationship between pore geometry and meniscus geometry during adsorption and desorption}%
	\label{pyg:tbl:kelvin-meniscus}
	\begin{tabular}{ccc}
		\toprule
		                    & \multicolumn{2}{c}{Assumed meniscus geometry}              \\
		\cmidrule{2-3}
		Pore geometry       & Adsorption                                    & Desorption \\
		\midrule
		infinite slit       & infinite radius cylindrical                   & concave    \\
		open-ended cylinder & cylindrical                                   & spherical  \\
		sphere              & spherical                                     & spherical  \\
		\bottomrule
	\end{tabular}
\end{table}

According to Rouquerol~\cite{rouquerolAdsorptionPowdersPorous2013},
in adopting this approach, it is assumed that:

\begin{itemize}

	\item The Kelvin equation is applicable over the pore
	      range (mesopores). Therefore in pores which are below a
	      certain size (around \SI{2.5}{\nano\meter}), the granularity
	      of the liquid-vapour interface becomes too large for classical
	      bulk methods to be applied.
	\item The meniscus curvature is controlled by pore size and
	      shape. Ideal shapes for the curvature are assumed.
	\item Pores are rigid and of well defined shape. Their geometry is
	      considered to be invariant across the entire adsorbate.
	\item The filling/emptying of each pore does not depend on its location.
	\item Adsorption on the pore walls is not different from
	      surface adsorption.

\end{itemize}

\paragraph{The Barrett, Joyner and Halenda (BJH) method}

The BJH method for calculating pore size distribution
is based on a classical description of the adsorbate behaviour
in the adsorbent pores~\cite{barrettDeterminationPoreVolume1951}.
Under this method, the guest is adsorbing on the pore walls
following an ideal model, and decreasing the apparent pore volume until
condensation takes place, filling the entire pore. The critical radius
is a sum of two radii, the adsorbed layer thickness, which can be
modelled by a thickness model (such as Halsey, Harkins \& Jura or similar
as presented in \autoref{pyg:charac:tcurve})
and a critical radius model for condensation/evaporation,
based on a form of the Kelvin equation.
%
\begin{equation}\label{pyg:eqn:condensradius}
	r_p = t + r_k
\end{equation}

The original model uses the desorption curve as a basis for calculating
pore size distribution. Between two points of the curve, the volume
desorbed can be described as the volume contribution
from pore evaporation and the volume from layer thickness decrease as
per \autoref{pyg:eqn:condensradius}. The computation is done
cumulatively, starting from the filled pores and calculating the volume
adsorbed in a pore for each point using the following equation:
%
\begin{align}
	V_p & = \Big( \frac{\bar{r}_p}{\bar{r}_k + \Delta t_n} \Big)^2
	\Big(\Delta V_n - \Delta t_n \sum_{i=1}^{n-1} \Delta A_p
	+ \Delta t_n \bar{t}_n \sum_{i=1}^{n-1} \frac{\Delta A_p}{\bar{r}_p}\Big)\intertext{where}
	A   & = 2 \Delta V_p / r_p
\end{align}
%
In the BJH equation:

\begin{itemize}

	\item \(\Delta A_p\) is the area of the pores
	\item \(\Delta V_p\) is the adsorbed volume change between two points
	\item \(\bar{r}_p\) is the average pore radius calculated as a 
		  sum of the Kelvin radius and layer thickness of the pores at pressure p between two measurement points
	\item \(\bar{r}_k\) is the average Kelvin radius between two
	      measurement points
	\item \(\bar{t}_n\) is the average layer thickness
	      between two measurement points
	\item \(\Delta t_n\) is the average change in layer thickness
	      between two measurement points

\end{itemize}

Then, by plotting \(\Delta V / (2*\Delta r_p)\) versus the width
of the pores calculated for each point, the pore size distribution
can be obtained.

\paragraph{The Dollimore-Heal (DH) method}

The DH or Dollimore-Heal
method~\cite{dollimorePoresizeDistributionTypical1970}
is an extension of the BJH method which takes into account the
geometry of the pores by introducing a length component.
Like the BJH method, it is based on a classical description of
the adsorbate behaviour in the adsorbent pores and uses the
same assumptions. The modified equation becomes:
%
\begin{align}
	V_p & = \Big(\frac{\bar{r}_p}{\bar{r}_k + \Delta t_n}\Big)^2
	\Big(\Delta V_n - \Delta t_n \sum_{i=1}^{n-1} \Delta A_p
	+ 2 \pi \Delta t_n \bar{t}_n \sum_{i=1}^{n-1} L_p\Big)       \\
	%
	A   & = 2 \Delta V_p / r_p                                   \\
	%
	L   & = \Delta A_p / 2 \pi r_p
\end{align}

The meaning of each symbol is the same as in the BJH equation.
As before, a plot of \(\Delta V/(2*\Delta r_p)\) versus the width of
the pores calculated for each point yields the pore size distribution.
The Dollimore-Heal method is used on the desorption branch,
of the isotherm but often also applied on the adsorption branch.

\subsubsection{Microporous size distribution}

When it comes to micropores (width \SI{< 2}{\nano\metre}), classical
fluid methods stop being viable. Adsorption of molecules in these pores
of comparable scale is highly dependent on the surface properties and
on guest-host interaction and leads to adsorbate phase densities
which are often very different than those in the bulk liquid state.
In order to model adsorption in such pores, a good description of
both solid-fluid and fluid-fluid potential functions is required.

\paragraph{The Horvath-Kawazoe (HK) method}

The H-K method attempts to describe the adsorption within pores
by calculation of the average potential energy for a
pore~\cite{horvathMethodCalculationEffective1983}.
The method starts by assuming the relationship between the gas
phase as being:
%
\begin{equation}
	R_g T \ln\Big(\frac{p}{p_0}\Big) = U_0 + P_a
\end{equation}
%
Here \(U_0\) is the potential function describing the surface to adsorbent
interactions and \(P_a\) is the potential function describing
guest-guest interactions. This equation is derived from the
equation of the free energy of adsorption at constant temperature where
the term \(T \Delta S^{tr}(w/w_{\infty})\) is assumed to be negligible.

If it is assumed that a Lennard-Jones-type potential function can
accurately describe the interactions between adsorbate and surface
molecules, then the two contributions to the total potential can be
replaced by the extended function in \autoref{pyg:eqn:hk}.
%
\begin{multline}\label{pyg:eqn:hk}
	RTln(p/p_0) =   N_A\frac{n_a A_a + n_A A_A}{2 \sigma^{4}(l-d)}
	\times \int_{d/_2}^{1-d/_2}
	\Big[
	- {\Big(\frac{\sigma}{r}\Big)}^{4}
	+ {\Big(\frac{\sigma}{r}\Big)}^{10}
	- {\Big(\frac{\sigma}{l-r}\Big)}^{4}
	+ {\Big(\frac{\sigma}{l-r}\Big)}^{4}
	\Big] \mathrm{d}x
\end{multline}
%
Here \(l\) is the width of the pore, \(d\) defined as \(d=d_a+d_A\) is
the sum of the diameters of the adsorbate and adsorbent molecules,
\(n_a\) is number of molecules of adsorbent
and \(A_a\) and \(A_A\) the Lennard-Jones potential constant of the
fluid molecule and solid molecule respectively. They are defined as:
%
\begin{align}
	A_a & = \frac{6mc^2\alpha_a\alpha_A}{\alpha_a/\varkappa_a + \alpha_A/\varkappa_A}
	%
	\intertext{and}
	%
	A_a & = \frac{3mc^2\alpha_A\varkappa_A}{2}
\end{align}
%
where \(m\) is the mass of an electron, \(\alpha_a\) and \(\alpha_A\) are
the polarizability of the adsorbate and adsorbate molecule
and \(\varkappa_a\) and \(\varkappa_A\) the magnetic susceptibility of
the adsorbate molecule and adsorbent molecule, respectively.

The HK method is applicable to slit pores, and it can be extended
through modification to cylindrical and spherical pores. It is worth noting
that there are several assumptions which limit its applicability.

\begin{itemize}

	\item The HK method is reliant on knowledge of the properties of
	      the surface atoms. This assumption is true only if the
	      material surface is homogenous. Furthermore,
	      longer range interactions with multiple surface layers are
	      not considered.
	\item Each pore is modelled as uniform and of infinite length.
	      Materials with varying pore shapes or highly interconnected
	      networks may not give realistic results.
	\item Only dispersive forces are accounted for.
	      If the adsorbate-adsorbent interactions
	      have other specific contributions, as is the case
	      for dipole-dipole or \(pi\)-backbonding interactions,
	      the Lennard-Jones potential function will not be
	      an accurate description.
	\item The model does not have a description of capillary condensation.
	      This means that the pore size distribution can only
	      be considered accurate up to a maximum of
	      \SIrange{3}{5}{\nano\metre}.

\end{itemize}

\subsubsection{Multiscale computational methods}

DFT theory emerged as a rigorous description of molecular
adsorption in pores~\cite{seatonNewAnalysisMethod1989}.
It calculates the properties of the fluid
directly from the forces acting between constituent molecules
through a statistical mechanical approach. Latter developments,
like non-local DFT (NLDFT)~\cite{tarazonaPhaseEquilibriaFluid1987},
which makes an account for short range molecule correlation
and therefore for the changes in the density profile
around the pore walls, and quenched solid state DFT
(QSDFT)~\cite{neimarkQuenchedSolidDensity2009}, which allows for
heterogeneity of pore walls to be incorporated in the model,
have improved the accuracy of the method.

The density functional theory approach can therefore simulate
adsorption isotherms on pores of different geometries and sizes.
By defining a pore geometry and running the simulation with a
range of pore radii, a collection of isotherms is obtained.
If an experimental isotherm is thought of as a sum of
adsorption isotherms in different size pores, then it stands
to reason that the preponderence of those pores can be calculated
through deconvolution.

Since the DFT method can model adsorbate condensation behaviour,
as well as micropore filling and multilayer adsorption, it can
be used for multiscale pore size distribution. The downside is
that DFT kernels are temperature, probe, pore-geometry and
adsorbent specific and as such are not universally applicable.
