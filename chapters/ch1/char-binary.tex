% !TEX root = ../../main.tex

\subsection{Predicting multicomponent adsorption}\label{pyg:iast}

Until now, we have only referred to isotherms pertaining to the adsorption
of a single adsorbate. However, besides gas storage, most if not all
industrial applications of adsorbents involve multiple chemical
species undergoing competitive adsorption. Experiments involving
several adsorbents are generally difficult and time consuming.
There is therefore a need to predict such multicomponent systems
in order to rapidly screen for potentially interesting separations
starting from pure component data.

To this end, several methods have been devised, with perhaps
the most common approach as considering the adsorbed phase as
an analogue to a fluid mixture, in a manner analogous to the
vacancy solution theory presented in \autoref{pyg:models:vst}.
This method, also known as ideal adsorbed solution theory (IAST)
will be presented here. Other multicomponent theories
such as real adsorbed solution theory (RAST) or the Nitta model
exist, but usually require more information such as specific
surface binary activity coefficients, and are therefore less suited to
a general approach. Furthermore, the IAST method has been
proven~\cite{cessfordEvaluationIdealAdsorbed2012,%
	vanheestIdentificationMetalOrganic2012} to be successful in the
prediction of multicomponent adsorption even when considering
adsorption in microporous materials such as MOFs.

Originally derived by \citet{myersThermodynamicsMixedgasAdsorption1965},
the IAST model is based on three main assumptions. First, that the
same surface area is available for all components. Then it is
assumed that the mixture behaves like an ideal solution at constant
spreading pressure and temperature. This means the spreading pressure of
each component is the same as the surface potential of the mixture \(\pi\)
and therefore the mean strength of interaction is equal between all
molecules of solution. Finally the surface adsorbed phase is assumed
to be at equilibrium with the gas phase. \autoref{pyg:eqn:iast}
can then be written, where \(p_{i,g}\) is the partial pressure of
the component in the gas phase, \( p_i^0(\pi)\) the pressure of
that component which will give the same spreading pressure on the
surface and \(x_i\) is its fraction in the adsorbed phase.
%
\begin{align}
	%
	p_{i,g}             & = p_i^0(\pi)x_i                              %
	\label{pyg:eqn:iast}                                               \\
	%
	\frac{\pi A}{R_g T} & = \int_{0}^{P_{i}^{0}} n_{ads,i} d\ln{(p_i)} %
	\label{pyg:eqn:iast-2}                                             \\
	%
	\sum_{1=1}^{N} x_i  & = 1                                          %
	\label{pyg:eqn:iast-3}
\end{align}

In order to account for non-ideality of the gas phase
fugacity can be used instead of pressure,
with a suitable equation of state required to relate it to bulk pressure.
The pressure and spreading pressure are then related through
the integrated Gibbs isotherm (\autoref{pyg:eqn:iast-2}). As the sum of all
fractions in the adsorbed phase is equal to unity (\autoref{pyg:eqn:iast-3}),
the equations can be solved for a given pressure and gas composition to give
the fractions of each component in the adsorbed phase at the hypothetical
pressure \(P_{i}^{0}\). The total amount adsorbed can then be calculated.
%
\begin{equation}
	\frac{1}{n_{ads,t}} = \sum_{1=1}^{N} \frac{x_i}{n_{ads,i}^0}
\end{equation}
%
Finally, the number of moles adsorbed for each component is given
by multiplying the total amount adsorbed by the fraction of each
component.
%
\begin{equation}
	n_{ads, i} = x_i n_{ads,t}
\end{equation}

In order to calculate the spreading pressure, an adsorption model
such as the ones presented in \autoref{pyg:models} is required.
By substituting the amount adsorbed (\(n_{ads}\)) in \autoref{pyg:eqn:iast}
and calculation of the resulting integral, an expression for
spreading pressure can be determined. The integral can only be
evaluated analytically in the case of a few models, for example in
the previously mentioned Henry, Langmuir and multi-site Langmuir,
BET, Quadratic and Temkin models. In other cases, a numerical approach
must be used for its calculation. An alternative to using a
model is to use an interpolation method between recorded experimental
points to approximate the amount adsorbed at a pressure \(p\). Numerical
quadrature is then used to evaluate the integral and obtain the
spreading pressure.