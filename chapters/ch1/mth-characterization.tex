\subsection{Characterisation of porous materials}

\subsubsection{BET surface area}

The BET surface area is one of the first standardised methods to calculate the
surface area of a porous material~\cite{brunauerAdsorptionGasesMultimolecular1938}. 
It is generally applied on isotherms obtained
through N2 adsorption at 77K, although other adsorbates (Ar, Kr) have been used.

It assumes that the adsorption happens on the surface of the material in
incremental layers according to the BET equation (\ref{eqn:pyg:bet}. 
Even if the adsorbent is porous, the initial amount adsorbed 
(usually between 0.05 - 0.4 \(p/p_0\)) can be
modelled through the BET equation in its linear form:

\begin{equation}
    \frac{p/p_0}{n_{ads} (1-p/p_0)} = \frac{1}{n_{m} C} + \frac{C - 1}{n_{m} C}(p/p_0)
\end{equation}

If we plot the isotherm points as
\({(p/p_0)}/{n_{ads}(1-p/p_0)}\) versus \(p/p_0\), a linear region
can usually be found. The slope and intercept of this line
can then be used to calculate \(n_{m}\), the amount adsorbed at the
statistical monolayer, as well as \(C\), the BET constant.

\begin{gather}
    n_{m} = \frac{1}{s+i} \\
    C = \frac{s}{i} + 1
\end{gather}

The surface area can then be calculated by using the moles
adsorbed at the statistical monolayer. If the specific area taken
by one of the adsorbate molecules on the surface is known, it is
inserted in the following equation together with Avogadro's number:

\begin{equation}
    a_{BET} = n_m A_N \sigma
\end{equation}


While a standard for surface area determinations, the BET area
should be used with care, as there are many assumptions made in
the calculation. To augment the validity of the BET
method, Rouquerol~\cite{rouquerolAdsorptionPowdersPorous2013} proposed
several checks to ensure that the BET region selected is valid:

\begin{itemize}

	\item The BET constant (\(C\)) obtained should be positive
    \item In the corresponding Rouquerol plot where \(n_{ads}(1-p/p_0)\)
    is plotted with respect to \(p/p_0\), the points chosen for BET
	analysis should be strictly increasing
	\item The loading at the statistical monolayer should be
	situated within the limits of the BET region

\end{itemize}

All these checks are implemented in pyGAPS.
Regardless, the BET surface area should still be interpreted carefully.
Since adsorption takes place on the pore surface, microporous materials
which have pores in similar size as the molecule adsorbed will not
give a realistic surface area. Furthermore, the cross-sectional area
of the molecule on the surface cannot be guaranteed. For example, 
nitrogen has been known to adopt a different conformation on the surface
of some materials due to inter-molecular forces, which effectively
lowers its cross-sectional area.

\subsubsection{Langmuir surface area}

The Langmuir equation (\ref{eqn:pyg:langmuir}) can be rearranged as:

\begin{equation}
	\frac{p}{n} = \frac{1}{K n_m} + \frac{p}{n_m}
\end{equation}

Assuming the data can be fitted with a Langmuir model, by plotting
\({P}/{n}\) against pressure, a line will be obtained. The slope and
intercept of this line can then be used to calculate \(n_{m}\),
the amount adsorbed at the monolayer, as well as K, the Langmuir constant.

\begin{gather}
	n_m = \frac{1}{s} \\
	K = \frac{1}{i * n_m}
\end{gather}

The surface area can then be calculated by using the moles adsorbed at the
monolayer using the same assumptions as when obtaining the BET surface area.

\begin{equation}
	a_{Langmuir} = n_m A_N \sigma
\end{equation}

The Langmuir method for determining surface area assumes that only one single
layer is adsorbed on the surface of the material. As most adsorption processes
(except chemisorption) don't follow this behaviour, it is important to regard
the Langmuir surface area as an estimate.

\subsubsection{Ideal isotherms or thickness functions}

A lot of work has been focused on obtaining the ideal curve 
which describes the thickness of the adsorbed
layer on a surface

\subsubsection{t-plot Method}

The t-plot method attempts to relate the adsorption on a material
with a thickness curve~\cite{lippensStudiesPoreSystems1965}. 
A plot is constructed, where the isotherm loading
data is plotted versus thickness values obtained through the model.
It stands to reason that, in the case when the experimental adsorption
curve follows the model, a straight line will be obtained with its
intercept through the origin. However, since in most cases there
are differences between adsorption in the pores and ideal surface
adsorption, the t-plot will deviate and form features which can
be analysed to describe the material characteristics.

\begin{itemize}

	\item A sharp vertical deviation will indicate condensation
	      in a type of pore.
	\item A gradual slope will indicate adsorption on the
	      wall of a particular pore.

\end{itemize}

The slope of the linear section can be used to calculate the area where
the adsorption is taking place. If it is of a linear region at the start
of the curve, it will represent the total surface area of the material.
If at the end of the curve, it will instead represent external surface
area of the sample. The formula to calculate the area is
where \(\rho_{l}\) is the liquid density of the adsorbate at experimental
conditions

\begin{equation}
	A = \frac{s M_m}{\rho_{l}}
\end{equation}

If the region selected is after a vertical deviation, the intercept of the line
will no longer pass through the origin. This intercept be used to calculate the
pore volume through the following equation:

\begin{equation}
	V_{ads} = \frac{i M_m}{\rho_{l}}
\end{equation}

Since the t-plot method is representing a difference between the
isotherm and a model, care must be taken to ensure that the model
actually describes the thickness of a layer of adsorbate on the
surface of the adsorbent. This is more difficult than it
appears as no universal thickness curve exists.
When selecting a thickness model, make sure that it is applicable
to both the material and the adsorbate.
Interactions at loadings that occur on the t-plot lower than the monolayer
thickness do not have any physical meaning.

\subsubsection{\(\alpha_s\) Method}

In order to extend the t-plot analysis with other adsorbents and non-standard
thickness curves, the \(\alpha_s\) method was 
devised~\cite{atkinsonAdsorptivePropertiesMicroporous1984}.
Instead of a formula that describes the thickness of the adsorbed layer, 
a reference isotherm is used. This isotherm is measured on a non-porous 
version of the material with the same surface characteristics and with 
the same adsorbate.
The \(\alpha_s\) values are obtained from this isotherm by regularisation with
an adsorption amount at a specific relative pressure, usually taken as 0.4 since
nitrogen hysteresis loops theoretically close at this value.

\begin{equation}
	\alpha_s = \frac{n_a}{n_{0.4}}
\end{equation}

The analysis then proceeds as in the t-plot method. 
The slope of the linear section can be used to calculate the area
where the adsorption is taking place. If it is of a linear region
at the start of the curve, it will represent the total surface area
of the material. If at the end of the curve, it will instead
represent external surface area of the sample.
The calculation uses the known area of the reference material.
If unknown, the area will be calculated here using the BET method.

\begin{equation}
	A = \frac{s A_{ref}}{(n_{ref})_{0.4}}
\end{equation}


If the region selected is after a vertical deviation, the intercept of the line
will no longer pass through the origin. This intercept be used to calculate the
pore volume through the following equation:

\begin{equation}
	V_{ads} = \frac{i M_m}{\rho_{l}}
\end{equation}


The reference isotherm chosen for the \(\alpha_s\) method must be a description
of the adsorption on a completely non-porous sample of the same material. It is
often impossible to obtain such non-porous versions, therefore care must be 
taken how the reference isotherm is defined.


\subsubsection{Isosteric heat}

The isosteric heats are calculated from experimental data using the Clausius-Clapeyron
equation as the starting point:

\begin{equation}
    \Big( \frac{\partial \ln P}{\partial T} \Big)_{n_a} = -\frac{\Delta H_{ads}}{R T^2}
\end{equation}

Where \(\Delta H_{ads}\) is the enthalpy of adsorption. In order to approximate the
partial differential, two or more isotherms are measured at different temperatures. The
assumption made is that the heat of adsorption does not vary in the temperature range
chosen. Therefore, the isosteric heat of adsorption can be calculated by using the pressures
at which the loading is identical using the following equation for each point:

\begin{equation}
    \Delta H_{ads} = - R \frac{\partial \ln P}{\partial 1 / T}
\end{equation}

and plotting the values of \(\ln P\) against \(1 / T\) we should obtain a straight
line with a slope of \(- \Delta H_{ads} / R\).

The isosteric heat is sensitive to the differences in pressure between the two isotherms. If
the isotherms measured are too close together, the error margin will increase.
The method also assumes that enthalpy of adsorption does not vary with temperature. If the
variation is large for the system in question, the isosteric heat calculation will give
unrealistic values.

Even with carefully measured experimental data, there are two assumptions used in deriving
the Clausius-Clapeyron equation: an ideal bulk gas phase and a negligible adsorbed phase
molar volume. These have a significant effect on the calculated isosteric heats of adsorption,
especially at high relative pressures and for heavy adsorbates.

\subsubsection{Kelvin model}

The standard kelvin equation for determining critical pore radius for condensation or
evaporation.

The Kelvin equation assumes that adsorption in a pore is not different than adsorption
on a standard surface. Therefore, no interactions with the adsorbent is accounted for.

Furthermore, the geometry of the pore itself is considered to be invariant across the
entire adsorbate.