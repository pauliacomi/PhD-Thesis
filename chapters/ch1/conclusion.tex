% !TEX root = ../../main.tex

\section{Conclusion}

In this chapter, a framework for adsorption data processing is
presented, which focuses on standardisation of porous material
characterisation through adsorption methods and large-scale
processing of isotherms. Prediction of multicomponent
adsorption is also made possible through IAST calculations.

Analysis of isotherms available in the NIST ISODB, which are collated
from literature, shows a comprehensive set of data. A brief comparison
between two ways of calculating surface area (BET and Langmuir) as applied
to nitrogen isotherms confirms long-standing recommendations of their
applicability and use. However, further analysis reveals a large
variability present in the measured isotherms on metal-organic
frameworks.

The other chapters in this thesis will explore the origin of some of
this variability, such as inherent defects in the material (\autoref{def}),
changes introduced by shaping (\autoref{shaping}) and of course, through
unique behaviours of MOFs such as flexibility arising from its building
blocks (\autoref{dut}).

\FloatBarrier%
\pagebreak