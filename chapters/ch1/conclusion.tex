% !TEX root = ../../main.tex

\section{Conclusion}

In this chapter, a framework for adsorption data processing is
presented, which focuses on standardisation of porous material
characterisation through adsorption methods and large-scale
processing of isotherms. Prediction of multicomponent
adsorption is also made possible through \gls{IAST} calculations.

Analysis of isotherms available in the \gls{NIST} ISODB, which are collated
from literature, shows a comprehensive set of data. A brief comparison
between two ways of calculating surface area (\gls{BET} and Langmuir) as applied
to nitrogen isotherms confirms long-standing recommendations of their
applicability and use. However, further analysis reveals a large
variability present in the measured isotherms on metal-organic
frameworks. It was initially hoped to use this dataset as a basis
for discovering structure-property relationships pertaining to 
metal-organic frameworks. However, the uncertainty present in the database
and consequently, in the scientific literature, precludes its 
use for such a purpose. Is therefore essential to rely on carefully
recorded lab data if a focus on a particular variable is required,
as where systematic errors can be greatly reduced.
It should be pointed out that the \gls{NIST} adsorption dataset can reveal
important insight into areas such as inherently repeatable
\glspl{MOF} that have the potential to be used as reference materials or the
existence of multiple phases of the same structure.

The other chapters in this thesis will explore the origin of some of
this variability, such as inherent defects in the material (\autoref{def}),
changes introduced by shaping (\autoref{shaping}) and of course, through
unique behaviours of \glspl{MOF} such as flexibility arising from its building
blocks (\autoref{dut}).

\FloatBarrier%
\pagebreak