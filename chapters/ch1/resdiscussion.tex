% !TEX root = ../../main.tex

\section{Case studies}

The NIST database of standard materials is a 

The complete dataset was downloaded using the publicly available API. 
The database yielded \~18500 isotherms.
Some sorting was performed on the dataset:

\begin{itemize}
    \item All isotherms which were recorded on a volume 
    basis were discarded. As the density of the samples in
    question is not known, they could not be converted to 
    a mass basis. 
    \item No isotherms with less than 6 points were considered,
    as sufficient datapoints are needed for a model.
    \item Isotherms were then converted into \si{\milli\mol\per\gram},
    to ensure a consistent unit set.
    \item Possible outliers were removed from the data by selecting 
    only isotherms recorded under \SI{100}{\bar}, with maximum capacities
    under \SI{100}{\milli\mol} and with a temperature of under 
    \SI{443}{\kelvin}.
\end{itemize}

The process of data collation reduced the number of isotherms
to \~15800. A distribution of the isotherms as a function of adsorbate
and temperature can be found in \autoref{def:pyg:nist-set}.

\begin{figure}[htb]
    \centering

    \missingfigure[figwidth=0.5\textwidth]{Co2 isotherms go here}

    \caption{\ce{CO2} adsorption isotherms at \SI{303}{\kelvin}}%
    \label{def:pyg:nist-set}
\end{figure}