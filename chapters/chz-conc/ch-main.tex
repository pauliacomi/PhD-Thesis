% !TEX root = ../../main.tex

% Reset graphics to the current folder
\graphicspath{ {\thisch/figures/} }

\chapter*{Conclusion and outlook}\label{conclusion}
\addcontentsline{toc}{chapter}{Conclusion and outlook}

In summary, this thesis investigates the contribution of several
different factors to the reproducibility of adsorption characteristics
of metal organic frameworks. While it is true that irregularities 
in measurement and processing methodology lead to non-trivial
sources of error~\cite{nguyenReferenceHighpressureCO22018, %
parkHowReproducibleAre2017}, the main conclusion to be drawn 
from this work is that \glspl{MOF}
present an intrinsic variability which depends on their crystal
structure and post processing.

The results in \autoref{pyg} show that, while useful for reference
purposes, published adsorption data contains too much
uncertainty for meaningful insight to be extracted about the properties
of adsorbent materials starting from structural chemical characteristics.
For this purpose, high quality laboratory data, recorded in strictly 
controlled environments and conditions is required in order to 
minimise systematic error. If standardised measurement procedures 
are devised for adsorption experiments, which are then extended 
by canonical processing methodology such as the Python processing 
framework herein presented, a database which can fulfil the accuracy
requirements for obtaining structure-property relationships may be
feasible. This kind of database would be invaluable when used in
conjunction with computer simulations
to allow for model confirmation and adjustment. Furthermore, using machine
learning techniques, it may also be possible to quickly predict the
applicability of particular materials for different applications
based on surface area, pore size and functionalisation.

The results presented in \autoref{pyg} also highlight that
\glspl{MOF}, in spite of the well-defined crystal structure and 
chemical formulae, are capable of displaying a high variability in
their adsorption properties. 
As a cursory observation of the HKUST-1 dataset shows, the form of the 
material (such as nanoparticles or hollow
spheres~\cite{liControllableSynthesisMetal2013}), as well as 
deliberate defects introduced in the 
framework~\cite{barinDefectCreationLinker2014}, may decrease or 
increase the surface area and pore volume of the resulting material.
Furthermore IRMOF-3, a flexible \gls{MOF}, 
is shown to have the poorest reproducibility of all studied materials,
likely due to the dependency of framework collapse on 
activation procedure~\cite{engelActivationDependentBreathingFlexible2017}.
The work also points out several \glspl{MOF} 
which may be inherently reproducible and can be further investigated as 
potential reference or benchmarking materials, such as 
ZIF-8 and MIL-100.
However, it can be seen that adsorption methodology common in the
\textbf{past} becomes unsuitable for the characterisation of \glspl{MOF}. 
A combined approach is required to further explore the underlying 
causes of variability.

Chapter~\ref{calo} introduces complementary characterisation 
for adsorbent materials using direct measurement of the
differential heat of adsorption. The technique is shown to be 
invaluable for obtaining insight into the energetics of 
adsorption and the contributions of guest-guest, guest-host 
and even host-host interactions. In this chapter,
using microcalorimetry and screening of multiple gaseous probes,
Zr Fumarate is highlighted as a potential adsorbent material for the 
separation of propane and propylene. Surprisingly, this \gls{MOF} shows
to be selective for propane over propylene, a desirable characteristic
for an industrial process where \ce{C3H6} is the required
outlet stream. Further experimental and simulation study of this
material is suggested in order to confirm the observed effect.
From, a methodology point of view, the challenge lies in combining
the measurement of the differential enthalpy of adsorption with a 
high throughput approach. 
Indeed, attempts are already underway which use inexpensive infrared 
sensors to rapidly screen materials based on the heat effect of 
adsorption~\cite{wollmannInfrasorbOpticalDetection2012}, although
they are currently designed for the measurement of a heat signal 
which can be correlated to integral rather than differential enthalpy.

The method can therefore afford a better characterisation of \glspl{MOF},
which is \textbf{presently} used in the final three chapters 
to dive deeper into the relationship between \gls{MOF} defects,
post-processing and constitutional properties,
and their adsorption characteristics, using predictors calculated from
adsorption isotherms and enthalpy curves.

Chapter~\ref{def} takes a closer look at the propensity of the 
crystalline structures of metal organic framework for defects,
both conventional analogues of crystal defects and ones unique
to porous materials. As defects can be beneficial for increasing 
surface area and capacity, and especially for the creation
of specific sites for adsorption and catalysis, methods for 
defect generation in \glspl{MOF} are of potential scientific interest.
In this chapter, the possibility to introduce missing linker and 
missing cluster defects in the highly stable framework of UiO-66(Zr)
through leaching in an acid solution is explored as an alternative 
to modulated synthesis. The method is shown to be able to generate
the two types of defects, with both monotopic acid and leaching 
solvent being a distinct influence on the resulting porosity.
Such defects are seen to dramatically affect the total capacity 
of the \gls{MOF}, increasing it by a factor of two in the case of \gls{TFA}
leaching in \gls{DMSO}. Their presence in what is assumed
to be the pristine structure of the material is also seen 
to be commonplace with this \gls{MOF}, and can be conjectured to be an
important source of variability in other as-synthesised materials.

Following the timeline of \glspl{MOF} towards industrialisation, the 
large-scale use of these adsorbents cannot occur without a suitable
support, be it shaped pellets, monoliths or membranes. Therefore,
the effect of the shaping of \glspl{MOF} should be studied in more detail,
a topic studied in \autoref{shaping}. The results prove that the method
of shaping can have an unexpected divergent effect on the adsorption 
properties of different frameworks. The comprehensive study on 
alumina shaped versions of three topical materials highlights
MIL-127 as a \gls{MOF} in which this forming method has the least impact on
its surface properties, surface area and pore volume across all 
probe gasses used. Going forward, one aspect which has been only 
summarily explored is the prospect of \gls{MOF}-binder interaction. 
As the surface of \gls{MOF} crystals lends itself to functionalisation, 
careful choice of binder or post-synthesis modification may open the
door to a new class of composite adsorbent materials.

Finally, in \autoref{dut}, the assumption of a rigid adsorbent 
is put into question as an intrinsic source of variability in adsorption
isotherms. In this case, an extreme example of framework flexibility
is explored, leading to the counterintuitive phenomenon of \gls{NGA}
in DUT-49.
Microcalorimetry is seen to be a powerful technique for the study
of not only the transition mechanism, but also of the types of
interactions with the underlying framework and the pore filling
mechanism. As part of a large collaborative effort comprising
computational and advanced characterisation methods, the
influence on the adsorption behaviour of physical factors such
as temperature, adsorptive and crystal size can be studied, allowing
a prediction of the conditions in which the phases of the structure
enter a metastable regime. A relationship between the condensation 
of adsorbate in the large mesoporous voids of the network and \gls{NGA}
can be observed, highlighting the contribution of condensation 
stresses to overcoming the energetic barrier of compliance, also 
evidenced through the empirical relationship between the enthalpy
of vaporisation of the adsorbate and the temperature limit 
of the transition. Furthermore, the impact of the framework
structure on the adsorption-induced contraction is investigated
through the measurement of butane isotherms and differential enthalpy
curves on several DUT-49 analogues. The flexibility modes of the 
linker are shown to play a crucial part in the appearance of 
\gls{NGA}, with rotational hindrance a key factor in increasing
(or lowering) the metastability range. However, two areas where 
advancements in scientific understanding are still required are 
in the quantification of crystal surface potential on the contraction
barrier, as well as a comprehensive theory of the adsorption-induced
stresses in porous materials.

Overall, the fundamental insight obtained from the study of 
DUT-49 may lead to the creation of new materials with counterintuitive
adsorption behaviours but also underlines the inherent compliant
nature of metal organic frameworks. In this context, the 
\textbf{future} of \glspl{MOF} rests in their potential use for
novel applications where flexibility and stimuli response may
lead to previously unreachable targets.

In conclusion, this thesis can be regarded as equal parts a quest for
answers and a search for new challenges in the ever-expanding 
field of metal organic frameworks. The methodology detailed within
can be seen to have applicability to the entire ``chain'' of
\gls{MOF} development, from fundamental studies to industrial optimisation,
but many questions still remain to be answered, trials for future
scientists and engineers alike.

\pagebreak
\bibliographystyle{unsrtnat}
\bibliography{\thisch/biblio/bib}