% !TEX root = ../../main.tex

% Reset graphics to the current folder
\graphicspath{ {\thisch/figures/} }

\chapter*{Conclusion and outlook}\label{conclusion}
\addcontentsline{toc}{chapter}{Conclusion and outlook}

The results in \autoref{pyg} show that, while useful for reference
purposes, published adsorption data contains too much
uncertainty for meaningful insight to be found into the properties
of adsorbent materials starting from ideal chemical characteristics.
For this purpose, high quality laboratory data, recorded in strictly 
controlled environments and conditions is needed, to minimise systematic
error. If standardised measurement procedures are devised for adsorption
experiments, which are then extended by canonical processing methodology 
such as the python processing framework previously described, a
database which can fulfil the accuracy requirements for obtaining
structure-property relationships may be feasible. This kind of database
can be invaluable when used in conjunction with computer simulations
to allow for model confirmation and adjustment. Furthermore, using machine
learning techniques, it may also be possible to quickly predict the
applicability of particular materials for different applications.

However, the work presented in \autoref{pyg} 

Chapter 2 
The challenge lies in combining a high throughput methodology with 
calorimetry. Some efforts are underway to use inexpensive

\cite{wollmannInfrasorbOpticalDetection2012}



\bibliographystyle{unsrtnat}
\bibliography{\thisch/biblio/bib}