% !TEX root = ../../main.tex

\section{Materials and methods}

\subsection{Materials}

The UiO-66(Zr) MOF used in this study was synthesised according
to the procedure described in \autoref{appx:synthesis:uio66def}.
This particular method was chosen to ensure that a structure with
a minimal number of intrinsic defects would be obtained.

The material was then gestated in a solvent solution of modulator
through.

To generate a comprehensive map of all of the potential influence
of different variables on defect concentration, a number of
leeching conditions have been selected. In order to verify the
contribution of the solvent in the leeching process
dimethyl formamide (DMF), water, ethanol and dimethyl sulfoxide (DMSO)
were used. The same monotopic acids which have been successfully used
as modulators in the synthesis of defective UiO-66
(formic acid (FA), acetic acid (AA), trifluoroacetic acid (TFA)
and benzoic acid (BA)) were used in various concentration ranges,
from 1 to 100 equivalents with respect to the BDC linker.
A summary of all generated materials can be found in
\autoref{leeching:tab:samples}

\begin{table}[p]
	\centering
	\caption{Samples used in the UiO-66(Zr) linker leeching study}
	\begin{tabular}{lcccc}
		\toprule
		\textbf{Sample name}
		   & \textbf{Solvent}
		   & \textbf{Modulator}
		   & \textbf{Concentration}
		   & \textbf{Observations}                           \\
		\midrule
		JM & \ce{DMF}               & \ce{FA}  & 1:1   & --- \\
		JM & \ce{DMF}               & \ce{FA}  & 1:5   & --- \\
		JM & \ce{DMF}               & \ce{FA}  & 1:10  & --- \\
		JM & \ce{DMF}               & \ce{FA}  & 1:20  & --- \\
		JM & \ce{DMF}               & \ce{FA}  & 1:100 & --- \\
		JM & \ce{DMF}               & \ce{AA}  & 1:1   & --- \\
		JM & \ce{DMF}               & \ce{AA}  & 1:5   & --- \\
		JM & \ce{DMF}               & \ce{AA}  & 1:10  & --- \\
		JM & \ce{DMF}               & \ce{AA}  & 1:20  & --- \\
		JM & \ce{DMF}               & \ce{AA}  & 1:100 & --- \\
		JM & \ce{DMF}               & \ce{BA}  & 1:1   & --- \\
		JM & \ce{DMF}               & \ce{BA}  & 1:5   & --- \\
		JM & \ce{DMF}               & \ce{BA}  & 1:10  & --- \\
		JM & \ce{DMF}               & \ce{BA}  & 1:20  & --- \\
		JM & \ce{DMF}               & \ce{BA}  & 1:100 & --- \\
		JM & \ce{H2O}               & \ce{FA}  & 1:10  & --- \\
		JM & \ce{H2O}               & \ce{FA}  & 1:100 & --- \\
		JM & \ce{H2O}               & \ce{AA}  & 1:10  & --- \\
		JM & \ce{H2O}               & \ce{AA}  & 1:100 & --- \\
		JM & \ce{H2O}               & \ce{TFA} & 1:10  & --- \\
		JM & \ce{H2O}               & \ce{TFA} & 1:100 & --- \\
		JM & \ce{H2O}               & \ce{BA}  & 1:10  & --- \\
		JM & \ce{H2O}               & \ce{BA}  & 1:100 & --- \\
		JM & \ce{MeOH}              & \ce{FA}  & 1:10  & --- \\
		JM & \ce{MeOH}              & \ce{FA}  & 1:100 & --- \\
		JM & \ce{MeOH}              & \ce{AA}  & 1:10  & --- \\
		JM & \ce{MeOH}              & \ce{AA}  & 1:100 & --- \\
		JM & \ce{MeOH}              & \ce{TFA} & 1:10  & --- \\
		JM & \ce{MeOH}              & \ce{TFA} & 1:100 & --- \\
		JM & \ce{MeOH}              & \ce{BA}  & 1:10  & --- \\
		JM & \ce{MeOH}              & \ce{BA}  & 1:100 & --- \\
		JM & \ce{DMSO}              & \ce{FA}  & 1:10  & --- \\
		JM & \ce{DMSO}              & \ce{FA}  & 1:100 & --- \\
		JM & \ce{DMSO}              & \ce{AA}  & 1:10  & --- \\
		JM & \ce{DMSO}              & \ce{AA}  & 1:100 & --- \\
		JM & \ce{DMSO}              & \ce{TFA} & 1:10  & --- \\
		JM & \ce{DMSO}              & \ce{TFA} & 1:100 & --- \\
		JM & \ce{DMSO}              & \ce{BA}  & 1:10  & --- \\
		JM & \ce{DMSO}              & \ce{BA}  & 1:100 & --- \\
		\bottomrule
	\end{tabular}%
	\label{leeching:tab:samples}
\end{table}%

\subsection{Methods for quantifying defects}

Determining the abundance of defects and characterising their
distribution is a major challenge.

One of the most accessible way of assessing the defectivity of
a MOF is thermogravimetry (TGA). Since through heating in
an oxygen-rich atmosphere, the MOF is normally reduced to
its metal oxide, a stoichiometric analysis of the TGA curve
can allow for the percentage of missing linkers to be
determined. TGA curves for this study were measured under
an air atmosphere using the method described in \autoref{appx:char:TGA}.
The curves are then normalized with respect to the weight at
\SI{600}{\degreeCelsius}, which is assumed to correspond to pure
\ce{ZrO2}. The maximum possible mass loss of a solvent-free
structure is calculated from the ratio of the fully-substituted
metallic cluster \ce{Zr6O4(OH)4(C6H4(COO)2)6}. The plateau
of the TGA curve between \SIrange{400}{500}{\degreeCelsius},
in the range where it is assumed that the molecules included in
the framework are completely evacuated, is used as a measure of the
number of BDC linkers which are missing.

\todo{example curve}

If the distribution of defects can introduce changes in the
long-range order and topology of their parent framework,
such as by the introduction of phase changes,
additional peaks can be observed through regular
powder diffraction techniques.
As~\citeauthor{cliffeCorrelatedDefectNanoregions2014} has
shown~\cite{cliffeCorrelatedDefectNanoregions2014},
the UiO-66 defect-free face-centered unit (\textbf{fcu})
net can be transformed through missing cluster defects into
a \textbf{reo} net. Short-range correlations of such domains
form nanoregions inside the parent structure and generate
diffuse scattering peaks observable at low angles in
powder X-ray (pXRD) diffraction patterns, corresponding to ``forbidden''
reflections in a primitive cubic superstructure.
As such, these pXRD peaks can be used to verify the existence of
missing cluster defects, but only when their abundance allows for
nanodomains of \textbf{reo} nets to be formed. In this study,
pXRD measurements were performed as described in \autoref{appx:char:pxrd}.

In order to check for the inclusion of a modulator in the framework,
the MOF can be digested with the help of a hydrofluoric acid and the
resulting solution can be analysed through proton nuclear magnetic 
resonance (\ce{^{1}H} NMR) or high performance liquid chromatography
(HPLC). In this study, \ce{^{1}H} NMR was used in the method presented
in \autoref{appx:char:nmr} to qualitatively and quantitatively assess
the presence of the capping agents in the UiO-66(Zr) samples. \todo{method}

Finally, both nitrogen adsorption at \SI{77}{\kelvin} and \ce{CO2} 
adsorption at \SI{303}{\kelvin} were used to describe the surface 
characteristics and porosity of the samples. The methods for obtaining
the isotherms are presented in detail in \autoref{appx:char:N2phys}
and \autoref{appx:char:highthroughput} for \ce{N2} and \ce{CO2}
respectively.
