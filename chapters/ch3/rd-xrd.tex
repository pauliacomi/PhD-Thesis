% !TEX root = ../../main.tex

\subsection{Crystalinity of leached samples}\label{def:xrd}

The crystalinity of the leached samples is verified through 
\gls{XRD}. The results in \autoref{appx:def:xrd} confirm that all 
resulting materials retain the same peaks in their powder
diffraction patterns.

\begin{figure}[b]
    \centering
    \includegraphics[width=0.8\textwidth]{xrd/xrd-h2o}
    \caption{Diffuse scattering peaks in the \ce{H2O} leached 
    samples, highlighted by black arrows.
    }\label{def:fig:xrd-defects}
\end{figure}\

A closer look at the low angle scattering (\autoref{def:fig:xrd-defects})
reveals the appearance of diffuse peaks, which are marked in the 
figure by black arrows, which correspond to forbidden
reflections of the (\textbf{reo}) phase. They confirm that the 
leached samples begin to exhibit phase coexistence of 
the original UiO-66(Zr) unit cell and the missing cluster
(\textbf{reo}) net. These peaks have the highest intensity in 
the \gls{TFA} treated materials, suggesting that it has the highest 
capability of introducing missing cluster defects.