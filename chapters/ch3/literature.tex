% !TEX root = ../../main.tex

\section{The defective nature of MOFs}

From a crystallographic point of view, defects can be described
as features which suspend the order of components in an ideally 
regular lattice. The building blocks in the case of MOFs can be either
individual atoms, molecules or other higher order structural building
units (SBUs). Any change with leads to local breaking of symmetry with
respect to the original structure can be viewed as a defect.

With respect to dimensionality, defects can be described 
as point defects, line defects (such as edge dislocations), 
plane defects (such as grain boundaries or stacking faults) and bulk 
defects (macroscopic voids, phase coexistence).

Broadly, we can refer to several types of point defects: 
\textbf{substitutional defects}, where an existing building unit is 
replaced or transformed into another, \textbf{inclusion} or 
\textbf{interstitial defects}, where a foreign component
or building block is included in the framework and \textbf{vacancy defects}
where one of the lattice sites is unoccupied.

In the context of MOFs, the same general categories of defects 
apply. However, due to the higher degrees of freedom available in 
these compounds, defects which are rarely seen in other materials 
can be regularly encountered. 
In most cases of point defects, the charge neutrality of the framework
is maintained through coordination of available counterions or solvent 
molecules. Through thermal activation, these weakly coordinated 
molecules can be removed to expose metal coordinatively unsaturated sites
(CUS) and increase the overall reactivity of the compound.

Vacancy defects are encountered through missing linker and 
missing cluster defects. 

Substitution defects 

A special case of substitutional defects which are present in MOFs 
are mixed valence defects. When metals with multiple stable oxidation
states, such as \ce{Cu (I-II)}, \ce{Fe (II-III)} etc.\ are part of 
the framework, a change in their oxidation state can occur. This has
been shown to be an integral part of copper paddlewheel and 
iron trimesate containing 
MOFs~\cite{yoonControlledReducibilityMetalOrganic2010} and can even be 
seen with the naked eye, as such defects have been shown 
to give HKUST-1 its common blue colour~\cite{mullerDefectsColorCenters2017}.

Another common feature of structured high porosity compounds is 
interpenetration. While not a defect in the classical sense, 
it has important effects on their properties.
With a large enough pore size, a secondary lattice can form in the
pore voids of the primary one. This imposes a limit on the 
common design strategy of isoreticular synthesis, but can 
introduce new features such as better adsorption through 
confinement, increase in active site count and even flexibility.

Finally, the surface of the MOF can also be regarded as a 
boundary or plane defect. The nature of the surface plays a role
in intra-particle interactions, important when considering 
the agglomeration behaviour or inclusion of crystals in 
a membrane~\cite{seminoMicroscopicModelMetal2016}. 
Crystal size effects can also be through of as a consequence
of surface characteristics, more specifically, of surface-to-volume
ratio. When considering soft materials,
~\cite{krauseEffectCrystalliteSize2018, %
vanduyfhuysThermodynamicInsightStimuliresponsive2018} it has 
been shown that their flexible behaviour depends 

defect types
- missing linkers
- missing cluster

- substitutions - mixed metal, mixed mof
- inclusions - metal particles
- surfaces

: photo - defect types

influence of properties
- increase of porosity
- exposing metal sites
- decrease in stability 

applications of defects
- increase capacity
- increase diffusion
- generate specific interactions
- expose or include reactive sites
-

Of course, defects are also problematic. Most metal organic frameworks
synthesised to date suffer from poor stability. Even if the stability
of the component parts is not an issue, the framework may not have 
enough structural stability to be able to sustain itself when fully
evacuated. The activation process itself can lead to framework 
collapse, due to the forces encountered in guest removal, necessitating 
complex activation processes, such as supercritical drying or 
preliminary solvent exchange. After activation, the metal-ligand bond
is susceptible to attack by adsorbed species. Here, defects in the framework 
have been shown to play a major role in its stability (or lack thereof).
~\cite{burtchWaterStabilityAdsorption2014}

introducing defects

The UiO-66(Zr) MOF and its derivatives are well known due to their thermal and chemical 
stability~\cite{cavkaNewZirconiumInorganic2008}. It is composed of
\ce{[Zr6O4(OH)4]^12+} clusters which are connected with benzene dicarboxilate (BDC) linkers to form
a face-centered cubic framework. It has shown promise~\cite{wiersumEvaluationUiO66GasBased2011}
in use for gas adsorption and catalytic applications.