% !TEX root = ../../main.tex

\FloatBarrier%
\pagebreak

\section{Conclusion}

In summary, a new method of generating missing linker and missing
cluster type defects in a UiO-66(Zr) framework has been presented.
This type of post-synthetic modification is an alternative to 
modulated synthesis, and can be useful when crystal growth is
not desirable. It also introduces new possibilities of framework
functionalisation, through sequential rounds of leaching and 
defect healing. It does, however, come at a cost, as it also 
partially dissolves the material.

A few trends emerge in regards to the capacity of different solvents 
and acids to induce defect formation.

\begin{itemize}
    \item The impact of the monotopic acid is mainly due to its 
    acidity, with pKa a being a good predictor of the 
    resulting defectivity.
    \item \gls{DMSO} and water stand out as the most effective solvents
    for defect generation.
    \item In general, the leached samples have a larger surface 
    area and pore volume, due to both missing linker and cluster
    defects. Benzoic acid is a special case where 
    due to its large molecule, may instead result in a loss of
    available pore space.
    \item High concentrations of acids may induce mesoporosity
    through large-scale defect generation. These conditions also
    make the material susceptible to complete 
    structural breakdown through attack at the crystal surface.
    \item The resulting defective structures may have different 
    bulk properties such as thermal and mechanical resistance due 
    to the influence of the defect capping agents.
    \item The coordinated acids can also have an impact on the
    adsorption properties through changes to the pore environment.

\end{itemize}

These results also show the vast influence that defects have on the 
adsorption properties of \glspl{MOF}. From different interactions with 
the material surface to vastly different capacities and, in some 
cases, even changing the sizes of pores or introducing new porosity.
When attempting to determine the adsorption performance and behaviour
of a compound it is important to consider whether the material in 
question is representative of its stoichiometric composition or,
as is more often the case, a complex interplay of ideal and 
non-ideal domains.

\pagebreak