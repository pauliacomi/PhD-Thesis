% !TEX root = ../../main.tex

\FloatBarrier%
\pagebreak

\section{Conclusion}

In summary, a new method of generating missing linker and missing
cluster type defects in a UiO-66(Zr) framework has been presented.
This type of post-synthetic modification is an alternative to 
modulated synthesis, and can be useful when crystal growth is
not desirable. It also introduces new possibilities of framework
functionalisation, through sequential rounds of leaching and 
defect healing.

A few trends emerge in regards to the capacity of different solvents 
and acids to induce defect formation.

\begin{itemize}
    \item The impact of the monotopic acid is mainly due to its 
    acidity, with pKa a being a good predictor of the 
    resulting defectivity.
    \item Bulky modulators suffer from kinetic effects, as seen
    with benzoic acid being the least effective in inducing 
    defect formation.
    \item As the acid replaces the
    terephtalate linker in an 2:1 ratio, a critical
    concentration exists after which no further leaching of 
    BDC happens. In solvents where the solubility of 
    the linker is low, higher amount of modulator are 
    needed to shift the equilibrium.
    \item High concentrations of acids may also induce complete 
    structural breakdown through attack at the crystal surface.
    \item The resulting defective structures may have different 
    bulk properties such as thermal and mechanical resistance due 
    to the influence of the defect capping agents.
    \item The capping 

\end{itemize}

These results also show the vast influence that defects have on the 
adsorption properties of MOFs. From different interactions with 
the material surface to vastly different capacities and, in some 
cases, even changing the sizes of pores or introducing new porosity.
When transitioning a material from a lab-scale to an application
it is important to consider that 