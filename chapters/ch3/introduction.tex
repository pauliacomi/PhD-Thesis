% !TEX root = ../../main.tex

\section{Introduction}

Ideality is a rarely encountered phenomenon in nature. In fact,
experience has shown that deliberate and concerted efforts 
must be made in order to obtain states of matter that have all
the properties described by an ``ideal state''.
Often the divergence is small and can be approximated 
away, as is the case when using the ideal equation of state for 
of noble gasses or ascribing the thermal radiation emitted by an object
to a black-body spectrum. It is, however, in non-ideality where the 
fascinating complexity of the world asserts itself and where 
advances in our understanding can be made. 

When considering crystals, the ideal description of an infinite
periodic repetition of building blocks in three dimensional space is
not applicable in the real world. From the necessary existence of 
crystal boundaries to the possible presence of other structural
irregularities, these so called crystal ``defects'' can have
varying effects on the bulk properties of the material.
Their presence is not necessarily a fault as
they often impart the material with beneficial characteristics
on multiple length scales.
An understanding of the interplay of defects in the 
lattices of alloys is an art in itself.
Crystal grain size and presence of additives such as 
carbon and nickel can completely change the hardness, 
ductility, tensile strength and corrosion resistance
of steel~\cite{reed-hillPhysicalMetallurgyPrinciples1992}. 
Various other forms 
of disorder can introduce completely new behaviours altogether.
The insertion of foreign elements into the crystal structure
of semiconductor materials such as silicon can alter its 
electronic properties and is responsible for the ubiquitousness
of computing devices based on the 
transistor~\cite{levyMicroelectronicMaterialsProcesses1989}. This application
of defects can be said to have ushered in the modern digital age,
just as the creation of steel led to the industrial revolution.
Other such emergent properties exist such as high temperature 
superconductivity~\cite{leggettWhatWeKnow2006} or thermoelectric 
behaviour~\cite{peiBandEngineeringThermoelectric2012} which are influenced
by defect-related spin effects or defect-mediated charge transfer.
As such, the kind of heterogeneity afforded by defects is a key attribute 
in condensed matter physics and a target for material design.
The scientific foray in their control, either through generation or
by inhibition can be termed ``defect engineering''.

As porous coordination polymers (PCPs) and their subclasses
metal organic frameworks (MOFs) and covalent organic frameworks (COFs)
became prominent topics of study in the field of porous materials,
the presence and desirability of defects in their ordered crystal
structures has been put into question. Investigation of the 
propensity of these compounds to form defects has suggested 
potential benefits in the fields of 
catalysis~\cite{shollDefectsMetalOrganic2015}, gas storage
and separation~\cite{%
    choiRoleStructuralDefects2018,%
    ghoshWaterAdsorptionUiO662014,%
    liSelectiveGasAdsorption2009%
}, and has even alluded to applications in
sensing and optoelectronics~\cite{cliffeMetalOrganicNanosheets2017}.
Furthermore, MOFs
such as UiO-66 and similar oxo-Zr coordination compounds have been 
shown to have intrinsically defective structures, where clustering 
and correlation of defects is in fact 
inevitable~\cite{cliffeCorrelatedDefectNanoregions2014}.
The engineering of defects in PCPs is currently the focus of many
initiatives as it is seen as a highly desirable method of tuning 
their properties through judicious design
~\cite{
    shollDefectsMetalOrganic2015,%
    bennettInterplayDefectsDisorder2016,%
    liangLinkingDefectsHierarchical2018%
}.

From the point of view of describing adsorption in 
such materials, the presence of defects introduces a conundrum.
If a MOF can be defective, and is often intrinsically so,
a slight change in synthesis conditions can introduce a 
large variability in its properties, and obtaining a 
``standard'' isotherm may be a challenge. A large scale
meta analysis of the correlation between material structure
and its adsorption performance cannot be achieved unless
the contribution of defects is assessed for each material.

\subsection*{Chapter summary}

In this chapter we explore the kind of changes in 
adsorption behaviour introduced through defect engineering with 
the high throughput processing tools presented in \autoref{pyg}.
First, the range of crystal defects that can be found in MOFs 
is presented. A summary of the known methods for controlling defects in
such materials is discussed, as well as the known impact on 
different properties.
Particular focus is placed on the zirconium variant of UiO-66,
due to its aforementioned remarkable stability to the presence of defects.
An alternative approach to defect generation in this MOF is explored,
through induced leaching of linkers in a solvent solution of monotopic
acids which have been shown to induce defect formation when present
during synthesis. The influence on defect type and preponderence 
of the acid, its concentration and 
the solvent itself is investigated, through the changes in 
adsorption behaviour.

\subsection*{Contributions}

This work was in done in collaboration with the DEFNET project partner
at the Centre for Surface Chemistry and Catalysis in KU Leuven. 
The study was designed by Paul Iacomi and João Marreiros during 
the secondment undertaken in COK as part of this project.
Two PWI master students, Giel Arnauts and Wouter Arts helped with
the synthesis, leaching and characterisation of the first trial batch.
João Marreiros conducted the synthesis, leaching, XRD and NMR of
further batches in Leuven. Paul Iacomi performed characterisation
through TGA and all physisorption measurements in Marseille.
Prof.\ Philip Llewellyn and Prof.\ Rob Ameloot provided critical
input in the direction of the study and the significance of the 
results.