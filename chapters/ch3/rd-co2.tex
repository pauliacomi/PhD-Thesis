% !TEX root = ../../main.tex

\subsection{Carbon dioxide isotherms}\label{def:co2}

The adsorption of carbon dioxide has been shown to be a good 
predictor for the presence of defects from comprehensive simulations
by \citeauthor{thorntonDefectsMetalOrganic2016}.
In their work~\cite{thorntonDefectsMetalOrganic2016} they have shown
that in formate-capped defects, a shift of the \ce{CO2} isotherm towards
higher loading is seen at pressures over \SI{5}{\bar}, accompanied 
by a lower capacity at low pressures.

Carbon dioxide isotherms on the x sample set is presented in 
figure y. It can be seen that only the formic acid-treated
samples show higher capacity throughout the measured pressure range.
It is likely that the activation which was performed at 
\SI{200}{\degreeCelsius} was insufficient for complete removal of
the coordinated acids.