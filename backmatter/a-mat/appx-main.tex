% !TEX root = ../../main.tex

% Reset graphics to the current folder
\graphicspath{ {\thisappx/figures/} }

\chapter{Synthesis method of referenced materials}%
\label{appx:synthesis}

\section{Takeda 5A reference carbon}%
\label{appx:synthesis:takeda}

The Takeda 5A carbon was purchased directly from the Takeda corporation.
It is a highly microporous material shaped as spheres with an average
diameter of \SI{1}{\milli\metre}.
The sample was activated at \SI{250}{\celsius} under 
secondary vacuum (\SI{5}{\milli\bar}) before any measurements.

\section{MCM-41 mesoporous silica}%
\label{appx:synthesis:mcm41}

MCM-41 (Mobil Composition of Matter No. 41) is a mesoporous silica 
(\ce{SiO2}) material with a narrow pore distribution. First synthesised 
by the Mobil Oil Corporation, it is produced through templated 
synthesis using micelle-forming surfactants.
The material referenced in this thesis was purchased from Sigma-Aldrich.
The activation procedure consists of heating at \SI{250}{\celsius} under 
secondary vacuum (\SI{5}{\milli\bar}).

\section{Zr fumarate MOF}%
\label{appx:synthesis:zrformate}

The synthesis of the Zr fumarate was performed in Peter Behren's 
group in Hannover, through modulated synthesis. This MOF can only
be synthesised through the addition of a modulator, in this case
fumaric acid, to the ongoing reactor, as detailed in the 
original publication~\cite{wissmannModulatedSynthesisZrfumarate2012}.

The procedure goes as follows: \ce{ZrCl4}
(\SI{0.517}{\milli\mol}, 1 eq) and fumaric acid 
(\SI{1.550}{\milli\mol}, 3 eq) are dissolved 
in \SI{20}{\milli\liter} N,N-dimethylformamide (DMF) 
and placed in a \SI{100}{\milli\liter} glass flask at room 
temperature. 20 equivalents of formic acid were added.
The glass flasks were Teflon-capped and heated in an oven at
\SI{120}{\degreeCelsius} for \SI{24}{\hour}. After cooling, 
the white precipitate was washed with \SI{10}{\milli\liter} 
DMF and \SI{10}{\milli\liter} ethanol, respectively. 
The washing process was carried out by centrifugation and 
redispersion of the white powder, which was then
dried at room temperature over night



\section{UiO-66(Zr) for defect study}%
\label{appx:synthesis:uio66def}

The UiO-66(Zr) sample preparation was adapted from Shearer et al.
~\cite{shearerTunedPerfectionIroning2014} as follows:
\ce{ZrCl4} (\SI{1.55}{\gram}, \SI{6.65}{\milli\mol}), an excess 
of terephtalic acid (BDC)
(\SI{1.68}{\gram}, \SI{10.11}{\milli\mol}), \ce{HCl} 37 \% solution 
(\SI{0.2}{\milli\liter}, \SI{3.25}{\milli\mol}) and N,N’-dimethylformamide
(DMF) (\SI{200}{\milli\liter}, \SI{2.58}{\mol}) were added to a 
\SI{250}{\milli\liter} pressure resistant Schott bottle. The mixture 
was stirred for 10 min, followed by incubation in a convection oven 
at \SI{130}{\celsius} for \SI{24}{\hour}. The resulting white 
precipitate was washed with fresh DMF (\(3 \times \) \SI{50}{\milli\liter}) 
followed by ethanol (\(3 \times \) \SI{50}{\milli\liter})
over the course of 48 h and dried at \SI{60}{\celsius}. 
After drying, the sample was activated 
on a vacuum oven by heating at \SI{200}{\celsius} under vacuum for 12 h. 
The yield was 78 \% white microcrystalline powder. Before the 
experiment, the sample was calcined at \SI{200}{\celsius} under
vacuum (\SI{5}{\milli\bar}) to remove any residual solvents
from the framework.

\section{UiO-66(Zr) for shaping study}%
\label{appx:synthesis:uio66shaping}

The scaled-up synthesis of UiO-66(Zr) was carried out in 
a \SI{5}{\liter} glass reactor (Reactor Master, Syrris, equipped with 
a reflux condenser and a Teflon-lined mechanical stirrer)
according to a previously reported 
method~\cite{ragonSituEnergyDispersiveXray2014}.
In short, \SI{462}{\gram} (\SI{2.8}{\mol}) of \ce{H2BDC} (98\%) was 
initially dissolved in \SI{2.5}{\liter} of dimethyl formamide (DMF, 
\SI{2.36}{\kilo\gram}, \SI{32.3}{\mol}) at room temperature. 
Then, \SI{896}{\gram} (\SI{2.8}{\mol}) of \ce{ZrOCl2 * $8$H2O}
(98\%) and \SI{465}{\milli\liter} of 37\% \ce{HCl} 
(\SI{548}{\gram}, \SI{15}{\mol}) were added to the mixture. 
The molar ratio of the final \ce{ZrOCl2 * $8$H2O/H2BDC/DMF/HCl} 
mixture was 1 : 1 : 11.6 : 5.4. The reaction mixture was vigorously
stirred to obtain a homogeneous gel. The mixture was then heated
to \SI{423}{\kelvin} at a rate of \SI{1}{\kelvin\per\minute}
and maintained at this temperature for \SI{6}{\hour} in the
reactor without stirring, leading to a crystalline UiO-66(Zr) solid.
The resulting product (\SI{510}{\gram}) was recovered from the 
slurry by filtration, redispersed in \SI{7}{\liter} of DMF at 
\SI{333}{\kelvin} for \SI{6}{\hour} under stirring, 
and recovered by filtration. 
The same procedure was repeated twice, using methanol (MeOH) 
instead of DMF. The solid product was finally dried at 
\SI{373}{\kelvin} overnight.

\section{MIL-100(Fe) for shaping study}%
\label{appx:synthesis:mil100shaping}

The synthesis of the MOF for the shaping study was 
done at the KRICT institute using a previously
published method~\cite{jeremiasAmbientPressureSynthesis2016}.
To synthesise the MIL-100(Fe) material 
\ce{Fe(NO3)3} was completely dissolved in water.
Then, trimesic acid (BTC) was added to the
solution; the resulting mixture was stirred at room temperature 
for 1h. The final composition was \ce{Fe(NO3)3*9H2O}:0.67 BTC:\ce{$n$H2O} 
(x= 55–280). The reactant mixture was heated at \SI{433}{\kelvin} for
\SI{12}{\hour} using a Teflon-lined pressure vessel. The synthesized 
solid was filtered and washed with deionized (DI) water.
Further washing was carried out with DI water and ethanol at
\SI{343}{\kelvin} for \SI{3}{\hour} and purified with a 
38 mM \ce{NH4F} solution at \SI{343}{\kelvin} for \SI{3}{\hour}. 
The solid was finally dried overnight at less than 
\SI{373}{\kelvin} in air.

\section{MIL-127(Fe) for shaping study}%
\label{appx:synthesis:mil127shaping}

MIL-127(Fe) was synthesized by reaction of
\ce{Fe(ClO4)3*6H2O} (\SI{3.27}{\gram}, \SI{9.2}{\milli\mol}) and 
\ce{C16N2O8H6} (\SI{3.3}{\gram}) in DMF (\SI{415}{\milli\liter}) 
and hydrofluoric acid (5 M, \SI{2.7}{\milli\liter}) at 
\SI{423}{\kelvin} in a Teflon
flask. The obtained orange crystals were placed in 
DMF (\SI{100}{\milli\liter}) and stirred at ambient temperature for 
\SI{5}{\hour}. The final product was kept at 
\SI{375}{\kelvin} overnight. MIL-127(Fe) was synthesized by reaction of
\ce{Fe(ClO4)3*6H2O} (\SI{3.27}{\gram}, \SI{9.2}{\milli\mol}) and 
\ce{C16N2O8H6} (\SI{3.3}{\gram}) in DMF
(\SI{415}{\milli\liter}) and hydrofluoric acid (5 M, 
\SI{2.7}{\milli\liter}) at \SI{423}{\kelvin} in a Teflon 
flask. The obtained orange crystals were placed in DMF 
(\SI{100}{\milli\liter}) and stirred at ambient temperature 
for \SI{5}{\hour}. The final product was
kept at \SI{375}{\kelvin} overnight.

\bibliographystyle{unsrtnat}
\bibliography{backmatter/biblio/bib}