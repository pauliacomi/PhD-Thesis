% !TEX root = ../../main.tex

% Reset graphics to the current folder
\graphicspath{ {\thisappx/figures/} }

\chapter{Common characterisation techniques}\label{appx:char}

\section{Thermogravimetry}\label{appx:char:TGA}

Thermogravimetry (TGA) is a standard laboratory technique where the
weight of a sample is monitored while ambient temperature is controlled.
Changes in sample mass can be correlated to physical events, such
as adsorption, desorption, sample decomposition or oxidation, depending
on temperature and its rate of change.

TGA experiments are carried out on approximately \SI{15}{\milli\gram} of 
sample with a TA Instruments Q500 up to \SI{800}{\degreeCelsius}.
The sample is placed on a platinum crucible and sealed in a temperature
controlled oven, under a total gas flow of \SI{100}{\cm\cubed\per\minute}.
Experiments can use a blanket of either air or argon. The temperature
ramp can be specified directly and should be chosen to ensure 
that the sample is in equilibrium with the oven temperature and no
thermal conductivity effects come into play. Alternatively,
a dynamic “Hi-Res” mode can be used which allows for automatic
cessation of heating rate while the sample undergoes mass loss.

The main purpose of thermogravimetry as used in this thesis is the
determination of sample decomposition temperature, to ensure
that thermal activation prior to adsorption is complete and 
that all guest molecules have been removed without loss of
structure. To this end, experiments are performed under an inert
atmosphere (argon), and the sample activation temperature is chosen
as \SIrange{50}{100}{\degreeCelsius} lower than the sample 
decomposition temperature.

\section{Bulk density determination}\label{appx:char:bulkdensity}

Bulk density is a useful metric for the industrial use of adsorbent
materials, as their volume plays a critical role in equipment sizing.

Bulk density is determined by weighing \SI{1.5}{\milli\litre} empty 
glass vessels and settling the MOFs inside. Powder materials are then added
in small increments and settled through vibration between each
addition. The full vessel is finally weighed, which allowed the bulk
density to be determined. The same cell is used in all experiments,
with cleaning through sonication between each experiment.

\section{Skeletal density determination}\label{appx:char:truedensity}

True density or skeletal density is determined through gas pycnometry
in a MicrotracBEL BELSORP-max apparatus. Helium is chosen
as the fluid of choice as it is assumed to be non-adsorbing.

The volume of a glass sample cell (\(V_c\)) is precisely measured through
dosing of the reference volume with helium up to (\(p_1\)), then opening the valve 
connecting the two and allowing the gas to expand up to (\(p_2\)). 
Afterwards approximately \SI{50}{\milli\gram} of sample are weighed
and inserted in a glass sample cell. After sample activation using the
supplied electric heater to ensure no solvent residue is left in the pores,
the same procedure is repeated to determine the volume of the cell
and the adsorbent. With the volume of the sample determined, the
density can be calculated by:
%
\begin{equation}
%
    V_s = V_c + \frac{V_r}{1- \frac{p_1}{p_2}}
%
\end{equation}


\section{Nitrogen physisorption at \SI{77}{\kelvin}}\label{appx:char:N2phys}

Nitrogen adsorption experiments are carried out on a 
Micromeritics Triflex apparatus. Approximately \SI{60}{\milli\gram}
of sample are used for each measurement.
Empty glass cells are weighed and filled with the samples, which 
are then activated in a Micromeritics Smart VacPrep up to 
their respective activation
temperature under vacuum and then back-filled with an inert
atmosphere. After sample activation, the cells are re-weighed
to determine the precise sample mass. The cells are covered with
a porous mantle which allows for a constant temperature gradient
during measurement by wicking liquid nitrogen around the
cell. Finally, the cells are immersed in a liquid nitrogen bath
and the adsorption isotherm is recoded using the volumetric
method. A separate cell is used to condense the adsorptive 
throughout the measurement for accurate determination of
its saturation pressure.

\section{Vapour physisorption at \SI{298}{\kelvin}}\label{appx:char:vapourphys}

Vapour adsorption isotherms throughout this work are measured
using a MicrotracBEL BELSORP-max apparatus in vapour mode.
Glass cells are first weighed and then filled with about
\SI{50}{\milli\gram} of sample. The vials are then heated
under vacuum up to the activation temperature of the material
and re-weighed in order to measure the exact sample mass
without adsorbed guests. The cells are then immersed in a
mineral oil bath kept at \SI{298}{\kelvin}. To ensure that the 
cold point of the system occurs in the material and to 
prevent condensation on cell walls, the reference volume, dead space 
and vapour source are temperature controlled through an
insulated enclosure.

\section{Gravimetric isotherms}\label{appx:char:gravimetry}

The gravimetric isotherms in this thesis are obtained 
using a commercial Rubotherm GmbH balance. Approximately 
\SI{1}{\gram} of dried sample is used for these experiments. 
Samples are activated in situ by heating under vacuum. 
The gas is introduced using a step-by-step method, and equilibrium is
assumed to have been reached when the variation of weight remained
below \SI{30}{\micro\gram} over a \SI{15}{\minute} interval. 
The volume of the sample is determined from a blank experiment 
with helium as the non-adsorbing
gas and used in combination with the gas density measured by the
Rubotherm balance to compensate for buoyancy.

\section{High throughput isotherm measuremnt}\label{appx:char:highthroughput}

A high-throughput gas adsorption apparatus is presented for the 
evaluation of adsorbents of interest in gas storage and
separation applications. This instrument is capable of measuring 
complete adsorption isotherms up to \SI{50}{\bar} on six samples in 
parallel using as little as 60 mg of material. Multiple adsorption 
cycles can be carried out and four gases can be used sequentially, 
giving as many as 24 adsorption isotherms 
in 24 h~\cite{wiersumExperimentalScreeningPorous2013}.

\section{Powder X-ray diffraction}\label{appx:char:pxrd}

Two XRD devices have been used to record the diffraction patterns.
Initial patterns were obtained on a STOE COMBI P diffractometer 
(monochromated Cu \(K_{\alpha 1}\)-radiation, \(\lambda= 1.54060 \si{\angstrom} \)) 
equipped with an IP-PSD detector in Bragg-Brentano transmission geometry.
Patterns were acquired in 0.05 \(2\theta \) steps, with a 1s/step acquisition time.
Latter patterns were recorded using a Malvern PANalytical Empyrean 
(monochromated Cu \(K_{\alpha 1}\)-radiation, \(\lambda= 1.54060 \si{\angstrom} \))
equipped with a Pixcel3D detector in Bragg-Brentano transmission geometry.
The patterns were recorded in 0.05 \(2\theta \) steps, with 
0.25s/step acquisition time.

\section{Adsorption manometry and calorimetry at \SI{303}{\kelvin}}\label{appx:char:ambient-calo}

\bibliographystyle{unsrtnat}
\bibliography{backmatter/biblio/bib}