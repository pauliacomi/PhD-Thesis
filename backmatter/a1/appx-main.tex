% !TEX root = ../../main.tex

% Reset graphics to the current folder
\graphicspath{ {\thisappx/figures/} }

\chapter{Common characterisation techniques}

\section{Thermogravimetry}\label{appx:char:TGA}

TGA experiments were carried out on ca. \SI{15}{\milli\gram} sample with a Q500 
(TA Instruments) apparatus in a dynamic “Hi-Res” mode using \SI{3}{\kelvin\per\minute}
under both nitrogen and argon flow at
\SI{30}{\cm\cubed\per\minute}. For comparison, the curves have been normalized at 
\SI{800}{\degreeCelsius}.

\section{Bulk density determination}

Bulk density was determined by weighing \SI{1.5}{\milli\litre} empty vessels and
settling the powder or the spheres of the MOFs inside. Powders have been added by small
increments and settled through vibration between each
addition. We finally weighed the full vessels, which allowed the bulk
density to be determined. The experiments were recorded with the
same vessel after cleaning.

\section{Nitrogen physisorption at \SI{77}{\kelvin}}

Nitrogen physisorption at \SI{77}{\kelvin} was used to calculate BET areas and accessible pore 
volumes~\cite{rouquerolAdsorptionPowdersPorous2013}. 
Approximately \SI{60}{\milli\gram} of sample were used for each measurement. 
The adsorption experiments were carried out on a Micromeritics Triflex apparatus.
The BET area on these microporous solids was calculated using the 
procedure devised by \citeauthor{rouquerolAdsorptionPowdersPorous2013} Accessible pore
volume was calculated from the amount adsorbed at \(p/p^0 = 0.2\). 
The pore sizes were calculated by applying the Dollimore-Heal method on the desorption
branch of the isotherm.

\section{Gravimetric isotherms measured on Rubotherm Balance}

The gravimetric isotherms in this study were obtained 
using a commercial balance (Rubotherm GmbH). Approximately 
\SI{1}{\gram} of dried sample was used for these experiments. Samples was activated
in situ by heating under vacuum. 
The gas was introduced using a step-by-step method, and equilibrium was
assumed to have been reached when the variation of weight remained
below \SI{30}{\micro\gram} over a \SI{15}{\minute} interval. The volume of the sample was
determined from a blank experiment with helium as the nonadsorbing
gas and used in combination with the gas density measured by the
Rubotherm balance to compensate for buoyancy.

\section{Vapour adsorption}

Adsorption of water or ethanol vapour was measured at \SI{298}{\kelvin} on a BELmax apparatus
(MicrotracBEL, Japan). Approximately \SI{50}{\milli\gram} of material was used for each
experiment.
