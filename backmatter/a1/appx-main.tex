% !TEX root = ../../main.tex

% Reset graphics to the current folder
\graphicspath{ {\thisappx/figures/} }

\chapter{Common characterisation techniques}

\section{Thermogravimetry}\label{appx:char:TGA}

Thermogravimetry (TGA) is a standard laboratory technique where the
weight of a sample is monitored while ambient temperature is controlled.
Changes in sample mass can be correlated to physical events, such
as adsorption, desorption, sample decomposition or oxidation, depending
on temperature and its rate of change.

TGA experiments were carried out on approximately \SI{15}{\milli\gram} of 
sample with a Q500 (TA Instruments) up to \SI{800}{\degreeCelsius}.
The sample is placed on a platinum crucible and sealed in a temperature
controlled oven, under gas flow of \SI{40}{\cm\cubed\per\minute}.
Experiments can use a blanket of either air or argon. The temperature
ramp can be specified directly and should be chosen to ensure 
that the sample is in equilibrium with the oven temperature and no
thermal conductivity effects come into play. Alternatively,
a dynamic “Hi-Res” mode can be used which allows for automatic
cessation of heating rate while the sample undergoes mass loss.

The main purpose of thermogravimetry as used in this thesis is the
determination of sample decomposition temperature, to ensure
that thermal activation prior to adsorption is complete and 
that all guest molecules have been removed without loss of
structure. To this end, experiments are performed under an inert
atmosphere (argon), and the sample activation temperature is chosen
as \SIrange{50}{100}{\degreeCelsius} lower than the sample 
decomposition temperature.

\section{Bulk density determination}

Bulk density is a useful metric for the industrial use of adsorbent
materials, as their volume plays a critical role in equipment sizing.

Bulk density was determined by weighing \SI{1.5}{\milli\litre} empty vessels and
settling the powder or the spheres of the MOFs inside. Powders have been added by small
increments and settled through vibration between each
addition. We finally weighed the full vessels, which allowed the bulk
density to be determined. The experiments were recorded with the
same vessel after cleaning.

\section{Skeletal density determination}

Bulk density was determined by weighing \SI{1.5}{\milli\litre} empty vessels and
settling the powder or the spheres of the MOFs inside. Powders have been added by small
increments and settled through vibration between each
addition. We finally weighed the full vessels, which allowed the bulk
density to be determined. The experiments were recorded with the
same vessel after cleaning.

\section{Nitrogen physisorption at \SI{77}{\kelvin}}

Nitrogen physisorption at \SI{77}{\kelvin} was used to calculate BET areas and accessible pore 
volumes~\cite{rouquerolAdsorptionPowdersPorous2013}. 
Approximately \SI{60}{\milli\gram} of sample were used for each measurement. 
The adsorption experiments were carried out on a Micromeritics Triflex apparatus.
The BET area on these microporous solids was calculated using the 
procedure devised by \citeauthor{rouquerolAdsorptionPowdersPorous2013} Accessible pore
volume was calculated from the amount adsorbed at \(p/p^0 = 0.2\). 
The pore sizes were calculated by applying the Dollimore-Heal method on the desorption
branch of the isotherm.

\section{Gravimetric isotherms measured on Rubotherm Balance}

The gravimetric isotherms in this study were obtained 
using a commercial balance (Rubotherm GmbH). Approximately 
\SI{1}{\gram} of dried sample was used for these experiments. Samples was activated
in situ by heating under vacuum. 
The gas was introduced using a step-by-step method, and equilibrium was
assumed to have been reached when the variation of weight remained
below \SI{30}{\micro\gram} over a \SI{15}{\minute} interval. The volume of the sample was
determined from a blank experiment with helium as the nonadsorbing
gas and used in combination with the gas density measured by the
Rubotherm balance to compensate for buoyancy.

\section{Vapour adsorption}

Adsorption of water or ethanol vapour was measured at \SI{298}{\kelvin} on a BELmax apparatus
(MicrotracBEL, Japan). Approximately \SI{50}{\milli\gram} of material was used for each
experiment.
