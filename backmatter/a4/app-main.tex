% !TEX root = ../../main.tex

% Reset graphics to the current folder
\graphicspath{ {\thisappx/figures/} }

\chapter{Calculation of uncertainty in adsorption measurements}%
\label{appx:uncertainty}

To obtain the experimental errors, the procedure described in 
the Guide to the Expression of Uncertainty in Measurement (REF )
was used. The quantity, enthalpy or pressure, is first was expressed
as a function f(y)f of other physical measured quantities through a
functional relationship (\autoref{appx:uncert:eqn:general}). 
The standard uncertainty (u_c (y))
is then calculated on the basis of \autoref{appx:uncert:eqn:standarduncert}, where u_i (x_i ) is
the standard uncertainty in each input quantity. Here it is assumed 
that the input quantities are independent and uncorrelated. 
The error margins (a_i) for each quantity were taken from manufacturer 
specifications of the equipment used for recording. They were then divided
by a value k_i k chosen to cover the expected variance in that quantity,
as each variable is assumed to be characterized by a probability 
distribution. The error introduced by the equation of state used 
(NIST REFPROP (REF )) were assumed to be minor compared to the error 
introduced by the physical quantities, with the same to be 
said regarding the error in the calorimetric heat signal, which 
represents less than 1\% of the error in enthalpy. Finally, the expanded 
uncertainty was calculated by choosing a suitable coverage factor of
1.645, corresponding to a 95\% confidence interval.

\begin{equation}\label{appx:uncert:eqn:general}
    f(y)=f(N_1,N_2 \cdots N_i)
\end{equation}

\begin{equation}\label{appx:uncert:eqn:standarduncert}
    u_c(y) = \sqrt( \sum_(i=1)^N {\Big( \frac{\partial f(y)}{\partial x_i} u_i x_i  \Big)}^2 )
\end{equation}

\bibliographystyle{unsrtnat}
\bibliography{backmatter/biblio/bib}